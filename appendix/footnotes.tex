% SVN info for this file
\svnidlong
{$HeadURL$}
{$LastChangedDate$}
{$LastChangedRevision$}
{$LastChangedBy$}

\chapter{Note aggiuntive}
\labelAppendix{footnotes}
\addtocontents{define}{\noindent\textls{\textsc{\textcolor{reddo}{Appendice C:}
\nowtitle}}
}{}
\addtocontents{theorema}{\noindent\textls{\textsc{\textcolor{reddo}{Appendice C:}
			\nowtitle}}
}{}
\begin{introduction}
‘‘... and they don't stop coming.''
\begin{flushright}
	\textsc{Smash Mouth,} sorpresi che ci siano delle note aggiuntive dopo duecento e passa pagine di appunti.
\end{flushright}
\end{introduction}

\noindent Riportiamo alcune note, precisazioni e dimostrazioni complementari agli argomenti dei capitoli principali.\\
Quanto indicato con il simbolo ⋆ sono degli \textit{approfondimenti non necessari} - ma possono essere comunque utili ed interessanti per un lettore curioso.
\section{Capitolo 7: magnetostatica}
\subsection{Area di una superficie delimitata da una curva piana chiusa}\label{AreaCurvaDelimitata}
\begin{lemming}[Area di una superficie delimitata da una curva piana chiusa]
	Data una curva $\gamma$ piana, chiusa, liscia a tratti, semplice e orientata positivamente, con parametrizzazione $\vba{r}$, si ha che il vettore area per la superficie piana racchiusa da $\gamma$ è
	\begin{equation}
		\Sigma\vbh{u}_n=\frac{1}{2}\int_{\gamma}\vba{r}\cross d\vba{s}
	\end{equation}
\end{lemming}
\begin{demonstration}
	Dato che la nostra curva $\gamma$ è una \textit{curva di Jordan}, ossia $\gamma$ divide il piano in una componente connessa \textit{interna} limitata e una \textit{esterna} limitata, ha senso porre un sistema di riferimento in \textit{coordinate polari}, con origine un punto arbitrariamente scelto nella componente del piano interna alla curva: se $\theta$ è l'angolo azimutale e $\rho$ la distanza dall'origine, si ha
	\begin{equation*}
		\begin{cases}
			x=\rho \cos\theta\\
			y=\rho \sin\theta
		\end{cases}
	\end{equation*}
	La curva $\gamma$ può essere quindi parametrizzata come 
	\begin{equation*}
		\vba{r}(\theta)=\rho(\theta)\cos \theta\vbh{u}_x+\rho(\theta)\sin \theta\vbh{u}_y
	\end{equation*}
	su $\left[\theta_1,\theta_2\right]$ e tale per cui $\vba{r}(\theta_1)=\vba{r}(\theta_2)$; $\rho(\theta)$ descrive la distanza dall'origine. La velocità risulta
	\begin{equation*}
		\vbd{r}(\theta)=\dv{t}\left(\rho(\theta)\cos \theta\right)\vbh{u}_x+\dv{t}\left(\rho(\theta)\sin \theta\right)\vbh{u}_y=\left(\dot{\rho}(\theta)\cos \theta-\rho(\theta)\sin \theta\right)\vbh{u}_x+\left(\dot{\rho}(\theta)\sin \theta+\rho(\theta)\cos \theta\right)\vbh{u}_y
	\end{equation*}
	Poiché l'elemento di area dalle coordinate cartesiane alle coordinate polari diventa
	\begin{equation*}
		dA=dxdy=\rho d\rho d\theta,
	\end{equation*}
	l'area delimitata dalla curva $\gamma$ è
	\begin{equation*}
		\Sigma=\int_{\Sigma}dxdy=\int_{0}^{2pi}\int_{0}^{\rho(\theta)}\int \rho d\rho d\theta= \int_{0}^{2pi}\frac{1}{2}\eval{\rho}_{0}^{\rho(\theta)} d\theta=\frac{1}{2}\int_{0}^{2pi}\rho^2(\theta)d\theta
	\end{equation*}
	Ora, calcolando il prodotto vettoriale 
	\begin{align*}
		\vba{r}\cross\vbd{r}&=
		\begin{vmatrix}
			\vba{u}_x & \vba{u}_y & \vba{u}_z\\
			\rho(\theta)\cos \theta & \rho(\theta)\sin \theta & 0\\
			\dot{\rho}(\theta)\cos \theta-\rho(\theta)\sin \theta & \dot{\rho}(\theta)\sin \theta+\rho(\theta)\cos \theta & 0
		\end{vmatrix}=\\
	&=\left(\rho^2\cos^2\phi + \rho \dot{\rho}\sin\phi\cos\phi-\rho \dot{\rho}\sin\phi\cos\phi+\rho^2\sin^2\phi\right)\vbh{u}_z=\rho^2\left(\phi\right)\vbh{u}_z
	\end{align*}
	Poichè il versore normale ad una superficie nel piano $xy$ quale $\Sigma$ è $\vbh{u}_z$ è immediato trovare che
	\begin{equation*}
		\frac{1}{2}\int_{\gamma}\vba{r}\cross d\vba{r}=\frac{1}{2}\int_{\phi_1}^{\phi_2}\vba{r}\cross\vbd{r}d\phi=\frac{1}{2}\int_{\phi_1}^{\phi_2}\rho^2\left(\phi\right)d\phi\vbh{u}_z=\Sigma\vbh{u}_z\qedhere
	\end{equation*} 
\end{demonstration}
\begin{observe} % TO DO: metterci teorema di stokes? https://en.wikipedia.org/wiki/Green%27s_theorem#Area_calculation
	% https://en.wikipedia.org/wiki/Vector_area
	% https://math.stackexchange.com/questions/74583/proof-that-vector-area-is-a-boundary-integral
	Tale risultato è equivalente a calcolare l'area di una superficie bidimensionale utilizzando il \textbf{teorema di Gauss-Green} e dunque è un'ulteriore corollario del \textit{teorema di Stokes}.\\
	In quanto tale, si può mostare che la relazione scritta vale per una \textit{qualunque} superficie che ha come bordo una determinata curva: il vettore area è determinato \textit{esclusivamente} dal bordo della superficie.
\end{observe}