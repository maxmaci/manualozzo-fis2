% SVN info for this file
\svnidlong
{$HeadURL$}
{$LastChangedDate$}
{$LastChangedRevision$}
{$LastChangedBy$}

\chapter{Note aggiuntive}
\labelAppendix{footnotes}
\addtocontents{define}{\noindent\textls{\textsc{\textcolor{reddo}{Appendice C:}
\nowtitle}}
}{}
\addtocontents{theorema}{\noindent\textls{\textsc{\textcolor{reddo}{Appendice C:}
			\nowtitle}}
}{}
\begin{introduction}
‘‘... and they don't stop coming.''
\begin{flushright}
	\textscsl{Smash Mouth,} sorpresi che ci siano delle note aggiuntive dopo duecento e passa pagine di appunti.
\end{flushright}
\end{introduction}

\noindent Riportiamo alcune note, precisazioni e dimostrazioni complementari agli argomenti dei capitoli principali.\\
Quanto indicato con il simbolo ⋆ sono degli \textit{approfondimenti non necessari} - ma possono essere comunque utili ed interessanti per un lettore curioso.
\section{Capitolo 7: magnetostatica}
\subsection{Area di una superficie delimitata da una curva piana chiusa}\label{AreaCurvaDelimitata}
\begin{lemming}[Area di una superficie delimitata da una curva piana chiusa]
	Data una curva $\gamma$ piana, chiusa, liscia a tratti, semplice e orientata positivamente, con parametrizzazione $\vba{r}$, si ha che il vettore area per la superficie piana racchiusa da $\gamma$ è
	\begin{equation}
		\Sigma\vbh{u}_n=\frac{1}{2}\int_{\gamma}\vba{r}\cross d\vba{s}
	\end{equation}
\end{lemming}
\begin{demonstration}
	Dato che la nostra curva $\gamma$ è una \textit{curva di Jordan}, ossia $\gamma$ divide il piano in una componente connessa \textit{interna} limitata e una \textit{esterna} limitata, ha senso porre un sistema di riferimento in \textit{coordinate polari}, con origine un punto arbitrariamente scelto nella componente del piano interna alla curva: se $\theta$ è l'angolo azimutale e $\rho$ la distanza dall'origine, si ha
	\begin{equation*}
		\begin{cases}
			x=\rho \cos\theta\\
			y=\rho \sin\theta
		\end{cases}
	\end{equation*}
	La curva $\gamma$ può essere quindi parametrizzata come 
	\begin{equation*}
		\vba{r}(\theta)=\rho(\theta)\cos \theta\vbh{u}_x+\rho(\theta)\sin \theta\vbh{u}_y
	\end{equation*}
	su $\left[\theta_1,\theta_2\right]$ e tale per cui $\vba{r}(\theta_1)=\vba{r}(\theta_2)$; $\rho(\theta)$ descrive la distanza dall'origine. La velocità risulta
	\begin{align*}
		\vbd{r}(\theta)&=\dv{t}\left(\rho(\theta)\cos \theta\right)\vbh{u}_x+\dv{t}\left(\rho(\theta)\sin \theta\right)\vbh{u}_y=\\
		&=\left(\dot{\rho}(\theta)\cos \theta-\rho(\theta)\sin \theta\right)\vbh{u}_x+\left(\dot{\rho}(\theta)\sin \theta+\rho(\theta)\cos \theta\right)\vbh{u}_y
	\end{align*}
	Poiché l'elemento di area dalle coordinate cartesiane alle coordinate polari diventa
	\begin{equation*}
		dA=dxdy=\rho d\rho d\theta,
	\end{equation*}
	l'area delimitata dalla curva $\gamma$ è
	\begin{equation*}
		\Sigma=\int_{\Sigma}dxdy=\int_{0}^{2pi}\int_{0}^{\rho(\theta)}\int \rho d\rho d\theta= \int_{0}^{2pi}\frac{1}{2}\eval{\rho}_{0}^{\rho(\theta)} d\theta=\frac{1}{2}\int_{0}^{2pi}\rho^2(\theta)d\theta
	\end{equation*}
	Ora, calcolando il prodotto vettoriale 
	\begin{align*}
		\vba{r}\cross\vbd{r}&=
		\begin{vmatrix}
			\vba{u}_x & \vba{u}_y & \vba{u}_z\\
			\rho(\theta)\cos \theta & \rho(\theta)\sin \theta & 0\\
			\dot{\rho}(\theta)\cos \theta-\rho(\theta)\sin \theta & \dot{\rho}(\theta)\sin \theta+\rho(\theta)\cos \theta & 0
		\end{vmatrix}=\\
	&=\left(\rho^2\cos^2\phi + \rho \dot{\rho}\sin\phi\cos\phi-\rho \dot{\rho}\sin\phi\cos\phi+\rho^2\sin^2\phi\right)\vbh{u}_z=\rho^2\left(\phi\right)\vbh{u}_z
	\end{align*}
	Poichè il versore normale ad una superficie nel piano $xy$ quale $\Sigma$ è $\vbh{u}_z$ è immediato trovare che
	\begin{equation*}
		\frac{1}{2}\int_{\gamma}\vba{r}\cross d\vba{r}=\frac{1}{2}\int_{\phi_1}^{\phi_2}\vba{r}\cross\vbd{r}d\phi=\frac{1}{2}\int_{\phi_1}^{\phi_2}\rho^2\left(\phi\right)d\phi\vbh{u}_z=\Sigma\vbh{u}_z\qedhere
	\end{equation*} 
\end{demonstration}
\begin{observe} % TO DO: metterci teorema di stokes? https://en.wikipedia.org/wiki/Green%27s_theorem#Area_calculation
	% https://en.wikipedia.org/wiki/Vector_area
	% https://math.stackexchange.com/questions/74583/proof-that-vector-area-is-a-boundary-integral
	Tale risultato è equivalente a calcolare l'area di una superficie bidimensionale utilizzando il \textbf{teorema di Gauss-Green} e dunque è un'ulteriore corollario del \textit{teorema di Stokes}.\\
	In quanto tale, si può mostare che la relazione scritta vale per una \textit{qualunque} superficie che ha come bordo una determinata curva: il vettore area è determinato \textit{esclusivamente} dal bordo della superficie.
\end{observe}
\section{Capitolo 11: oscillazioni elettriche e correnti alternate}
\subsection{Fasori}\label{fasori}
\begin{define}[Fasore]
	Data una grandezza sinusoidale
	\begin{equation*}
		\xi(t)=A\cos(\omega t+\oldphi)
	\end{equation*}
	dove l'\textit{ampiezza} $A$, la pulsazione $\omega$ e la fase $\oldphi$ sono costanti temporali, la sua \textbf{rappresentazione analitica}\index{rappresentazione analitica} o \textbf{fasore}\index{fasore} è
	\begin{equation}
		\hat{\xi}(t)=A\cos(\omega t+\oldphi)+iA\sin(\omega t+\oldphi)=Ae^{i(\omega t+\oldphi)}=Ae^{i\oldphi}e^{i\omega t}
	\end{equation}
\end{define}
Il fasore può essere rappresentato nel \textit{piano complesso di Argand-Gauss} come un vettore di lunghezza $A$ e angolo $\oldphi$, che ruota con velocità angolare $\omega$.
%TODO: grafico
In realtà, in molti contesti il termine di rotazione $e^{i\omega t}$ risulta essere comune ad altri fasori e pertanto le operazioni che andremo a fare si possono eseguire direttamente sui termine $Ae^{i\oldphi}$ - sarà poi facile reinserire $e^{i\omega t}$ alla fine dei calcoli. Di conseguenza, potrà essere utile passare alla \textit{rappresentazione vettoriale} per le nostre operazioni tra fasori con la stessa pulsazione.
\begin{attention}
	Per il motivo sopracitato, in diversi contesti il termine ‘‘\textit{fasore}'' fa riferimento esclusivamente a $Ae^{i\theta}$.
\end{attention}
Si osservi che, poiché la grandezza originale non è altro che la \textit{parte reale} del fasore...
\begin{equation}
	\xi(t)=\Re{\hat{\xi}(t)}
\end{equation}
... nella rappresentazione vettoriale dei fasori la grandezza originale è la proiezione sulla (retta dei reali).
\begin{notate}
	Tralasciando il termine di rotazione, un fasore di ampiezza $A$ e angolo $\oldphi$ si indica come
	\begin{equation}
		\xi=A\angle\oldphi
	\end{equation}
\end{notate}
\paragraph{Moltiplicazione per una costante (scalare)}
La \textbf{moltiplicazione} di un fasore
\begin{equation*}
	\hat{\xi}(t)=Ae^{i\oldphi}e^{i\omega t}
\end{equation*}
per una costante scalare complessa
\begin{equation*}
	\hat{k}=Be^{i\theta}
\end{equation*}
è ancora un fasore, che modifica l'\textit{ampiezza} e la \textit{fase} del fasore originale.
\begin{gather}
	\hat{k}\hat{\xi}(t)=Be^{i\theta}\cdot Ae^{i\oldphi}e^{i\omega t}=ABe^{i(\oldphi+\theta)}e^{i\omega t}\\
	k\xi(t)=AB\cos(\omega t+(\oldphi+\theta))
\end{gather}
%TODO: grafico
In notazione fasoriale:
\begin{equation}
	\hat{k}\cdot A\angle\oldphi=AB\angle\left(\oldphi+\theta\right)
\end{equation}
\paragraph{Somma di fasori}
La \textbf{somma} di due fasori
\begin{align*}
	\hat{\xi_1}(t)=A_1e^{i\oldphi_1}e^{i\omega t}&&\hat{\xi_2}(t)=A_2e^{i\oldphi_2}e^{i\omega t}
\end{align*}
è ancora un fasore.
\begin{gather}
	\hat{\xi_1}(t)+\hat{\xi_2}(t)=A_1e^{i\oldphi_1}e^{i\omega t}+A_2e^{i\oldphi_2}e^{i\omega t}=\left(A_1e^{i\oldphi_1}+A_2e^{i\oldphi_2}\right)e^{i\omega t}=A_3e^{i\oldphi_3}e^{i\omega t}\\
	\xi_1(t)+\xi_2(t)=A_3\cos(\omega t+\oldphi_3)
\end{gather}
dove
\begin{gather*}
	A_3^2=\left(A_1^2\cos\oldphi_1+A_2^2\cos\oldphi_1\right)^2+\left(A_1^2\sin\oldphi_1+A_2^2\sin\oldphi_1\right)^2=A_1^2+A_2^2+2A_1A_2\cos(\oldphi_1-\oldphi_2)\\
	\oldphi_3=\begin{cases}
		\textrm{sgn}(A_1 \sin(\oldphi_1)  + A_2 \sin(\oldphi_2)) \frac{\pi}{2}&\text{se}\ A_1 \cos\oldphi_1 + A_2 \cos\oldphi_2 = 0\\
		\arctan\left(\frac{A_1 \sin\oldphi_1 + A_2 \sin\oldphi_2}{A_1 \cos\oldphi_1 + A_2 \cos\oldphi_2}\right)&\text{se}\ A_1 \cos\oldphi_1 + A_2 \cos\oldphi_2 >0\\
		\pi + \arctan\left(\frac{A_1 \sin\oldphi_1 + A_2 \sin\oldphi_2}{A_1 \cos\oldphi_1 + A_2 \cos\oldphi_2}\right) &\text{se}\ A_1 \cos\oldphi_1 + A_2 \cos\oldphi_2 <0
	\end{cases}
\end{gather*}
%TODO: grafico
In notazione fasoriale:
\begin{equation}
	A_1\angle\oldphi_1+A_2\angle\oldphi_2=A_3\angle\oldphi_3
\end{equation}
Alternativamente, si può fare la \textit{somma vettoriale} dei vettori di coordinate
\begin{align*}
	\hat{\xi_1}(t)=\left(A_1\cos(\omega t+\oldphi_1),A_1\sin(\omega t+\oldphi_1)\right)&&
	\hat{\xi_2}(t)=\left(A_2\cos(\omega t+\oldphi_2),A_2\sin(\omega t+\oldphi_2)\right)
\end{align*}
per produrre il vettore risultante
\begin{equation*}
	\hat{\xi_1}(t)+\hat{\xi_2}(t)=\left(A_3\cos(\omega t+\oldphi_3),A_3\sin(\omega t+\oldphi_3)\right)
\end{equation*}
\paragraph{Derivata e integrale}
La \textbf{derivata temporale} o l'\textbf{integrale} rispetto a $t$ di un fasore produce un altro fasore.
\begin{gather}
	\dv{t} \hat{\xi}(t)=\dv{t}\left(Ae^{i\oldphi}e^{i\omega t}\right)=Ae^{i\oldphi}i\omega e^{i\omega t}=\omega Ae^{i\oldphi}e^{i\frac{\pi}{2}} e^{i\omega t}=\omega Ae^{i(\oldphi+\frac{\pi}{2})}e^{i\omega t}\\
	\dv{t} \xi(t)=\omega A\cos\left(\omega t+\theta+\frac{\pi}{2}\right)
\end{gather}
In altre parole, \textit{derivare} un fasore temporalmente è equivalente a moltiplicare per la costante scalare $i\omega$.
\begin{equation}
	\dv{t} \hat{\xi}(t)=i\omega\hat{\xi}(t)=e^{i\frac{\pi}{2}}\omega\hat{\xi}(t)
\end{equation}
In notazione fasoriale:
\begin{equation}
	\dv{t} A\angle\oldphi=i\omega \cdot A\angle\oldphi=\omega A\angle\left(\oldphi+\frac{\pi}{2}\right)
\end{equation}
In modo analogo, integrare un fasore temporalmente è equivalente a dividere per la costante scalare $i\omega$.
\begin{equation}
	\int \hat{\xi}(t) dt=\frac{1}{i\omega}\hat{\xi}(t)=\frac{e^{-\frac{\pi}{2}}}{\omega}\hat{\xi}(t)
\end{equation}
In notazione fasoriale:
\begin{equation}
	\int A\angle\oldphi dt=\frac{i\omega} \cdot A\angle\oldphi=\frac{A}{\omega}\angle\left(\oldphi-\frac{\pi}{2}\right)
\end{equation}
\paragraph{Rapporto tra fasori}
Il \textbf{rapporto} tra due fasori
\begin{align*}
	\hat{\xi_1}(t)=A_1e^{i\oldphi_1}e^{i\omega t}&&\hat{\xi_2}(t)=A_2e^{i\oldphi_2}e^{i\omega t}
\end{align*}
prende il nome di \textbf{impendenza}\index{impendenza}:
\begin{equation}
	 Z=\frac{\hat{\xi_1}(t)}{\hat{\xi_2}(t)}=\frac{A_1e^{i\oldphi_1}\Ccancel[red]{e^{i\omega t}}}{A_2e^{i\oldphi_2}\Ccancel[red]{e^{i\omega t}}}=\frac{A_1}{A_2}e^{i(\oldphi_1-\oldphi_2)}
\end{equation}
Tuttavia, \textit{non} è un fasore, perché non corrisponde ad una grandezza sinusoidale variabile nel tempo.