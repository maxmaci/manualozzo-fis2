% SVN info for this file
\svnidlong
{$HeadURL$}
{$LastChangedDate$}
{$LastChangedRevision$}
{$LastChangedBy$}

\chapter{Pensieri e ringraziamenti}
\labelChapter{ringraziaments}

\begin{introduction}
‘‘\textbf{Lafayette}: Ehi, Napoleone! Direi che questa è la fine.\\
\textbf{Napoleone}: Un momento, il capo sono io! Lo dico io quando è la fine!\\
\textbf{[La parola \textsf{FINE} lo colpisce alla testa.]}\\
\textbf{Napoleone}: È la fine.''
\begin{flushright}
	\textscsl{Gli Aristogatti.}
\end{flushright}
\end{introduction}

\lettrine[findent=1pt, nindent=0pt]{A}{ Gennaio, quando} avevo concluso il Manualozzo\texttrademark\ di Analisi Matematica 3 mi ero detto ‘‘Bene, ho scritto tre Manualozzi\texttrademark. Tre è un bel numero per smettere''.\\
Eppure, all'inizio del nuovo semestre, mi fu gentilmente chiesto di farne uno nuovo - per di più per un corso come Fisica II, la bestia nera di ogni matematico. Chiaramente mi rifiutai: l'esperienza dei Manualozzi\texttrademark\ è stata bella, ma mo' basta! Ma evidentemente la mia dipendenza da Manualozzi\texttrademark\ ha preso il sopravvento, perché alla fine ne state leggendo uno. Sigh. Menomale che Elisa mi ha aiutato sbobinando le lezioni - anche se mi meraviglio ogni volta che sia in grado di farlo già in \LaTeX\ mentre il prof spiega.\\
\newline
\noindent Il testo che avete di fronte è in realtà una versione \textit{incompleta} di quella che inizialmente doveva essere: per questioni di tempo e di salute, non son riuscito a lavorare alla parte dedicata alle onde, né quella di relatività e quantistica. Se troverò mai tempo di scrivere una nuova edizione cercherò almeno di trattare le onde.\\
Ciò nonostante, credo che questo sarà davvero l'ultimo Manualozzo\texttrademark. Un po' poeticamente, concludo questa esperienza come ho iniziato: con un argomento di ambito fisico. Anche se so già che ben presto mi rimangerò queste parole.\\
\newline
\noindent Essendo anche oramai finita la Triennale, vorrei utilizzare questo spazio per ringraziare e salutare le tante persone con cui sono venuto in contatto durante questi anni e che, chi di più, chi di meno, hanno influenzato la mia vita universitaria. Ciò nonostante, non penso di dilungarmi troppo: ho già scritto quasi 300 pagine in \LaTeX\ e se proprio posso evitarne altre non mi dispiacerebbe farlo.\\
\newline
A Elisa Antuca, a cui va il mio più sentito ringraziamento per l'aiuto che mi ha dato non solo in questo Manualozzo\texttrademark, ma anche per tutti quelli precedenti. Per essere un pinguino ci sa fare con \LaTeX.\\
A Francesca Colombo, che da vera amica con il suo essere solare mi ha portato un po' di luce nei momenti più bui. E apprezzo davvero il suo magico potere di convincermi a non lavorare troppo, non so come faccia.\\
A Guido Buffa, che con la simpatia - e sì, anche con il suo sarcasmo - sono sicuro che se si desse alla \textit{stand up comedy} diventerebbe famoso... e starei in prima fila a ridere di buon gusto. Purtroppo alla fine non sono diventato un membro produttivo della società a causa sua e di Farming Simulator, purtuttavia non mi lamento.\\
A Julian Kerpaci, che con la sua vivacità e il suo sorriso in faccia mi rallegra spesso la giornata. Forse mi rallegra anche il fatto di averlo battuto al Fantacalcio. \\
A Samuele Corsato, che fra poco potrebbe persino superarmi in \LaTeX. {\small Forse.} {\tiny Ti tengo d'occhio.}\\
A Marco Lugarà, che non è assolutamente colui che mi spinse a scrivere questo Manualozzo\texttrademark. {\small Ogni tanto dovrei dirgli di no, mannaggia a me.}\\
A Matteo Cagnotti, Emiliano Colla, Lorenzo Ferrara, Andrea Scalenghe, Charif El Gataa, Matteo Bracco, Chiara Comoglio, ma anche a Matilda Urani, Andrea Natale, Pietro Raviola, Riccardo Ponte, Francesco Ruatta, Zoe Wynants, Eugenio del Nero, Orazio Nicolosi, He Xiaohui, Davide Miolano, Lisa Galvagni - insomma, a tutto il \textit{Chen Fanpage \& friends} - e a tutta la gente simpatica che ho conosciuto in tre anni di università: avrei tante cose da dire, ma il margine della pagina è troppo stretto per farlo.\\
A Elena Maserati, che devo ancora capire come fai preferire Fisica a Matematica, tsk. Prima o poi ti farò piacere la Matematica, Diofà!
\begin{flushright}
	\textcolor{redill}{\href{https://www.youtube.com/watch?v=CjUVTEExfBg}{\textscsl{See you space Manualozzo...}}}
\end{flushright}
\vfill
\begin{center}
	\begin{frame}{}
		\animategraphics[loop,autoplay,width=0.5\linewidth]{7}{images/meme/rick-}{0}{93}
	\end{frame}
\end{center}
\vspace{5pt}
\begin{center}
{\scriptsize Una piccola curiosità: questa immagine e anche quella di copertina sono state create interamente dall'intelligenza artificiale \textbf{DALL·E 2} basandosi l'input \textit{``A surrealist painting of a magnet in the style of The Treachery of Images by René Magritte''}.\\
\textit{Il tradimento delle immagini}, huh... chissà se questa immagine mi tradirà.}
\end{center}