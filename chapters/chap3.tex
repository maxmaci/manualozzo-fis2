% SVN info for this file
% SVN info for this file
\svnidlong
{$HeadURL$}
{$LastChangedDate$}
{$LastChangedRevision$}
{$LastChangedBy$}

\chapter{Il potenziale elettrico} % TO DO: Il lavoro elettrico?
\labelChapter{PotenzialeElettrico}
% TO DO: quando parlo del potenziale e del campo elettrico a tratti, sostituire il geq o leq con < e > 
\begin{introduction}
	‘‘La situazione si complica. Ora, ce ne sono due di loro!''
	\begin{flushright}
		\textsc{Nute Gunray,} scoprendo la convergenza puntuale dopo quella uniforme. % TO DO: quote
	\end{flushright}
\end{introduction}
\lettrine[findent=1pt, nindent=0pt]{P}{er}  % TO DO: intro
\section{Circuitazione di un campo vettoriale}
% TO DO: concezione intuitiva di circuitazione
\begin{define}[Circuitazione di un campo vettoriale]
	Il \textbf{circuitazione di un campo vettoriale lungo una curva chiusa}\index{circuitazione di un campo vettoriale} $\gamma$, parametrizzata da una funzione $\funz[\vba{r}]{\left[a,b\right]}{\realset^3}$, è
	\begin{equation}
		\Gamma_{\gamma}\left(\vba{E}\right)=\oint_{\gamma}\vba{E}\vdot d\vba{s}=\int_{a}^b\vba{E}\vdot\frac{d\vba{r}}{dt}dt
	\end{equation}
\end{define}
\section{Forze conservative e campi vettoriali conservativi}
\paragraph{Dalle forze conservative...}
La circuitazione ha particolare rilevanza in ambito fisico; la sua prima applicazione che vedremo è in una caratterizzazione dei\textit{campi vettoriali conservativi}. Prima però, riguardiamo rapidamente il concetto di lavoro e del suo ruolo per quelle forze dette forze conservatrici, trattato nel corso di \textsc{Fisica I}.
\begin{define}[Lavoro]
	Dati due punti $\vba{r}_A$ e $\vba{r}_B$, il \textbf{lavoro}\index{lavoro} di una forza $\vba{F}$ lungo una curva $\gamma$ tra i due punti è definito come
	\begin{equation}
		W_{\gamma_1}=\int_{\gamma_1}\vba{F}\vdot d\vba{s}
	\end{equation}
\end{define}
In generale, il lavoro dipende dal \textit{percorso effettuato}: il lavoro compiuto da una stessa forza lungo due curve $\gamma_1$ e $\gamma_2$  è differente. Per un caso particolare di forze, tuttavia, il lavoro dipende esclusivamente dagli estremi e non dal percorso effettuato.
\begin{define}[Forza conservativa]
	Una forza $\vba{F}$ è detta \textbf{conservativa}\index{forza!conservativa} se per qualunque curva $\gamma_1,\ \gamma_2$ tra due punti $\vba{r}_A$ e $\vba{r}_B$ il lavoro è
	\begin{equation}
		W_{\gamma_1}=W_{\gamma_2}
	\end{equation}
	In altre parole, $\vba{F}$ è conservativa se il lavoro dipende \textit{solo} dai punti iniziali e finali e non quale sia la curva lungo la quale si calcola.
\end{define}
\begin{proposition}[Caratterizzazione delle forze conservative]\label{forzaconservativaproposizione} % TO DO: controllare che non sia il teorema del gradiente?
	Se una forza $\vba{F}$ è conservativa, allora esiste una funzione $\funz[U]{\realset^3}{\realset}$ detta \textbf{energia potenziale}\index{energia!potenziale} tale che\footnote{Il meno è presente per motivi storici.}
	\begin{equation}
		\vba{F}=-\grad{U}
	\end{equation}
	e tale per cui
	\begin{equation}
		W=U(\vba{r}_A)-U(\vba{r}_B)=-\Delta U
	\end{equation}
	In altri termini:
	\begin{gather}
		\vba{F}=-\grad{U}\\
		W=\int_A^{B}\vba{F}\vdot d\vba{s}=-\int_A^B\grad{U}\vdot d\vba{s}
	\end{gather}
\end{proposition}
\begin{demonstration}
	Consideriamo una curva $\gamma$ parametrizzata da
	\begin{equation*}
		\vba{r}(t)=\left(x(t),y(t),z(t)\right),\ \quad t\in\left[t_1,t_2\right]
	\end{equation*}
	Posto
	\begin{equation*}
		\begin{cases}
			\vba{r}(t_A)=\vba{r}_A\\
			\vba{r}(t_B)=\vba{r}_B
		\end{cases}
	\end{equation*}
si ha
\begin{align*}
	W\\=\int_{t_A}^{t_B}\vba{F}\vdot\dv{\vba{r}}{t}dt=-\int_{t_A}^{t_B}\grad{U}\vdot\dv{\vba{r}}{t}dt=-\int_{t_A}^{t_B}\dv{t}U\left(\vba{r}(t)\right)dt=\\
	&=-\left(U(\vba{r}(t_B))-U(\vba{r}(t_A))\right)=U(\vba{r}_A)-U(\vba{r}_B)
\end{align*}
Infatti,
\begin{equation*}
	\dv{t}U(\vba{r}(t))=\pdv{U}{x^i}\pdv{x^i}{t}=\grad{U}\vdot\pdv{r}{t}.\qedhere
\end{equation*}
\end{demonstration}
\begin{observe}
	Il potenziale è sempre definito a meno di costante additiva. Infatti, se considero due potenziali $U$ e $U'=U+U_0$ dove $U_0$ è una costante reale, si ha che
	\begin{equation*}
		\vba{F}=-\grad{U'}=-\grad{(U+U_0)}=-\grad{U}-\underbrace{\grad{U_0}}_{=0}=-\grad{U}
	\end{equation*}
\end{observe}
\paragraph{...ai campi vettoriali conservativi}
In modo analogo a come abbiamo fatto per le forze conservative, possiamo facilmente definire un \textit{campo vettoriale conservativo}.
\begin{define}[Campo vettoriale conservativo]
	Una campo vettoriale $\vba{G}$ è detto \textbf{conservativo}\index{campo!conservativo} se per qualunque curva $\gamma_1,\ \gamma_2$ tra due punti $\vba{r}_A$ e $\vba{r}_B$ si ha
	\begin{equation}
		\int_{\gamma_1}\vba{G}\vdot d\vba{s}=\int_{\gamma_1}\vba{G}\vdot d\vba{s}
	\end{equation}
\end{define}
\begin{proposition}[Caratterizzazione dei campi vettoriali conservativi]\label{campivettorialiconservativiproposizione} % TO DO: controllare che non sia il teorema del gradiente?
	Se un campo vettoriale$\vba{G}$ è conservativo, allora esiste un campo scalare $\funz[\oldphi]{\realset^3}{\realset}$ detto \textbf{potenziale} tale per cui\footnote{Il meno è presente per motivi storici.}
	\begin{equation}
		\begin{cases}
			\vba{G}=-\grad{\oldphi}\\
			\Gamma_{\gamma}(\vba{G})=0
		\end{cases}
	\end{equation}
per ogni curva chiusa $\gamma$.
\end{proposition}
\begin{demonstration}
	La dimostrazione è analoga a quella della proposizione \ref{forzaconservativaproposizione}. Per ottenere il risultato come enunciato nella tesi - ossia come circuitazione - si noti che, avendo
	\begin{equation*}
		\int_{\gamma}\vba{G}\vdot d\vba{s}=\oldphi(\vba{r}_B)-\oldphi(\vba{r}_A),
	\end{equation*}
	allora, poiché si ha $\vba{r}_a=\vba{r}_B$, vale
	\begin{equation*}
		\Gamma_{\gamma}(\vba{G})=\oint_{\gamma}\vba{G}\vdot d\vba{s}=0\qedhere
	\end{equation*}
\end{demonstration}
\begin{observe}
	Il potenziale è sempre definito a meno di costante additiva. Infatti, se considero due potenziale $\oldphi$ e $\oldphi'=\oldphi+\oldphi_0$ dove $\oldphi_0$ è una costante reale, si ha che
	\begin{equation*}
		\vba{G}=-\grad{\oldphi'}=-\grad{(\oldphi+\oldphi_0)}=-\grad{\oldphi}-\underbrace{\grad{\oldphi_0}}_{=0}=-\grad{\oldphi}
	\end{equation*}
\end{observe}
\section{Potenziale elettrico}
Nei diversi esempi di campi elettrostatici visti nel \autoref{leggediCoulomb} abbiamo sempre trovato un potenziale che ci permetteva di semplificare notevolmente la trattazione del problema. Come preannunciato, questo non è un caso: infatti, il \textit{campo elettrostatico} è sempre conservativo.
\begin{theorema}[Il campo elettrostatico è conservativo]
	Il campo elettrostatico $\vba{E}$ è conservativo, ossia è il gradiente (cambiato di segno) di un opportuno campo scalare $V$.
	\begin{equation}
		\vba{E}=-\grad{V}
	\end{equation}
\end{theorema}
\begin{demonstration}
	Dimostriamolo inizialmente per il campo elettrostatico generato da una carica puntiforme.
	Il campo di Coulomb è 
\begin{equation*}
	\vba{E}=\frac{q}{4\pi\epsilon_0r^2}\vbh{u}_r
\end{equation*}
Ricordiamo che l'operatore nabla in coordinate sferiche diventa
\begin{equation*}
	\grad =\pdv{r}\vbh{u}_r+\frac{1}{r}\pdv{\theta}\vbh{u}_\theta+\frac{1}{r\sin\theta}\pdv{\phi}\vbh{u}_\phi
\end{equation*}
Si verifica facilmente che
\begin{equation}
	V=\frac{q}{4\pi\epsilon_0r}
\end{equation}
è il potenziale del campo di Coulomb:
\begin{equation*}
	-\grad{V}=-\pdv{r}\left(\frac{q}{4\pi\epsilon_0r}\right)\vbh{u}_r=\frac{q}{4\pi\epsilon_0r^2}\vbh{u}_r=\vba{E}\qedhere
\end{equation*}
Si osservi che per i campi elettrostatici generati da un sistema di cariche $q_i$ vale il principio di sovrapposizione; noto che ciascuno di questi campi sono conservativi e
\begin{equation*}
	\vba{E}_i=-\grad{V}_i
\end{equation*}
allora anche il campo complessivo generato dal sistema di cariche è conservativo e il potenziale è la somma dei potenziali:
\begin{equation}
	\vba{E}=\sum_i\vba{E}_i=-\grad{V}
\end{equation}
dove
\begin{equation}
	V=\sum_iV_i
\end{equation}
Il caso di una distribuzione continua di carica segue da queste relazioni, passando al continuo.
\end{demonstration}
Ricapitolando:
\begin{itemize}
	\item Per una singola carica $q$, centrata nell'origine, il potenziale in $\vba{r}$ è:
	\begin{equation}
		V(x,y,z)=\frac{q}{4\pi\epsilon_0r}
	\end{equation}
	\item Per un sistema di cariche $q_i$, ciascuna posta in $\vba{r}_i$, il potenziale in $\vba{r}$ è
	\begin{equation}
		V(x,y,z)=\sum_i\frac{q_i}{4\pi\epsilon_0\abs{\vba{r}-\vba{r}_i}}
	\end{equation}
	\item Per una distribuzione continua di cariche in un volume $V$ è
	\begin{equation}
		V(x,y,z)=\frac{1}{4\pi\epsilon_0}\int_{V}\frac{\rho(x',y',z')}{\abs{\vba{r}-\vba{r}'}}dx'dy'dz'
	\end{equation}
	\item Per una distribuzione continua di cariche su una superficie $\Sigma$ è
	\begin{equation}
		V(x,y,z)=\frac{1}{4\pi\epsilon_0}\int_{\Sigma}\frac{\sigma(x',y',z')}{\abs{\vba{r}-\vba{r}'}}d\Sigma
	\end{equation}
	\item Per una distribuzione continua di cariche su una superficie $\Sigma$ è
	\begin{equation}
		V(x,y,z)=\frac{1}{4\pi\epsilon_0}\int_\mathcal{l}\frac{\lambda(x',y',z')}{\abs{\vba{r}-\vba{r}'}}d\mathcal{l}
	\end{equation}
\end{itemize}
\begin{observe}
	il potenziale si considera a tutti gli effetti un campo scalare \textit{continuo}: se così non fosse, le derivate spaziali sarebbero infinite nelle discontinuità e quindi avremmo in certi punti del dominio di definizione del potenziale un campo elettrostatico infinito, che nella pratica non è possibile avere.
\end{observe}
\begin{attention}
	Come potremo vedere più avanti, in generale il campo \textit{elettrico} può dipendere dal tempo; tuttavia, in tal caso il campo \textbf{non} è conservativo e dunque non esiste un potenziale. Solo il campo elettrostatico, cioè un campo elettrico che \textit{non} varia nel tempo, ammette un potenziale.
\end{attention}
Una conseguenza immediata del fatto che il campo elettrostatico è conservativo è il seguente.
\begin{corollary}[La circuitazione del campo elettrostatico è nulla]
	Su ogni curva chiusa $\gamma$ nello spazio, la circuitazione del campo elettrostatico è nulla.
	\begin{equation}
		\Gamma_{\gamma}(\vba{E})=0
	\end{equation}
\end{corollary}
\begin{demonstration}
	Il campo elettrostatico è conservativo, dunque per la caratterizzazione dei campi vettoriali conservativi\footnote{Si veda la proposizione \ref{campivettorialiconservativiproposizione} a pag. \pageref{campivettorialiconservativiproposizione}.} segue la tesi. 
\end{demonstration}% TO DO: aggiungere esempio del capitolo 2 sul campo elettrico di una sfera?
\paragraph{Energia potenziale e potenziale elettrico}
La forza di Coulomb è conservativa e ammette un potenziale, o per meglio dire un'\textbf{energia potenziale}\index{energia!potenziale}, $U$ tale che $vba{F}=-\grad{U}$, mentre abbiamo dimostrato poc'anzi che $\vba{E}=-\grad{V}$. Dalla legge $\vba{F}=q\vba{E}$ che lega campo elettrostatico e forza di Coulomb si ha immediatamente una relazione tra l'energia potenziale e il potenziale elettrico:
\begin{equation}
	U=qV\label{EnergiaPotenziale}
\end{equation}
\begin{example}
	Per un carica puntiforme $Q$ si ha potenziale
	\begin{equation}
		V=\frac{Q}{4\pi\epsilon_0r}
	\end{equation}
	e l'energia potenziale che un'altra carica $q$ ha, se soggetta al campo elettrostatico generato da $Q$ è
	\begin{equation}
		U=\frac{qQ}{4\pi\epsilon_0r}
	\end{equation}
\end{example}
\paragraph{Unità di misura}
Dalla relazione \ref{EnergiaPotenziale} si definisce l'unità di misura del potenziale, il \textbf{volt}\index{volt}.
\begin{equation*}
	\left[V\right]=\frac{[U]}{[q]}=\unit[per-mode = fraction]{\joule\per\coulomb}
\end{equation*}
\begin{units}~\\
	\textbf{\textsc{Potenziale:}} volt ($\unit{\volt}$) o joule su coulomb $\left(\unit[per-mode = fraction]{\joule\per\coulomb}\right)$.\\
	\textit{\textbf{Dimensioni:}} $[V]=\dfrac{[J]}{[C]}=\mathsf{M} \mathsf{L}^2  \mathsf{T}^{-3}\mathsf{I}^{-1}$
\end{units}
\noindent Abbiamo già visto che il campo elettrico, dalla formula $\vba{F}=q\vba{E}$, ha unità di misura il newton su coulomb $\left(\unit[per-mode = fraction]{\newton\per\coulomb}\right)$. Tuttavia, grazie al fatto che $\vba{E}$ è conservativo ed è quindi un \textit{gradiente} rispetto ad una variabile che ha dimensioni di una lunghezza, può essere definito anche come \textbf{volt su metro} $\left(\unit[per-mode = fraction]{\volt\per\metre}\right)$.
\begin{units}~\\
	\textbf{\textsc{Campo elettrico:}} volt su metro $\left(\unit[per-mode = fraction]{\volt\per\metre}\right)$ o newton su coulomb $\left(\unit[per-mode = fraction]{\newton\per\coulomb}\right)$.\\
	\textit{\textbf{Dimensioni:}} $[E]=\dfrac{[F]}{[q]}=\mathsf{L}\mathsf{M}\mathsf{T}^{-3}\mathsf{I}^{-1}$.
\end{units}
\begin{observe}
	Come già detto, il potenziale è definito a meno di una costante additiva. Generalmente si pone il sistema di riferimento in modo che il potenziale all'infinito (o ai bordi del dominio di definizione) sia una costante, generalmente zero per $V\to\infty$.\\
	Poiché non si può considerare un sistema di questo genere, l'unica condizione misurabile realmente (ed operativamente) è la \textbf{differenza di potenziale}\index{differenza di potenziale}.
\end{observe}
\paragraph{Potenziale e attrattività delle cariche}
\begin{examplewt}[Armature elettriche]
	Consideriamo due piastre elettrostatiche di segno opposto, come in figura.
	% TO DO: immagine
	Il funzionamento di tale sistema non è dissimile, a livello puramente qualitativo, da un dipolo elettrico: tra le due piastre il campo è sostanzialmente analogo a quello sull'asse verticale congiungente i dipoli e quindi è diretto dalla piastra positiva a quella negativa.\\
	Poiché $\vba{F}=q\vba{E}$, le cariche positive saranno attratte verso la parte negativa, quelle negative verso la piastra positiva.
	
	Si osserva che, essendo $\grad{V}=-\vba{E}$, il gradiente del potenziale è un campo vettoriale diretto - quando lo consideriamo dentro l'armatura - diretto dalla piastra negativa a quella positiva, cioè dal potenziale minore a quello maggiore.
	
	% TO DO: spiegare perché funziona, che non mi è chiaro, MAZZOOLLLDI?
	Si ha che
	\begin{equation*}
		V_1<V_2
	\end{equation*}
\end{examplewt}
Questo ragionamento si può anche generalizzare in altri contesti, osservando dunque che il campo elettrico è diretto dal \textbf{potenziale maggiore al potenziale minore}, e quindi
\begin{itemize}
	\item cariche \textbf{positive} si muovono verso la zona di \textbf{minor} potenziale.
	\item cariche \textbf{negative} si muovono verso la zona di \textbf{maggior} potenziale.
\end{itemize}
\paragraph{Superfici equipotenziali}
\begin{define}[Superfici equipotenziali]
	Data un sistema in cui si ha una certa funzione di potenziale $V$, le \textbf{superfici equipotenziali}\index{superfici equipotenziali} sono gli insiemi descritti dall'equazione
	\begin{equation}
		V(\vba{r})=\textrm{const}
	\end{equation}
	Le superfici equipotenziali sono sempre ortogonali, punto per punto, a $\grad{V}$ e quindi anche al campo vettoriale elettrostatico $\vba{E}$.
\end{define}
\begin{examples}~{}
	\begin{itemize}
		\item Per il potenziale della carica puntiforme, $V=\frac{q}{4\pi\epsilon_0r}=\mathrm{const}$ implica $r=\mathrm{const}$:le superfici equipotenziali sono circonferenze concentriche di raggio $r$, al variare di $r$.
		% TO DO: inserire immagini
		\item Per il dipolo elettrico $+/-$:
		% TO DO: inserire immagini
		\item Per il dipolo elettrico $+/+$:
	\end{itemize}
\end{examples}
\section{Discontinuità di campo elettrico tra superfici}
Precedentemente abbiamo ricavato il campo elettrostatico e il potenziale di volumi uniformemente carichi, come una sfera. Ci chiediamo ora quale sia il campo elettrostatico e di conseguenza il potenziale di una \textbf{superficie cava} uniformemente carica, come ad esempio una superficie sferica o un cilindro.
\paragraph{Superficie sferica uniformemente carica}
Si consideri una superficie sferica di raggio $R$ con densità superficiale costante $\sigma$. Per semplicità, poniamo il sistema di riferimento in modo che l'origine coincida con il centro della sfera.
La carica totale sulla superficie è
\begin{equation}
	q=4\pi R^2 \sigma
\end{equation}
Distinguiamo, come al solito, due casi: il campo elettrico interno ($r<R$) e quello esterno ($r>R$) alla sfera.
\begin{itemize}
	\item[$\mathbf{r<R}$] Utilizziamo la legge di Gauss su una superficie $\Sigma$ di raggio $r<R$ centrata nell'origine. Dalla definizione di flusso abbiamo che
	\begin{equation*}
		\Phi_{\Sigma}\left(\vba{E}\right)=\int E(r)d\Sigma=4\pi r^2 E_r(r)
	\end{equation*}
dato che $E(r)$ è costante su $\Sigma$.
Tuttavia, poiché la carica è concentrata tutta sulla sfera di raggio $R$, la superficie $\Sigma$ \textit{non} contiene alcuna carica; pertanto, per la legge di Gauss
\begin{equation*}
	4\pi r^2 E_r(r)=\Phi_{\Sigma}\left(\vba{E}\right)=\frac{q_{interna}}{\epsilon_0}=0
\end{equation*}
da cui segue che
\begin{equation}
	\vba{E}(r)=0
\end{equation}
\item[$\mathbf{r>R}$] Sulla base di osservazioni precedenti, il comportamento esterno è analogo a quello di una carica puntiforme nell'origine.
\begin{equation}
	\vba{E}(r)=\frac{\sigma R^2}{\epsilon_0r^2}\vbh{u}_r=\frac{q}{4\pi\epsilon_0r^2}\vbh{u}_r
\end{equation}
\end{itemize}
Il campo elettrico, pertanto, è discontinuo e vale
\begin{equation}
	\vba{E}(r)=\begin{cases}
		0&\text{se}\ r\leq R\\
		\frac{\sigma R^2}{\epsilon_0r^2}\vbh{u}_r=\frac{q}{4\pi\epsilon_0r^2}\vbh{u}_r &\text{se}\ r\geq R
	\end{cases}
\end{equation}
mentre invece il potenziale è continuo e vale
\begin{equation}
	V(r)=
	\begin{cases}
	\frac{\sigma R}{\epsilon_0}=\frac{q}{4\pi\epsilon_0 R}&\text{se}\ r\leq R\\
	\frac{\sigma R^2}{\epsilon_0 r}=\frac{q}{4\pi\epsilon_0 r}&\text{se}\ r\geq R
	\end{cases}
\end{equation}
Si osserva una discontinuità pari a $\frac{\sigma}{\epsilon_0}$ tra il campo elettrico interno ed esterno alla superficie.
\paragraph{Cilindro uniformemente carico}
Si consideri una superficie cilindrica di raggio $R_0$ e lunghezza $L$ con densità superficiale costante $\sigma$, dove le cariche sono disposte sulla faccia laterale. Per semplicità, poniamo il sistema di riferimento in modo che l'asse $z$ passi per l'asse del cilindro.
La carica totale sulla superficie è
\begin{equation}
	q=A_{laterale} \sigma=2\pi R_0 L \sigma
\end{equation}
Distinguiamo due casi: il campo elettrico interno ($R<R_0$) e quello esterno ($R>R_0$) al cilindro.
\begin{itemize}
	\item[$\mathbf{R<R_0}$] Utilizziamo la legge di Gauss su un cilindro $\Sigma$ di raggio $R<R_0$ con asse sull'asse $z$. Dalla definizione di flusso abbiamo che
	\begin{equation*}
		\Phi_{\Sigma}\left(\vba{E}\right)=\int E(R)d\Sigma=2\pi R L E_R(R)
	\end{equation*}
	dato che $E(R)$ è costante su $\Sigma$.
	Tuttavia, poiché la carica è concentrata tutta sul cilindro di raggio $R_0$, la superficie $\Sigma$ \textit{non} contiene alcuna carica; pertanto, per la legge di Gauss
	\begin{equation*}
		2\pi R L E_R(R)=\Phi_{\Sigma}\left(\vba{E}\right)=\frac{q_{interna}}{\epsilon_0}=0
	\end{equation*}
	da cui segue che
	\begin{equation}
		\vba{E}(R)=0
	\end{equation}
	\item[$\mathbf{R>R_0}$] In modo simile al caso della sfera, il comportamento esterno è analogo a quello di un filo carico. % TO DO: check mazzoldi?
	\begin{equation}
		\vba{E}(r)=\frac{\sigma R_0}{\epsilon_0r}\vbh{u}_R=\frac{q}{2\pi\epsilon_0Lr}\vbh{u}_R
	\end{equation}
\end{itemize}
Il campo elettrico, pertanto, è discontinuo e vale
\begin{equation}
	\vba{E}(r)=\begin{cases}
		0&\text{se}\ r\leq R\\
		\frac{\sigma R_0}{\epsilon_0r}\vbh{u}_R=\frac{q}{2\pi\epsilon_0Lr}\vbh{u}_R &\text{se}\ r\geq R
	\end{cases}
\end{equation}
mentre invece il potenziale è continuo e vale
\begin{equation}
	V(r)=
	\begin{cases}
		-\frac{\sigma R_0}{\epsilon_0}\log R=-\frac{q}{2\pi\epsilon_0L}\log R&\text{se}\ r\leq R\\
		-\frac{\sigma R_0}{\epsilon_0}\log r=-\frac{q}{2\pi\epsilon_0L}\log r&\text{se}\ r\geq R
	\end{cases}
\end{equation}
Si osserva una discontinuità pari a $\frac{\sigma}{\epsilon_0}$ tra il campo elettrico interno ed esterno alla superficie.
\paragraph{Discontinuità di campo elettrico tra superfici}
Sebbene l'andamento del campo elettrico nei due esempi precedenti è leggermente diverso, per entrambi i casi la discontinuità tra campo elettrico interno ed esterno in uno stesso punto è pari al valore $\frac{\sigma}{\epsilon_0}$. Non è una coincidenza fortuita, bensì possiamo mostrare che questo è \textit{sempre} così.
\begin{proposition}[Discontinuità di campo elettrico tra superfici]
	La differenza di campo elettrico tra due lati di una superficie carica è, punto per punto, pari a 
	\begin{equation*}
		\Delta \vba{E}(\vba{r})=\frac{\sigma}{\epsilon_0}\vbh{u}_n
	\end{equation*}
\end{proposition}
\begin{demonstration}
	Dimostriamo tale proprietà per un foglio carico (supponiamo positivamente), dato che la superficie si può considerare almeno localmente come un foglio 
	carico.
	
	Posta una carica positiva su una faccia del foglio, ci aspettiamo che si allontani da esso ‘‘dallo stesso lato del foglio'' a causa di un campo elettrico $\vba{E}_1$; viceversa, mettendo una particella positiva dall'altra faccia è prevedibile che la particella sarà respinta dalla superficie da quello stesso lato dalla forza generata dal campo elettrico $\vba{E}_2$, ossia dal verso opposto di $\vba{E}_1$: ci dovrà essere necessariamente una discontinuità di campo elettrico per avere questo cambio drastico di verso.
	
	% TO DO: inserire immagine
	Disegniamo un circuito rettangolare $\gamma=[ABCD]$ che interseca il campo e che sia sufficientemente piccolo in modo da considerare $\vba{E}$ costante sul circuito. Dato che la circuitazione circuitazione lungo $\gamma$ è influenzata solo dalle componenti tangenziali $E_{i,t}$ e non dalle componenti perpendicolari $E_{i,n}$ al circuito, si ha
	\begin{equation*}
		0=\Gamma_{\gamma}\left(\vba{E}\right)=E_{1,t}d_{AB}-E_{2,t}d_{CD}=\left(E_{2,t}-E_{1,t}\right)d_{AB}
	\end{equation*}
Da cui segue che
\begin{equation*}
	E_{2,t}=E_{1,t},
\end{equation*}
ossia che le componenti tangenziali devono essere uguali.

%TO DO: inserire immagini
Consideriamo lo stesso foglio, questa volta intersecandolo con una superficie cilindrica con altezza sufficientemente piccola in modo che  il campo elettrico è considerabile costante lungo la superficie di base. Calcolando il flusso e confrontandolo con quello ottenuto dalle legge di Gauss, ricordiamo che lungo la superficie laterale esso è nullo:
\begin{equation*}
	\frac{q_{A_{base}}}{\epsilon_0}=\Phi_{\Sigma}\left(\vba{E}\right)=E_{1,n}A_{base}-E_{2,n}A_{base}
\end{equation*}
Da cui segue invece
\begin{equation*}
	E_{1,n}-E_{2,n}=\frac{\sigma}{\epsilon_0}
\end{equation*}
Allora, facendo la differenza punto per punto, si ha
\begin{equation*}
	\Delta\vba{E}=\underbrace{\left(E_{1,t}-E_{2,t}\right)}_{=0}\vbh{u}_t+\underbrace{\left(E_{1,n}-E_{2,n}\right)}_{=\frac{\sigma}{\epsilon_0}}\vbh{u}_n=\frac{\sigma}{\epsilon_0}\vbh{u}_n\qedhere
\end{equation*}
\end{demonstration}
\section{Leggi di Maxwell per l'elettrostatica}
A questo punto siamo arrivati ad avere tutti gli strumenti e i risultati necessari per enunciare le \textbf{leggi di Maxwell relative all'elettrostatica}.\\
Tuttavia, prima di far ciò è importante riprendere in mano alcuni risultati matematici e fisici che ci serviranno a tal scopo.
\begin{remember}~\\
		\begin{tabular}{p{0.47\textwidth}p{0.47\textwidth}}
			\begin{itemize}
				\item[1a] \textbf{Teorema della divergenza:} si consideri un volume $V\subseteq\realset^3$ compatto con bordo liscio $\partial V$. Dato un campo vettoriale differenziabile $\vba{G}$ in un intorno di $V$, allora
				\begin{equation*}
					\int_V\div{\vba{G}}=\int_{\partial V}\vba{G}\vdot \vbh{u}_nd\Sigma
				\end{equation*}
				o, equivalentemente,
				\begin{equation*}
					\int_V\div{\vba{G}}=\Phi_\Sigma\left(\vba{G}\right)
				\end{equation*}
			\end{itemize} &
			\begin{itemize}
			\item[1b] \textbf{Teorema del rotore:} si consideri una curva $\funz[\gamma]{\left[a,b\right]}{\realset^3}$ semplice - ossia senza intersezioni con sé stessa, chiusa e liscia a tratti; si consideri inoltre una superficie $\Sigma$ liscia tale che $\partial \Sigma=\gamma$. Dato un campo vettoriale differenziabile $\vba{G}$ in un intorno di $V$, allora
			\begin{equation*}
				\int_\Sigma\grad{\vba{G}}\vdot\vbh{u}_nd\Sigma=\oint_{\gamma}\vba{G}\vdot d\vba{s}
			\end{equation*}
			o, equivalentemente,
			\begin{equation*}
				\Phi_\Sigma\left(\grad{\vba{G}}\right)=\Gamma_\gamma\left(\vba{G}\right)
			\end{equation*}
		\end{itemize}\\
			\begin{itemize}
				\item[2a] \textbf{Legge di Gauss:} il flusso del campo elettrostatico $\vba{E}$ attraverso un superficie \textit{chiusa} è eguale alla quantità di carica contenuta all'\textbf{interno} della superficie, comunque siano distribuite, divisa per $\epsilon_0$.
				\begin{equation*}
					\Phi_\Sigma\left(\vba{E}\right)=\frac{1}{\epsilon_0}\int_{V}\rho\left(\vba{r}\right)dV
				\end{equation*}
				dove $V$ è uno spazio delimitato da $\Sigma$, ossia tale che $\partial V=\Sigma$.
			\end{itemize} &
			\begin{itemize}
				\item[2b] \textbf{Circuitazione del campo elettrico nulla:} su ogni curva chiusa $\gamma$ nello spazio, la circuitazione del campo elettrostatico è nulla.
				\begin{equation*}
					\Gamma_{\gamma}(\vba{E})=0
				\end{equation*}
			\end{itemize}
		\end{tabular}
\end{remember}
\begin{theorema}[Equazioni di Maxwell per l'elettrostatica]
	Dato il campo elettrostatico $\vba{E}$ e una densità di carica $\rho$, valgono le seguenti relazioni:
		\begin{align}
			\div{\vba{E}}&= \frac{\rho}{\epsilon_0} &
			\curl{E}&= 0
		\end{align}
\end{theorema}
\begin{demonstration}
	Per ottenere la prima legge, partiamo dalla legge di Gauss
	\begin{equation}
		\Phi_\Sigma\left(\vba{E}\right)=\frac{q}{\epsilon_0}
	\end{equation}
scritta nella sua formulazione integrale:
	\begin{equation*}
		\int_{\partial V}\vba{E}\vdot\vbh{u}_nd\Sigma=\frac{1}{\epsilon_0}\int_{V}\rho\left(\vba{r}\right)dV
	\end{equation*}
Applichiamo il teorema della divergenza al primo membro:
\begin{equation*}
	\int_V\div{E}dV=\int_V\frac{\rho}{\epsilon_0}dV
\end{equation*}
Poiché questa relazione è vera per un qualunque volume $V$ arbitrario, si deve necessariamente avere uguaglianza degli integrandi, ottenendo
\begin{equation*}
	\div{E}=\frac{\rho}{\epsilon_0}
\end{equation*}
Per ottenere la seconda legge, partiamo dalla circuitazione nulla del campo elettrostatico
\begin{equation*}
	\Gamma_{\gamma}\left(\vba{E}\right)=0
\end{equation*}
scritta nella sua formulazione integrale:
\begin{equation*}
	\oint_{\gamma}\vba{E}\vdot d\vba{s}=0
\end{equation*}
Applicando il teorema del rotore al membro non nullo:
\begin{equation*}
	\int_{\Sigma}\grad{\vba{E}}\vdot \vbh{u}_nd\Sigma=0
\end{equation*}
Poiché questa relazione è vera per una qualunque superficie $\Sigma$ arbitraria, si deve necessariamente avere che l'unico termine non dipendente dalla superficie sia sempre nullo.
\begin{equation*}
	\grad{\vba{E}}=0\qedhere
\end{equation*}
\end{demonstration}
\begin{observe}
	Mentre la prima equazione vale in generale, la seconda vale solo in elettrostatica, considerando un campo elettrico statico e in assenza di campo magnetico.\\
	Nel caso generale, vedremo che il rotore del campo elettrico dipende dalla variazione temporale del campo magnetico.
\end{observe}
\begin{examplewt}[Sfera uniformemente carica]
	Verifichiamo che vale la prima equazione dell'elettrostatica nel caso del campo generato da una sfera carica uniformemente:
	\begin{equation*}
		\vba{E}(r)=\begin{cases}
			\frac{\rho R}{3\epsilon_0}\vbh{u}_r&\text{se}\ r\leq R\\
			\frac{q}{4\pi\epsilon_0 R^2}\vbh{u}_r &\text{se}\ r\geq R
		\end{cases}
	\end{equation*}
	Data la divergenza di un campo in coordinate sferiche\footnote{Nelle ‘‘XXX'', a pagina \pageref{DivergenzaSferiche} è possibile trovare a grandi linee il procedimento per ricavare la divergenza in coordinate sferiche.}, che ricordiamo essere
	\begin{equation*}
		\div{\vba{G}}=\frac{1}{r^2}\pdv{r}\left(r^2G_r\right)+\frac{1}{r\sin\theta}\pdv{\theta}\left(G_\theta\sin\theta\right)+\frac{1}{r\sin\theta}\pdv{G_\phi}{\phi}
	\end{equation*}
	allora si ha, per un punto \textit{esterno} alla sfera (in cui \textit{non} c'è alcuna densità di corrente) vale
	\begin{equation}
		\div{\vba{E}}=\div{\frac{q}{4\pi\epsilon_0 R^2}\vbh{u}_r}=\frac{1}{r^2}\pdv{r}\left(r^2\frac{q}{4\pi\epsilon_0 R^2}\right)=0
	\end{equation}
	mentre per un punto \textit{interno} alla sfera (in cui si ha una densità di corrente $\rho$) vale
	\begin{equation*}
		\div{\vba{E}}=\div{\left(\frac{\rho R}{3\epsilon_0R^2}\vbh{u}_r\right)}=\frac{1}{r^2}\pdv{r}\left(r^2\frac{\rho R}{3\epsilon_0R^2}\right)=\frac{1}{\Ccancel[red]{r^2}}\frac{\Ccancel[red]{3}\Ccancel[red]{r^2}}{\Ccancel[red]{3}\epsilon_0}=\frac{\rho}{\epsilon_0}
	\end{equation*}
\end{examplewt}
\begin{comment}
	\begin{digression} % TO DO: spostare alla fine delle equazioni
		Non dobbiamo pensare che le equazioni di Maxwell sono una conseguenza di tutti i risultati e gli esperimenti visti fin'ora, ma semmai il \textit{contrario}: sono le equazioni di Maxwell che sono così \textit{fondamentali} da descrivere l'intera teoria elettromagnetica.\\ Volendo, si poteva trattare l'elettromagnetismo enunciando le equazioni di Maxwell e solo successivamente \textit{declinare} i risultati particolari da quelle; la scelta di fare l'esatto opposto è stata fatta per seguire grossomodo l'\textit{approccio storico} alla faccenda. % TO DO: completare in modo più interessante.
	\end{digression}
\end{comment}
\section{Equazione di Poisson e di Laplace}
L'irrotazionalità del campo elettrostatico garantita dalla seconda equazione di Maxwell ci dice che, almeno localmente, è anche conservativo:
\begin{equation*}
	\curl{\vba{E}}=0\iff \exists V\colon \vba{E}=-\grad{V}
\end{equation*}
Sostituendo nella prima legge di Maxwell, ossia la legge di Gauss, otteniamo
\begin{equation*}
	\grad{E}=\frac{\rho}{\epsilon}\implies\div{\grad{V}}=-\frac{\rho}{\epsilon_0}
\end{equation*}
che è un'equazione alle derivate parziali detta \textbf{equazione di Poisson}\index{equazione!di Poisson}.
\begin{equation}
	\laplacian{V}=-\frac{\rho}{\epsilon_0}
\end{equation}
Questa equazione differenziale ci descrive il potenziale in una regione dove è presente una sorgente di densità di carica $\rho$.

In una regione priva di cariche si ha $\rho\equiv0$; l'\textbf{equazione di Laplace}\index{equazione!di Laplace} descrive il potenziale in tale regione.
\begin{equation}
	\laplacian{V}=0
\end{equation}

Imponendo delle opportune \textit{condizioni di contorno}, che siano di natura \textit{fisica} o imposte come tali per \textit{convenzione}, potremmo idealmente ricavare le soluzioni di queste equazioni e determinare in modo prettamente matematico il potenziale - e di conseguenza anche il campo elettrostatico. Il problema principale è che \textit{non sempre} è possibile trovare facilmente una soluzione; tuttavia, per alcuni specifici casi, ad esempio campi che presentano delle \textit{simmetrie} interessanti, possiamo calcolare senza troppi problemi il potenziale.
\paragraph{Equazioni di Poisson e di Laplace con simmetria sferica}
Consideriamo un campo a simmetria sferica, ossia dipendente esclusivamente dalla distanza radiale:
\begin{equation*}
	V(\vba{r})=V(r)\vbh{u}_r
\end{equation*}
Dato che il laplaciano in coordinate sferiche è
	\begin{equation*}
		\laplacian=\frac{1}{r^2}\pdv{r}\left(r^2\pdv{r}\right)+\frac{1}{r^2\sin\theta}\pdv{\theta}\left(\sin\theta\pdv{\theta}\right)+\frac{1}{r^2\sin^2\theta}\pdv[2]{\phi}
	\end{equation*}
allora in un punto dello spazio dove \textit{non} c'è densità di carica si ha potenziale dato dalla soluzione dell'\textit{equazione di Laplace}
\begin{equation*}
	\frac{1}{r^2}\pdv{r}\left(r^2\pdv{r}V(r)\right)=0
\end{equation*}
Facendo gli opportuni calcoli...
\begin{gather*}
	\pdv{r}\left(r^2\pdv{r}V(r)\right)=0\\
	r^2\pdv{r}V(r)=A\\
	\pdv{r}V(r)=\frac{A}{r^2}\\
\end{gather*}
... otteniamo il potenziale
\begin{equation}
	V(r)=-\frac{A}{r}+B
\end{equation}
dove $A$ e $B$ sono costanti date dalle condizioni al contorno.

In un punto dello spazio dove c'è densità di carica $\rho$ si ha potenziale dato dalla soluzione dell'\textit{equazione di Poisson}
\begin{equation*}
	\frac{1}{r^2}\pdv{r}\left(r^2\pdv{r}V(r)\right)=-\frac{\rho}{\epsilon_0}
\end{equation*}
Consideriamo il caso di $\rho$ costante, per semplicità. Facendo gli opportuni calcoli...
\begin{gather*}
	\pdv{r}\left(r^2\pdv{r}V(r)\right)=-\frac{\rho}{\epsilon_0}r^2\\
	r^2\pdv{r}V(r)=-\frac{\rho}{3\epsilon_0}r^3+C\\
	\pdv{r}V(r)=-\frac{\rho}{3\epsilon_0}r+\frac{C}{r^2}\
\end{gather*}
... otteniamo il potenziale
\begin{equation}
	V(r)=-\frac{\rho}{6\epsilon_0}r^2-\frac{C}{r}+D
\end{equation}
dove $A$ e $B$ sono costanti date dalle condizioni al contorno.
\begin{examplewt}[Sfera uniformemente carica]
	Il campo elettrostatico della sferica uniformemente carica di raggio $R$ è un campo a simmetria radiale, pertanto il potenziale soddisfa, all'interno e all'esterno della sfera, le equazioni di Poisson e Laplace trovate prima.
	\begin{equation}
		V(r)=
		\begin{cases}
			-\frac{\rho}{6\epsilon_0}r^2-\frac{C}{r}+D&\text{se}\ r\leq R\\
			-\frac{A}{r}+B&\text{se}\ r\geq R
		\end{cases}
	\end{equation}
	Ci basta ora imporre le condizioni al contorno.
	\begin{itemize}
		\item Per \textit{convenzione}, si suppone che il potenziale per $r\to\infty$ tenda a $0$, dato che il campo elettrico si considera trascurabili a enormi distanze.\footnote{Questo è lecito farlo perché lo \textbf{zero del potenziale} è arbitrario, grazie al fatto che il potenziale stesso è definito a meno di costanti: sostanzialmente, come decido di misurare il potenziale è una scelta di chi studia il sistema, anche se generalmente ci sono motivi fisici (come in questo caso) o geometrici per fare una certa scelta. Ciò non significa, tuttavia, che tale scelta è \textit{insignificante}, dato che \textit{ogni} valore del potenziale deve essere misurato tenendo conto di tale zero.}
		\begin{equation*}
			\lim_{r\to+\infty}V(r)=0
		\end{equation*}
		Imponendo ciò, si trova
		\begin{equation*}
			B=0
		\end{equation*}
		\item Quando siamo a grandi distanze, la sfera carica uniformemente è assimilabile ad una carica puntiforme, pertanto l'altra condizione al limite è che il campo elettrico della sfera all'esterno sia quello della sfera; da ciò è necessario imporre 
		\begin{equation*}
			A=-\frac{q}{4\pi\epsilon_0}
		\end{equation*}
	\end{itemize}
	\begin{itemize}% TO DO: controllare questa prima condizione: ad esempio in simmetria cilindrica per r=0 crolla il campo.
		\item Sulla base della continuità del potenziale, sul dominio del campo elettrico il potenziale si considera \textit{finito}, pertanto poniamo
		\begin{equation*}
			C=0
		\end{equation*}
		in modo da togliere il termine $\frac{1}{r}$, che renderebbe il potenziale infinito in $r=0$.
		\item Per garantire la continuità del potenziale si deve imporre
		\begin{equation*}
			V_{interno}(R)=V_{esterno}(R)
		\end{equation*}
			Risolvendo l'equazione
		\begin{equation*}
			-\frac{\rho}{6\epsilon_0}R^2+D=\frac{q}{4\pi\epsilon_0R}\implies D=\frac{q}{4\pi\epsilon_0R}+\frac{\rho}{6\epsilon_0}R^2
		\end{equation*}
	\end{itemize}
Il potenziale complessivo è unico ed è
\begin{equation}
	V(r)=
	\begin{cases}
		\frac{\rho}{6\epsilon_0}\left[R^2-r^2\right]+\frac{q}{4\pi\epsilon_0R}=\frac{\rho}{6\epsilon_0}\left[R^2-r^2\right]+\frac{\rho R^2}{3\epsilon_0}&\text{se}\ r\leq R\\
		\frac{q}{4\pi\epsilon_0r}=\frac{\rho R^3}{3\epsilon_0r}&\text{se}\ r\geq R
	\end{cases}
\end{equation}
\end{examplewt}
\paragraph{Equazione di Laplace con simmetria cilindrica}
Consideriamo un campo a simmetria cilindrica, ossia dipendente esclusivamente dalla distanza assiale:
\begin{equation*}
	V(\vba{r})=V(R)\vbh{u}_R
\end{equation*}
Dato che il laplaciano in coordinate cilindriche è
\begin{equation*}
	\laplacian=\frac{1}{R}\pdv{R}\left(R\pdv{R}\right)+\frac{1}{R^2}\pdv[2]{\theta}+\pdv[2]{z}
\end{equation*}
allora in un punto dello spazio dove non c'è densità di carica si ha potenziale dato dalla soluzione dell'\textit{equazione di Laplace}
\begin{equation*}
	\frac{1}{r}\pdv{r}\left(r\pdv{r}V(r)\right)=0
\end{equation*}
Facendo gli opportuni calcoli...
\begin{gather*}
	\pdv{r}\left(r\pdv{r}V(r)\right)=0\\
	r\pdv{r}V(r)=A\\
	\pdv{r}V(r)=\frac{A}{r}\\
\end{gather*}
... otteniamo il potenziale
\begin{equation}
	V(r)=A\log{r}+B
\end{equation}
dove $A$ e $B$ sono costanti date dalle condizioni al contorno.

% TO DO: add soluzione poisson?

