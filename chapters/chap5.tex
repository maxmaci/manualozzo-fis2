% SVN info for this file
\svnidlong
{$HeadURL$}
{$LastChangedDate$}
{$LastChangedRevision$}
{$LastChangedBy$}

\chapter{La corrente elettrica}
\labelChapter{teoriamisura}

\begin{introduction}
	‘‘Se la tua nuova teoria si può esprimere con grande semplicità, allora esisterà un'eccezione patologica ad esso!'' % TO DO: quotes
\begin{flushright}
	\textsc{Adrian Mathesis,} adirato per essere stato bocciato da Riemann.
\end{flushright}
\end{introduction} % TO DO: intro
\lettrine[findent=1pt, nindent=0pt]{N}{ei} capitoli precedenti abbiamo esplorato i fenomeni elettrostatici. 
\section{Corrente elettrica e forza elettromotrice}
% Ci sono diversi tipi di conduttori. % TO DO: intro buona
I conduttori \textit{metallici} sono costituiti, a livello microscopico, da un \textit{reticolo spaziale} i cui vertici sono \textit{ioni positivi}, cioè atomi che hanno perso uno o più elettroni, e al cui interno si muovono gli \textit{elettroni liberi}, gli unici portatori di carica nei metalli.
Ciascuno di questi elettroni è libero di muoversi in una sua direzione e con una propria velocità, dovute alla situazione termica dell'oggetto. Non è evidentemente fattibile studiare il moto di \textit{ogni} singolo elettrone, dato che il numero di elettroni liberi in un conduttore è estremamente elevato.
\begin{examplewt}[Elettroni liberi in diversi materiali]
	Ricordiamo che la \textbf{costante di Avogadro}\index{costante!di Avogadro} è definito come il numero di particelle per mole di un qualunque materiale:
	\begin{equation*}
		N_A=\SI[exponent-product=\ensuremath{\cdot}]{6,022e23}{\per\mole}
	\end{equation*}
	In questo caso, dato che per ogni atomo di metallo c'è generalmente un elettrone libero, questo numero corrisponde al numero di \textit{elettroni liberi} in una mole di un certo elemento chimico.\\
	Definiamo $\rho$ la densità del materiale e $A$ il numero di massa, cioè quanti grammi pesa una mole del materiale; possiamo calcolare la \textbf{densità di carica} in diversi materiali.
	\begin{itemize}
		\item \textbf{Rame (Cu)}% TO DO: do we need mchem package?
		\begin{equation*}
			n_{\textrm{Cu}}=\frac{N_A\cdot\rho_{\mathrm{Cu}}}{A_{\mathrm{Cu}}}=\frac{\SI[exponent-product=\ensuremath{\cdot}]{6,022e23}{\per\mole}\cdot\SI[exponent-product=\ensuremath{\cdot}]{8,96e3}{\kilogram\per\cubic\meter}}{\SI[exponent-product=\ensuremath{\cdot}]{63,55e-3}{\kilogram\per\mole}}=\SI[per-mode = fraction,exponent-product=\ensuremath{\cdot}]{8,49e28}{\mathrm{el}\per\cubic\metre}
		\end{equation*} 
		\item \textbf{Argento (Ag)}% TO DO: do we need mchem package?
		\begin{equation*}
			n_{\textrm{Ag}}=\frac{N_A\cdot\rho_{\mathrm{Ag}}}{A_{\mathrm{Ag}}}=\frac{\SI[exponent-product=\ensuremath{\cdot}]{6,022e23}{\per\mole}\cdot\SI[exponent-product=\ensuremath{\cdot}]{10,5e3}{\kilogram\per\cubic\meter}}{\SI[exponent-product=\ensuremath{\cdot}]{107,87e-3}{\kilogram\per\mole}}=\SI[per-mode = fraction,exponent-product=\ensuremath{\cdot}]{5,86e28}{\mathrm{el}\per\cubic\metre}
		\end{equation*}
	L'ordine di grandezza è lo stesso per tutti i conduttori metallici.
	\end{itemize} 
\end{examplewt}
Ci conviene studiare il moto medio degli $N$ elettroni nel materiale. Tuttavia, in \textit{assenza} di un campo elettrico non percepiamo alcun movimento preferenziale degli elettroni: ogni elettrone si muove in modo del tutto \textit{casuale} e dunque la somma dei loro moti, e di conseguenza la velocità media, sarà \textit{nulla}:
\begin{equation}
	\left<\vba{v}\right>=\frac{1}{N}\sum_{i=1}^{N}\vba{v}_i=0
\end{equation}
Consideriamo invece la seguente situazione: prendiamo un conduttore con potenziale $V_1$ e lo colleghiamo ad un altro conduttore con potenziale $V_2>V_1$ tramite un filo trascurabile. Sappiamo che il campo elettrico tra i conduttori è diretto dal potenziale maggiore al minore; gli elettroni - portatori di carica negativi - andranno dal primo conduttore verso il secondo tramite il filo che li collega in un tempo il cui limite inferiore è dell'ordine di
\begin{equation*}
	t\sim\frac{d}{c},
\end{equation*}
dove $d$ è la lunghezza del filo e $c$ la \textit{velocità della luce}, fino a raggiunge l'equilibrio.
\begin{equation*}
	V_1'=V_2'=V
\end{equation*}
Siamo in presenza del fenomeno di \textbf{conduzione elettrica}\index{conduzione!elettrica}.
\begin{define}[Corrente elettrica]
	Un moto \textit{ordinato} di elettroni liberi di un conduttore in una certa direzione è detto una \textbf{corrente elettrica}.
\end{define}
Dato che la velocità della luce è estremamente elevata, l'equilibrio è raggiunto quasi istantaneamente e la corrente elettrica è di breve vita. Per poter indurre un moto consistente e \textit{duraturo} di cariche dobbiamo \textit{mantenere} una differenza di potenziale.

Per far ciò, ci serve un \textbf{generatore di forza elettromotrice}\index{generatore!di forza elettromotrice} ($\mathrm{f.e.m.}$), un marchingegno che trasforma energia \textit{non} elettrica in energia elettrica tale da avere, ai capi del generatore, una differenza di potenziale $\Delta V$, \textit{indipendentemente} da cosa ci si collega.
\begin{digressionwt}[Pila di Volta]
	Il primo generatore di questo tipo fu la \textbf{pila di Volta}\index{pila di Volta}. Tale generatore consisteva, nella sua forma più semplice costituita da una singola \textit{cella}, in un disco di \textit{zinco} (lo chiameremo \textbf{anodo}) e uno di \textit{rame} (il \textbf{catodo}), separate da una stoffa imbevuta di una soluzione elettrolitica come acqua e acido solforico.
	% TO DO: insert image
	Lo zinco sulla superficie dell'anodo è ossidato dalla soluzione elettrolitica e si dissolve nell'elettrolita come ioni carichi positiva, lasciando due elettroni liberi nel metallo.
	\begin{equation*}
		\text{anodo (ossidazione)}\colon \mathrm{Zn}\to\mathrm{Zn}^{2+}+2\mathrm{e}^{-}
	\end{equation*}
	Quando lo zinco entra nell'elettrolite, due atomi positivi di idrogeno dell'elettrolite accettano due elettroni dalla superficie dal catodo di rame, riducendoci ad una molecola di idrogeno non carica.
	\begin{equation*}
		\text{catodo (riduzione)}\colon2\mathrm{H}^{+}+2\mathrm{e}^{-}\to\mathrm{H}_2
	\end{equation*}
	Gli elettroni usati nel rame per formare le molecole di ossigine provengono da un filo esterno che lo collega al disco di zinco; l'idrogeno prodotto nella riduzione si disperde in forma gassosa.

	Misurando la $\mathrm{d.d.p.}$ tra i dischi si osserva un valore fisso di circa $\SI{0,76}{\volt}$; se impilassimo più celle la differenza di potenziale aumenterebbe. Il valore della  $\mathrm{d.d.p.}$ misurata dipende dalla coppia di metalli scelti.
\end{digressionwt}
% L'energia elettrica è una forma di energia più "nobile", che si trasforma bene in altre forme di energia come il calore, ma che si ottiene difficilmente da altre forme, come dal calore o da reazioni chimiche, senza notevole dispersione di energia.
\subsection{Intensità di corrente}
\begin{define}[Intensità di corrente]
	L'\textbf{intensità di corrente elettrica}\index{intensità di corrente elettrica} è definita come la rapidità con cui fluiscono delle cariche attraverso una certa superficie $\Sigma$:
	\begin{equation}
		I=\lim_{\Delta t}\frac{\Delta q}{\Delta t}=\dv{q}{t}
	\end{equation}
\end{define}
\begin{attention}
	La corrente viene studiata per ragioni storiche supponendo che a muoversi siano \textit{cariche positive}, anche se nella maggior parte dei materiali che conducono corrente elettrica (ad esempio, i metalli) la corrente è portata da \textit{cariche negative}. 
\end{attention}
\begin{define}[Velocità di deriva]
	La \textbf{velocità di deriva}\index{velocità!di deriva} è la velocità media che hanno $N$ particelle cariche in un materiale a causa di un campo elettrico $\vba{E}$:
	\begin{equation}
		\vba{v}_d=\frac{1}{N}\sum_{i=1}^{N}\vba{v}_i
	\end{equation}
	La velocità di deriva ha la stessa direzione del campo elettrico.
\end{define}
I concetti di velocità di deriva e intensità di corrente, come possiamo facilmente immaginare, sono strettamente correlati. Consideriamo un filo conduttore percorso da da portatori di carica con carica $e$: essendo in presenza di una forza elettromotrice - e quindi di un campo elettrico - le cariche si muovono mediamente con velocità di deriva $\vba{v}_d$ e percorreranno, in un intervallo infinitesimo di tempo $\Delta t$, un tratto di filo
\begin{equation*}
	d=\abs{\vba{v}_d}\Delta t
\end{equation*}
La carica complessiva che passa attraverso una superficie infinitesima $d\Sigma$ in un tempo $\Delta t$ è quella contenuta in $\Delta V$, che corrisponde al volume di un cilindro infinitesimo di basi $d\Sigma\vba{u_n}\vdot\vbh{u}_d$ e altezza $d$, dove $\vba{u}_n$ è il versore normale alle superficie e $\vbh{u}_d$ è il versore nella direzione e verso della velocità di deriva:
\begin{align*}
	dV&=d\Sigma\vba{u_n}\vdot\vbh{u}_d d=d\Sigma\vba{u_n}\vdot\vbh{u}_d \abs{\vba{v}_d}\Delta t\\
	&=d\Sigma\vba{u_n}\vdot\vba{v}_d \Delta t =\\
	&=d\Sigma\cos\theta\abs{\vba{v}_d}\Delta t
\end{align*}
\begin{align*}
	\Delta q&=n_{+}edV=n_{+}e d\Sigma\vba{u_n}\vdot\vba{v}_d \Delta t=\\
	&n_{+}e  d\Sigma\cos\theta\abs{\vba{v}_d}\Delta t
\end{align*}
Qui $\theta$ è l'angolo tra i due versori, $n_{+}$ la densità di cariche positive liberi per unità di volume ed $e$ la carica delle particelle libere. La carica di corrente infinitesima è dunque
\begin{align*}
	dI=\frac{\Delta q}{\Delta t}&=n_{+}edV=n_{+}e d\Sigma\vba{u_n}\vdot\vba{v}_d=\\
	&=n_{+}e  d\Sigma\cos\theta\abs{\vba{v}_d}
\end{align*}
Possiamo semplificare questa notazione introducendo il concetto di \textbf{densità di corrente}. 
\begin{define}[Densità di corrente]
	La \textbf{densità di corrente}\index{densità!di corrente} è il campo vettoriale che ad ogni punto in un conduttore associa un vettore, la cui direzione è la velocità di deriva delle cariche \textit{positive} in tal punto e il cui modulo è pari alla quantità di carica che attraversa in un unità di tempo un unità di area della sezione perpendicolare del conduttore in tal punto. In altre parole,
	\begin{equation}
		\vba{j}=n_{+}e\vba{v}_d
	\end{equation}
\end{define}
Riscrivendo,
\begin{equation*}
	dI=\vba{j}\vdot \vba{u_n}d\Sigma
\end{equation*}
da cui si ottiene che l'intensità di corrente attraverso una superficie finita $\Sigma$ è
\begin{equation}
	I=\int_{\Sigma}\vba{j}\vdot \vba{u_n}d\Sigma=\Phi_{\Sigma}\left(\vba{j}\right)
\end{equation}
Se, come nei conduttori metallici, i portatori di carica (con carica $-e$) sono negativi, la velocità di deriva $\vba{v}_{d,-}$ ha stessa direzione e verso opposto del campo elettrico; la densità di corrente è invece nella stessa direzione del campo elettrico perché la carica è negativa:
\begin{equation}
	\vba{j}=-n_{-}e\vba{v}_{d,-}
\end{equation}
Se consideriamo invece fluidi ionizzati, soluzioni elettrolitiche o semiconduttori, i portatori di carica sono di segno misto e possono avere velocità di deriva differenti $\vba{v}_{d,+}$ e $\vba{v}_{d,-}$. La densità è ottenuta come una somma vettoriale di due quantità concordi che hanno lo stesso verso del campo elettrico:
\begin{equation}
	\vba{j}=n_{+}e\vba{v}_{d,+}-n_{-}e\vba{v}_{d,-}
\end{equation}
\paragraph{Unità di misura}
\begin{units}~\\
	\textbf{\textsc{Corrente elettrica:}} ampere ($\unit{\ampere}$).\\
	\textit{\textbf{Dimensioni:}} $[I]=\mathsf{I}$
\end{units}
L'\textbf{ampere}\index{ampere} - e non il \textit{coulomb}, come ci si potrebbe aspettare - è l'unica unità di misura fondamentale del SI che introdurremo in questa trattazione. Come precedentemente detto, $\SI{1}{\coulomb}$ è definito come la carica che una corrente da $\SI{1}{\ampere}$ attraversa una data superficie in $\SI{1}{\second}$.
Nella pratica sono utilizzati i suoi \textit{sottomultipli}, ad esempio:
\textit{sottomultipli}, ad esempio:
\begin{itemize}
	\item \textit{milliampere}: $\SI[per-mode = fraction,exponent-product=\ensuremath{\cdot}]{1}{\milli\ampere}=\SI[per-mode = fraction,exponent-product=\ensuremath{\cdot}]{e-3}{\ampere}$.
	\item \textit{microampere}: $\SI[per-mode = fraction,exponent-product=\ensuremath{\cdot}]{1}{\micro\ampere}=\SI[per-mode = fraction,exponent-product=\ensuremath{\cdot}]{e-6}{\ampere}$.
	\item \textit{nanoampere}: $\SI[per-mode = fraction,exponent-product=\ensuremath{\cdot}]{1}{\nano\ampere}=\SI[per-mode = fraction,exponent-product=\ensuremath{\cdot}]{e-9}{\ampere}$.
\end{itemize}
\begin{units}~\\
	\textbf{\textsc{Densità di corrente:}} ampere su metro quadro $\left(\unit[per-mode = fraction]{\ampere\per\square\meter}\right)$.\\
	\textit{\textbf{Dimensioni:}} $[j]=\dfrac{[I]}{[\Sigma]}=\mathsf{I}\mathsf{L}^{-2}$
\end{units}
\subsection{Conservazione della carica e corrente elettrica}
Data una densità $\vba{j}$, abbiamo trovato che la carica totale che passa nell'unità di tempo attraverso un volume $V$ è data dal flusso della densità tramite il bordo della $\Sigma$:
\begin{equation*}
	I=\int_{\Sigma}\vba{j}\vdot\vbh{u}_nd\Sigma
\end{equation*}
Sulla superficie, l'integrando $\vba{j}\vdot\vbh{u}_n$ non ha necessariamente segno costante; dato che $\vba{j}$ ha direzione dipendente dalla carica di deriva, segno diversi corrispondono a due situazioni differenti.
\begin{itemize}
	\item quando le cariche \textit{negative entrano} la superficie o le cariche \textit{positive escono}, $\vba{j}\vdot\vbh{u}_n<0$.
	\item  quando le cariche \textit{negative escono} la superficie o le cariche \textit{positive entrano}, $\vba{j}\vdot\vbh{u}_n>0$.
\end{itemize}
Dal \textit{principio di conservazione della carica} nessuna carica nel conduttore (isolato) si può annichilire e sparire, e quindi la carica complessiva deve rimanere \textit{costante}. Se la carica interna $q_{int}$ alla superficie diminuisce, tale carica deve essere uscita dalla superficie e quindi ha cambiato l'intensità di corrente $I$ che l'attraversa; in particolare, essa dovrà corrispondere alla \textit{variazione temporale} della \textit{carica interna}
\begin{equation}
	I=\int_{\Sigma}\vba{j}\vdot\vbh{u}_nd\Sigma=-\pdv{q_{int}}{t}
\end{equation}
Il meno nell'espressione è dato dal fatto che se l'integrale è complessivamente positivo, ciò corrisponde ad una \textit{diminuzione} della carica interna - che ricordiamo si considera rispetto a quella positiva - e pertanto ha derivata negativa.

Se $V$ è il volume racchiuso da una superficie chiusa $\Sigma$, la carica interna è ovviamente
\begin{equation*}
	q_{int}=\int_V\rho dV
\end{equation*}
Possiamo derivare temporalmente entrambi i termini e scambiare\footnote{Non motiveremo \textit{rigorosamente} perché si può fare questo scambio. Gli analisti si mettano il cuore in pace.} integrale e derivata in quanto il volume è invariante temporalmente.
\begin{equation*}
	\int_{\Sigma}\vba{j}\vdot\vbh{u}_nd\Sigma=-\pdv{q_int}{t}=-\int_V\pdv{\rho}{t} dV % TO DO: completare
\end{equation*}
Usando il \textit{teorema della divergenza} si ha, per qualsiasi volume $V$, che
\begin{equation*}
	\int_V\div{\vba{j}}dV=-\int_V\pdv{\rho}{t} dV
\end{equation*}
ottenendo quindi l\textbf{equazione di continuità}\index{equazione!di continuità}:
\begin{equation}
	\div{\vba{j}}+\pdv{\rho}{t}=0
\end{equation}
\begin{digression}
	Questa relazione vale in generale per descrivere il trasporto di una certa quantità \textit{conservata}, energia e quantità di moto.
	Nello specifico, $\rho$ è l'ammontare della quantità per unità di volume (una densità), mentre $\vba{j}$ è il flusso della quantità.\\
	Se la quantità \textit{non} si conserva, la legge si generalizza come
	\begin{equation}
		\div{\vba{j}}+\pdv{\rho}{t}=\sigma
	\end{equation}
dove $\sigma$ rappresenta quanta quantità viene generata (se positiva) o rimossa (se negativa) per unità di volume e per unità di tempo.
\end{digression}
\paragraph{Il caso stazionario}
	Consideriamo il \textit{caso stazionario} dell'equazione di continuità, ossia quando la densità di carica $\rho$ risulta essere costante nel tempo. Dall'equazione di continuità segue  che
	\begin{equation*}
		\div{\vba{j}}=0
	\end{equation*}
	ossia la densità di corrente è un \textbf{campo solenoidale}\footnote{Vedi capitolo XXX yadda yadda}. Consideriamo una porzione tubulare di conduttore come in figura.
	% TO DO: add figura
	Sappiamo che il flusso sulla superficie che delimita questa porzione è nullo, ma esso è determinato completamente dai flusso sulle sezioni perpendicolari del conduttore\footnote{La densità di corrente sulla superficie laterale del tubo è tangente e quindi il suo flusso è nullo.}
	\begin{equation*}
		0=\oint_{\Sigma}\vba{j}\vdot\vbh{u}_nd\Sigma=\int_{\Sigma_1}\vba{j}_1\vdot\vbh{u}_nd\Sigma_1+\int_{\Sigma_2}\vba{j}_2\vdot\vbh{u}_nd\Sigma_2
	\end{equation*}
	Ma se consideriamo le intensità di correnti lungo le sezioni...
	\begin{align*}
		I_1&\coloneqq\int_{\Sigma_1}\vba{j}_1\vdot\vbh{u}_1d\Sigma_1=-\int_{\Sigma_1}\vba{j}_1\vdot\vbh{u}_nd\Sigma_1\\
		I_2&\coloneqq\int_{\Sigma_2}\vba{j}_2\vdot\vbh{u}_2d\Sigma_2=\int_{\Sigma_2}\vba{j}_2\vdot\vbh{u}_nd\Sigma_1
	\end{align*}
	...allora segue che $0=I_1-I_2$, ossia
	\begin{equation}
		I_1=I_2
	\end{equation}
Nel caso stazionario, l'intensità di corrente è \textbf{sempre costante}.

Riprendendo l'analogia con fluidodinamica tra intensità di corrente e portata, la portata è costante nei fluidi incomprimibili, cioè quelli la cui densità di massa è $\rho=\mathrm{const}$.
\section{Legge di Ohm}
Ad inizio del capitolo, abbiamo descritto brevemente come sono costituiti i conduttori metallici. Tale descrizione, in realtà, corrisponde al \textbf{modello di Drude-Lorentz} della \textit{conduzione elettrica} proposto dal fisico tedesco Paul \textbf{Drude} nel 1900 ed espanso nel 1905 dall'olandese Hendrick Antoon \textbf{Lorentz}. Il modello permette di spiegare, nell'ambito della teoria classica dell'elettromagnetismo, le proprietà di trasporto degli elettroni nei materiali - in particolare i metalli - tramite la teoria cinetica: gli elettroni si comportano, secondo questa interpretazione, in modo molto simile ad un \textit{flipper}\footnote{Per gli amici d'oltreoceano o per coloro che si divertivano ai giochi pre-installati su \textit{Windows XP}\texttrademark\ , un \textit{pinball}.}, in cui elettroni rimbalzanti \textit{urtano} continuamente con un reticolo cristallini di ioni fissi.
\begin{digression}
	Tale modello fu integrato nel 1933 da Arnold Sommerfeld and Hans Bethe con i risultati della teoria quantistica nel \textbf{modello di Drude-Sommerfeld}. Una differenza, ad esempio, è che il modello non parla esplicitamente di ioni o di reticoli cristallini, la cui assenza viene giustificata in termini quantistici.
\end{digression}
Approfondiamo meglio questo modello. Gli elettroni liberi si muovono attraverso il reticolo cristallino in modo completamente disordinato; nel loro moto, gli elettroni si vanno a scontrare continuamente con gli ioni in interazioni che chiamiamo \textit{urti}. Tra un urto e il successivo il moto è libero e in traiettoria rettilinea, cosicché il moto degli elettroni si possa rappresentare come un spezzata di segmenti con direzione e verso variabili. Senza campo elettrico \textit{non} c'è una direzione privilegiata e quindi una corrente.

Si può definire
\begin{itemize}
	\item un \textbf{tempo medio di percorrenza} $\tau$.
	\item un \textbf{cammino libero medio} $\mathcal{l}$ tra due urti successivi
\end{itemize}
che sono legati tra di loro dalla velocità media $v$ degli elettroni nel metallo.
\begin{equation}
	\mathcal{l}=v\tau
\end{equation}
Da queste supposizioni microscopiche possiamo derivare una legge microscopica. All'applicazione di un campo elettrico $\vba{E}$ non si muoverà più di moto rettilineo uniforme, ma subisce un accelerazione
\begin{equation*}
	\vba{a}=\frac{\vba{F}}{m_{e}}=-\frac{e}{m_{e}}\vba{E}
\end{equation*}
dove $m_e$ è la massa dell'elettrone. Alla distribuzione casuale delle velocità si sovrappone quindi una velocità data da questa accelerazione; poiché è più piccola rispetto a quella che l'elettrone possiede di per sé, il tempo medio $\tau$ non cambia in modo significativo. Se $\vba{v}_i$ è la velocità dopo un urto, la velocità subito prima l'urto successivo sarà
\begin{equation*}
	\vba{v}_{i+1}=\vba{v}_i-\frac{e}{m_{e}}\vba{E}\tau
\end{equation*}
Calcoliamo la velocità media su $N$ elettroni, con $N$ molto grande, in modo da definire la \textit{velocità di deriva} indotta dal campo elettrico:
\begin{equation*}
	\vba{v}_d=\frac{1}{N}\sum_{i=1}^{N}\vba{v}_{i+1}=\underbrace{\frac{1}{N}\sum_{i=1}^{N}\vba{v}_i}_{=0}-\frac{e}{m_{e}\tau}\vba{E}=-\frac{e}{m_{e}\tau}\vba{E}
\end{equation*}
La media delle velocità dopo l'urto è casuale e perciò è nulla; la velocità di deriva è quindi
\begin{equation}
	\vba{v}_d=-\frac{e}{m_{e}}\vba{E}\tau
\end{equation}
Se $n_{-}$ la densità di elettroni liberi per unità di volume, la densità di corrente che consegue a questo modo ordinato è
\begin{equation}
	\vba{j}=-n_{-}e\vba{v}_d=\frac{n_{-}e^2\tau}{m_{e}}\vba{E}=\sigma\vba{E}
\end{equation}
dove
\begin{equation*}
	\sigma=\frac{n_{-}e^2\tau}{m_{e}}
\end{equation*}
è una grandezza caratteristica del materiale nota come \textbf{conduttività}.
L'equazione
\begin{equation}
	\vba{j}=\sigma\vba{E}
\end{equation}
è nota come \textbf{legge di Ohm della conduzione elettrica}\index{legge!di Ohm}, dal fisico tedesco Georg \textbf{Ohm} che nel 1827 introdusse un caso specifico di tale equazione per spiegare dei risultati sperimentali da lui studiati. La legge si può scrivere anche nella forma
\begin{equation}
	\vba{E}=\rho\vba{j}
\end{equation}
dove
\begin{equation*}
	\rho=\frac{1}{\sigma}
\end{equation*}
è detta \textbf{resistività}.
\begin{define}[Conduttività e resistività]
	La \textbf{conduttività}\index{conduttività} è una grandezza associata ai conduttori definita come
	\begin{equation}
		\sigma=\frac{n_{-}e^2\tau}{m_{e}}
	\end{equation}
	che rappresenta la difficoltà della corrente a muoversi nel conduttore ed è caratteristica del \textit{materiale} con cui è fatto.\\
	Il valore
	\begin{equation}
		\rho=\frac{1}{\sigma}
	\end{equation}
	viene detto \textbf{resistività}\index{resistività} e rappresenta la difficoltà della corrente a muoversi nel conduttore.
\end{define}
Non tutti i conduttori rispettano questa legge, ma quelli che lo fanno sono detti \textbf{conduttori ohmici}\index{conduttore!ohmico}.
\begin{examplewt}[Esempi di conduttori ohmici e non ohmici]~
	\begin{itemize}
		\item \textbf{Ohmici:} fili conduttori di metalli come rame, o argento, resistenze ideali.
		\item \textbf{Non ohmici:} filamento di tungsteno delle lampade a incandescenza, diodi, semiconduttori.
	\end{itemize}
\end{examplewt}
\subsection{Legge di Ohm nei conduttori metallici}
Consideriamo un conduttore metallici cilindrico di lunghezza $d=\overline{AB}$ e sezione $\Sigma$. Ai capi del conduttore è applicata, con un generatore di $\mathrm{f.e.m.}$, una $\mathrm{d.d.p.}$ pari a
\begin{equation*}
	V=V_A-V_B=\int_{A}^{B}\vba{E}\vdot d\vba{s}=Ed>0
\end{equation*}
Il campo elettrico è costante (con modulo $\abs{\vba{E}}=E$) e diretto, parallelo all'asse del cilindro, da $A$ verso $B$; in regime stazionario, anche la densità di corrente $\vba{j}$ è costante (con modulo $\abs{\vba{j}}=j$) e scorre nella stessa direzione\footnote{Ciò è dovuto alla scelta storica di orientare la densità di corrente con le cariche positive. Gli elettroni, invece, percorrono il conduttore da $B$ verso $A$.}. Allora, dalla \textit{legge di Ohm}, segue che
\begin{equation*}
	I=\int_{\Sigma}\vba{j}\vdot\vbh{u}_nd\Sigma=j\Sigma=\sigma E\Sigma=\frac{V}{d}\sigma\Sigma=V\frac{\Sigma}{\rho d}
\end{equation*}
Definita la \textbf{resistenza} come
\begin{equation*}
	R=\frac{\rho d}{\Sigma}
\end{equation*}
si ottiene la \textbf{legge di Ohm}\index{legge!di Ohm} nella forma amata dagli elettrotecnici:
\begin{equation}
	V=IR
\end{equation}
La legge si può scrivere anche nella forma
\begin{equation}
	I=GV
\end{equation}
dove
\begin{equation*}
	G=\frac{1}{R}
\end{equation*}
è detta \textbf{conduttanza}.
\begin{define}[Resistenza e conduttanza]
	La \textbf{resistenza}\index{resistenza} è una grandezza associata ai conduttori metallici di lunghezza $d$ e sezione $\Sigma$, definita come
	\begin{equation*}
		R=\frac{\rho d}{\Sigma}
	\end{equation*}
	che rappresenta la difficoltà della corrente a muoversi nel conduttore. Il termine $\rho$ è la \textit{resistività} del conduttore e dipende dal materiale.\\
	Il valore
	\begin{equation}
		G=\frac{1}{R}
	\end{equation}
	viene detto \textbf{conduttanza}\index{conduttanza} e rappresenta la facilità della corrente a muoversi nel conduttore.
\end{define}
Un conduttore 
\paragraph{Unità di misura}
\begin{units}\index{ohm}~\\
	\textbf{\textsc{Resistenza elettrica:}} ohm ($\unit{\ohm}$) o volt su ampere $\left(\unit[per-mode = fraction]{\volt\per\ampere}\right)$.\\
	\textit{\textbf{Dimensioni:}} $[R]=\dfrac{[V]}{[I]}=\mathsf{M} \mathsf{L}^2  \mathsf{T}^{-3}\mathsf{I}^{-2}$
\end{units}
\begin{units}\index{siemens}~\\
	\textbf{\textsc{Conduttanza elettrica:}} siemens ($\unit{\siemens}$), mho  ($\mho$) o ampere su volt $\left(\unit[per-mode = fraction]{\ampere\per\volt}\right)$.\\
	\textit{\textbf{Dimensioni:}} $[G]=\dfrac{[I]}{[V]}=\mathsf{I}^{2}\mathsf{T}^{3}\mathsf{M}^{-1} \mathsf{L}^{-2}$
\end{units}
\begin{digressionwt}[\textit{E mho il siemens?} Una tragicommedia sulla conduttanza]
	Nei libri di testo contemporanei l'unità di misura associata alla conduttanza è il \textit{siemens}; tuttavia, in alcuni un po' datati è possibile trovare l'alquanto buffo \textit{mho}. Inoltre, se scartabellassimo libri ancora più vecchi troveremmo sì il \textit{siemens}, ma per indicare la resistenza! Che pasticcio hanno combinato i fisici con questa grandezza?\newline~\\
	Facciamo un po' d'ordine e torniamo indietro al 1860. L'ingegnere Werner Siemens era tra i proprietari di un'azienda che costruiva telegrafi in Germania, Russia e Regno Unito; per migliorare il loro funzionamento aveva bisogno di studiare a livello pratico la resistenza dei conduttori al passaggio della corrente, ma una buona unità di misura di tale grandezza, che sia riproducibile, non esisteva. Siemens propose lui stesso negli \textit{Annalen der Physik und Chemie} quella che diventerà nota come \textit{l'unità di mercurio del Dr. Siemens}: essa corrispondeva alla resistenza elettrica presente in una colonna di mercurio con lunghezza di un metro e sezione uniforme di $\SI{1}{\milli\metre\squared}$ mantenuta alla temperatura di zero gradi Celsius.\\
	Tale unità --- il cui nome completo ha quel non so che di cinema espressionista tedesco e potrebbe figurare bene come pellicola accanto a \textit{Il gabinetto del dottor Caligari} --- verrà semplicemente chiamata \textit{siemens}, nome che non causerà assolutamente alcuna confusione in futuro.\\
	Seppur sia simile, almeno concettualmente, ad altre unità basate sul mercurio come l'\textit{atmosfera}, in realtà si rivelò problematica proprio nel suo tentativo di essere riproducibile: per definire questa colonna di mercurio erano necessari dei tubi di vetro, i quali però non avevano sezioni costanti. Altri fattori come pressione dell'aria, umidità... potevano influenzare questa misurazione. Inoltre, non era coerente con altre unità di misura preesistenti! Nonostante questi problemi, fu comunque utilizzata per diversi anni.\newline~\\
	La ricerca di un'unità di misura migliore proseguì. L'anno successivo, il 1861, Latimer Clark e Sir Charles Bright presentarono un articolo all'\textit{Associazione Britannica per l'Avanzamento della Scienza}, suggerendo di creare uno standard per la resistenza e di chiamarla in onore del fisico tedesco Georg Ohm... chiamandola \textit{ohma}. Si stabilì subito una commissione, a cui parteciparono fisici dal calibro di James Clark Maxwell e Lord Kelvin, per inventare un unità che fosse coerente con il sistema metrico francese e pratica da utilizzare - a differenza di quella di Siemens.
	Nel terzo verbale della commissione, nel 1864, si riproposte di chiamarla in onore di Ohm e dunque si riferirono all'unità di misura come... \textit{ohmad}. A volte mi chiedo se i fisici ci sono o ci fanno. Solo nel 1867 il termine \textit{ohm} si userà in modo diffuso.
	
	A dir la verità, l'\textit{ohm} definito dall'Associazione Britannica non era così differente da quello di Siemens, dato che cambiava soltanto la lunghezza della colonna di mercurio da $\SI{100}{\centi\meter}$ a $\SI{104.7}{\centi\meter}$. Ciò nonostante, dato che avere due unità di misura differenti per una stessa grandezza era ridicolo, nel 1881 al \textit{Congresso Internazionale degli Elettricisti} si decise di compiere una scelta definitiva tra le due: l'unità di misura della resistenza non doveva essere il \textit{siemens}, ma l'\textit{ohm}... anche se nel frattempo la colonna di mercurio si allungò a $\SI{104.9}{\centi\meter}$.\newline % TO DO: aggiustare citazione
	
	\noindent Come ricordò Maxwell al convegno, ‘‘le dimensioni contano\textsuperscript{{$\left[\parbox{15em}{\tiny\textit{Fonte: Maxwell ‘‘Manualotso'', un mercoledì pranzo, in un anonimo 100Montaditos}}\right]$}}'' e negli anni l'unità di misura rimase la stessa, ma la colonnina di mercurio cambiò lunghezza più e più volte per adattarsi a studi sempre più analitici - stranamente non cambio lo spessore, ma evidentemente quello non contava più di tanto. La colonna di mercurio puro rimase lo standard fino alla \textit{Conferenza Generale sui Pesi e le Misure} del 1948, dove l'\textit{ohm} fu ridefinito in termini assoluti. Attualmente, il \textit{siemens} di Siemens vale circa $\SI{0.9537}{\ohm}$ moderni.
	
	Il povero Siemens, nonostante l'unità della resistenza non prese il suo nome, non si perse d'animo e continuò a sperimentare con l'elettromagnetismo: nel 1867 brevettò una delle prime dinamo industriali --- casualmente lo stesso giorno in cui Sir Charles Wheatstone brevettò una sua personale versione della dinamo. Successivamente, nel 1888 divenne nobile, trasformando il suo cognome in \textit{von Siemens}, e negli anni a seguire la sua azienda si espanse fino a diventare l'odierna multinazionale \textit{Siemens AG}.\newline~\\
	La resistenza elettrica aveva finalmente ottenuto un'unità di misura, ma ne mancava ancora una per la \textit{conduttanza}. O meglio, siccome tale grandezza era il reciproco della resistenza, mancava soltanto il nome dell'unità di misura: dopotutto, se i reciproci dei \textit{secondi} si chiamano \textit{hertz}, anche il reciproco della resistenza merita un nome, che diamine!
	
	Il primo ad accorgersi di cotale mancanza fu Lord Kelvin. Basandosi su alcune idee fornitegli dai suoi studenti,  nel 1883 Lord Kelvin propose al grande pubblico di utilizzare il termine \textit{mho}.\\
	Se non ve ne foste accorti, \textit{mho} è letteralmente \textit{ohm} letto al contrario - perché un \textit{mho} è il reciproco di un \textit{ohm}.\\
	Non fu l'unica proposta avanza in quell'incontro: Kelvin propose entusiasta - sempre su un idea di origine studentesca - che la corretta pronuncia di \textit{mho} si dovesse ottenere prendendo una registrazione di \textit{ohm} con il fonografo di Edison e ascoltandola al contrario.\\
	Non si sa se Kelvin non si accorse di essere stato preso in giro dagli studenti o Kelvin stava cercando di prendersi gioco del suo pubblico, ma sta di fatto che \textit{mho} prese inesplicabilmente piede come nome per la conduttanza; anche il simbolo del \textit{mho} fu ottenuto ribaltando la Omega maiuscola dell'\textit{ohm}.
	
	Saltiamo molti anni e nella \textit{Conferenza Generale sui Pesi e le Misure} 1971 si decise che questo scherzo era durato abbastanza: l'unità di misura della conduttanza verrà chiamato \textit{siemens}... aspe', \textit{di nuovo siemens}?\\
	Eh sì, gli scienziati lì riuniti volevano dedicare un unità di misura a Siemens --- non hanno neanche specificato se a Werner von Siemens o al fratello Sir William Siemens --- senza tener conto che in passato è stata usata un'unità di misura dallo stesso nome per tutt'altri scopi. Non si curarono di questo e il \textit{siemens} fu approvato: da allora, usare il \textit{mho} per la conduttanza è considerato un nome non accettabile (e non consono) per un'unità di misura del SI e pertanto deve essere rigorosamente evitato.
	
	% TO DO: conclusione
\end{digressionwt} % yrneh % daraf
Dopo questa breve ma importante digressione passiamo alle unità 
\section{Potenza dissipata da una resistenza}
In un conduttore elettrico in cui scorre una corrente elettrica, la \textbf{potenza} necessaria per spostare una carica è data da
\begin{equation}
	P=\pdv{W}{t}=\vba{F}\vdot \vba{v}_d=e\vba{E}\vdot\vba{v}_d
\end{equation}
Questa energia viene dispersa nell'ambiente sotto forma di \textit{calore}.\\
Se $n$ è il numero di cariche per unità di volume, la \textbf{densità di potenza}, ossia la potenza per unità di volume, è
\begin{equation}
	P_V=ne\vba{E}\vdot\vba{v}_d=\vba{j}\vdot \vba{E}
\end{equation}
con $\vba{j}$ la densità di corrente. La potenza totale dissipata segue facilmente da
\begin{equation}
	P=\int_{V}P_VdV
\end{equation}
Consideriamo il caso particolare di un conduttore ohmico cilindrico di lunghezza $d$ e di superficie $\Sigma$ in situazione di corrente stazionaria. La densità di corrente e il campo elettrico sono costanti e paralleli, dunque densità di potenza è costante e pari a
\begin{equation}
	P_V=jE
\end{equation}
quindi la potenza totale è data da
\begin{equation*}
	P=\int_{V}P_VdV=P_V\mathcal{Vol}=jE\Sigma d=\underbrace{\Sigma j}_{=I}\underbrace{Ed}_{=V}=IV
\end{equation*}
dove $V$ è il calo di potenziale ai capi del conduttore. In sostanza, in presenza di un calo di potenziale si ha una dispersione di energia sotto forma di calore: maggiore è la corrente nel conduttore, maggiore sarà l'energia dispersa.

Cosa succede, a livello microscopico? La $\mathrm{d.d.p.}$ ai capi del conduttore generano un campo elettrico che induce una velocità di deriva nei portatori di carica, fornendo a loro un'energia cinetica. Quando le particelle urtano gli ioni del reticolo cristallino, nell'urto (elastico) viene ceduta energia cinetica dagli elettroni agli ioni, i quali tuttavia sono immobili e quindi \textit{vibrano}, trasformando quella energia sotto forma di calore. Quanta energia viene dispersa dipende dal materiale e dalla geometria del conduttore: questa informazione fisico-empirica è presentata matematicamente nella resistività $\rho$ e, di conseguenza, dalla resistenza.

Questo è quello che viene chiamato \textbf{effetto Joule}\index{effetto Joule}; nella sua forma più basica questo si esprime dalla legge
\begin{equation}
	P=IV
\end{equation}
Se assumiamo che il conduttore trasforma completamente l'energia in calore, allora
\begin{equation}
	P=IV=I^2R=\frac{V^2}{R}
\end{equation}
\begin{example}
	Qualunque elettrodomestico o oggetto che produce calore, in una forma o nell'altra, a partire da corrente elettrica si basa sull'effetto Joule, come lampadine, fon, forni...
\end{example}
\paragraph{Lavoro compiuto dal campo elettrico}
Per ottenere l'energia dispersa in un periodo di tempo $t$, ossia il lavoro compiuto dal campo elettrico nel conduttore per spostare le cariche, ci basta integrare rispetto al tempo la potenza:
\begin{equation}
	U=W=\int_{0}^{t}Pdt=\int_{0}^t IVdt
\end{equation}
Se la corrente è stazionaria, si ha
\begin{equation}% V=IR
	U=W=\frac{I^2V}{2}=\frac{I^3R}{2}=\frac{V^3}{2R^2}
\end{equation}
\section{Circuiti elettrici}
\begin{define}[Circuito elettrico]
	Un \textbf{circuito elettrico} è un insieme interconnesso di componenti elettrici, collegati da fili conduttori in un percorso chiuso in modo che la corrente elettrica possa fluire con continuità.
\end{define}
Di seguito potete vedere un esempio di circuito circolare.
\begin{center}
	\begin{frame}{Circuito elettrico}
		\animategraphics[loop,autoplay,width=\linewidth]{20}{images/scimmia-}{0}{4}
	\end{frame}
\end{center}