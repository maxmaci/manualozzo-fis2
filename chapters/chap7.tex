% SVN info for this file
\svnidlong
{$HeadURL$}
{$LastChangedDate$}
{$LastChangedRevision$}
{$LastChangedBy$}

\chapter{Magnetostatica}
\labelChapter{magnetismo}

\begin{introduction}
	‘‘È una rottura pensare alle convergenze, ma a volte bisogna davvero farlo.''
	\begin{flushright}
		\textsc{Sinai Robins,} ricordandosi delle innumerevoli convergenze di funzioni tre giorni prima dell'esame di Analisi Matematica 3.
	\end{flushright}
\end{introduction}
\lettrine[findent=1pt, nindent=0pt]{C}{on}
\section{Magnetismo}
Inizialmente  gli studi dei fenomeni magnetici erano separati da quelli sull'elettricità, dato che i due fenomeni risultavo completamente scorrelati. Con gli occhi del fisico odierno, sappiamo perché non si fece tale collegamento: le leggi che descrivono i fenomeni \textit{magnetostatici}, i primi osservati, non sono dipendenti da aspetti di natura \textit{elettrica}.

Tuttavia, questo non è più il caso quando consideriamo fenomeni se introduciamo la corrente elettrica.