% SVN info for this file
\svnidlong
{$HeadURL$}
{$LastChangedDate$}
{$LastChangedRevision$}
{$LastChangedBy$}

\chapter{Relatività ristretta}
\labelChapter{relristretta}

\begin{introduction}
	‘‘La relatività ristretta è tutta la Fisica.''
	\begin{flushright}
		\textsc{Albert Einstein,} cercando di vendere i suoi libri di Relatività ristretta agli ignari studenti di Fisica 1.
	\end{flushright}
\end{introduction}

%%LEZ 26, 02/05/2022
%
%%LIBRI:
%%relatività: Enzo Barone, ha molte più cose ma è un bel libro, anche se non sufficiente per tutto
%
%introduzione alla fisica moderna
%relatività ristretta
%passaggio da fisica classica a quantistica
%
%finora siamo arrivati fino al 1860, vediamo un po' il contesto storico
%
%
%onde elettromagnetiche
%
%la fisica sembrava padroneggiare tutti i fenomeni accessibili all'esperienza quotidiana
%momento storico: "ormai sappiamo quasi tutto"
%
%1894 Michelson, sperimentale americano
%	progresso solo nella precisione decimale
%1900	Lord Kelvin, in realtà William Thomson, uno dei tre, si occupa di termodinamica e meccanica statistica, gli altri due sono JJ Thompson che corpre l'elettrone è una particella, George Thompson scopre l'elettrone ed è un'onda, padre e figlio
%	soltanto due nuvole (problemi) sull'orizzonte della fisica teorica:
%		moto della materia attraverso l'etere
%		teorema di equipartizione in meccanica statistica
%		
%poi successe l'esatto contrario, problemi che portano a soluzioni: relatività ristretta e meccanica quantistica (in ordine)		
%
%1900 (dicembre)	Planck: ipotesi dell'energia quantizzata, introduce il quanto
%1905	Einstein: propone la relatività ristretta
%1913	modello atomico di Bohr
%1915	relatività generale
%1925	sviluppo completo della meccanica quantistica ad opera di Heisenberg e Shroedinger
%
%c'è stata un'intera rivoluzione concettuale
%
%spezziamo una lancia per Michelson: scoperte nei decimali profondi in realtà è vera: quello che si è scoperto sono perfezionamenti della classica, che la riproduce dopo un certo limite
%	relatività ristretta: velocità della luce "a infinito", ma in realtà è una quantità dimensionale, dipende dall'unità di misura che metto, in realtà considero fenomeni più piccoli della velocità della luce, in particolare prendo $\frac{v}{c}\ll 1$
%	quantistica: dopo costante di Planck	mando a 0 $h$, in realtà prendo quantità molto più grandi di $h$, cioè preso $S$ con le dimensioni di un'azione e $\frac{S}{h}\gg 1$
%
%
%RELATIVITà
%Prima di Einstein c'era già la relatività, la \textit{relatività Gelileiana}
%cambiamento di atteggiamento mentale: bisogna pensare alle proprietà di trasformazione, a volte ci sono dei principi id invarianza
%
%A) sistema di riferimento
%	 sistema di riferimento, cioè un sistema di coordinate: sono le coordinate dello spazio ed una del tempo
%		sono del tipo $(x,y,z,t)=(\vba r, t)$
%	abbiamo già fatto delle scelte: 
%		dov'è l'origine degli assi
%		l'orientazione
%		l'origine dei tempi, cioè quando inizio a contare il tempo
%% TODO: IMMAGINE
%	ma le leggi della fisica non dovrebbero dipendere da queste scelte!
%	sono invarianze che ci aspettiamo, ma dobbiamo capire in che senso sono delle invarianze
%	
%	Guardiamo le equazioni di Newton, la seconda legge, per un singolo punto materiale
%		m\dv[2]{\vba r}{t}=\vba F \implies \vba r(t)
%	cioè posso determinare $\vba r(t)$ questo perchè è equazione differenziale del secondo ordine, basta avere condizione iniziale
%	abbiamo così 3 equazioni
%		la scelta dell'origine degli assi non fa variare le equazioni
%			\vba r\rightarrow \vba r'=\vba r+\vba r_0
%	dobbiamo però fare attenzione: spesso ragioniamo in termini di potenziali, lì dobbiamo fare la trasformazione anche sul campo di forze, come ad esempio spostare la molla per un oscillatore armonico
%	in generale la forza ci viene già data in un particolare sistema di coordinate
%	quindi è meglio fare un esempio concreto e realistico di quello che succede con le leggi fisiche: sistemi che interagiscono, punti materiali che gravitano o cariche elettriche
%	se ho tanti punti ho tante equazioni
%		m\dv[2]{\vba r_i}{t}=\vba F, i=1,\dots ,n
%	tipicamente succede che la forza che agisce sulla particella i dipende da dove sono tutte le altre, cioè da tutte le possibili differenze
%		m\dv[2]{\vba r_i}{t}=\vba F_i(\vba r_j -\vba r_k), i=1,\dots ,n
%	esempio:
%		gravità \vba F_i=\sum_{i\neq j} Cm_im_j\frac{\vba r_i-\vba r_j}{\abs{\vba r_i-\vba r_j}^3}
%	questo avremmo potuto usarlo come principio per dedurlo, cioè la fisica non dipende da dove siamo, quindi non è possibile che ci sia il segno + o qualsiasi altra combinazione possibile tranne il - avremmo dei problemi in questo senso
%	i principi di simmetria limitano la forma delle leggi fisiche
%	\begin{enumerate}
%		\item	quelle equazioni sono invarianti in forma quando facciamo
%		\vba r_i\rightarrow \vba r_i'=\vba r_i+\vba r_0
%		che è a 3 parametri
%		\item possodefinire l'origine dei tempi
%			t\rightarrow t'=t+t_0
%		che è ad 1 parametro
%		\item orientazione
%			\vba r\rightarrow \vba r'=R \vba r, R\in O(3)
%%TODO: guardare comando per gruppo ortogonale
%		cioè $R$ matrice ortogonale che ha 3 parametri 
%		l'equazione non è invariata ma sappiamo come si trasforma, perchè stiamo lavorando come i vettori, che possiamo ruotare e sappiamo come fanno
%	\end{enumerate}
%	Otteniamo così 
%		\vba r'=R\vba r		\vba F'=R\vba F
%		m\dv[2]{\vba r_i}{t}=\vba F'_i(\vba r_i'-\vba r_j')
%	oltre che invarianz viene detta \textit{covarianza}
%	non rimangono le stessa ma si trasformano allo stesso modo
%	è un modo intelligente di ridefinire un vettore dopo le frecce e coordinate, cioè sotto azione di rotazioni
%	vettore è roba che sta nella rappresentazione fondamentale di $O(3)$
%	quindi le equazioni di Newton sono invarianti sotto le trasformazioni di cui sopra, che formano un gruppo non abeliano con 7 parametri con pezzi abeliani (traslazioni)
%	poniamoci il problema di vedere se abbiamo contato tutto
%	radice fisica da Galileo, c'è un'ulteriore invarianza
%		\vba r_i \rightarrow\vba r'_i=\vba r_i +\vba vt
%	che è a 3 parametri
%	un esempio è quello dei due treni che viaggiano parallelamente e non si sa chi parte
%	le leggi della dinamica non vedono il moto rettilineo uniforme!
%	ed è la \textit{relatività galileiana}
%	è vero per lo stesso motivo di prima
%		\vba r_i-\vba r_j=\vba r'_i-\vba r'_j
%		\dv[2]{\vba r_i}{t}=\dv[2]{\vba r'_i}{t}
%	questo funzione perchè la forza è proporzionale all'accelerazione, cioè la derivata seconda e non la prima come per Arisostele
%	
%	Abbiamo così 10 parametri, che costituiscono il \textit{gruppo di Galileo}
%	
%	
%	
%B) Sistemi di riferimento inerziali
%	domanda: in quale set di sistemi di riferimento sono valide le leggi di Newton?
%	sono interne o esterne?
%	esempio: giostra: forza centrifuga: non c'è effettivamente nessuno che sta tirando, ma la percepiamo perché siamo in un sistema di riferimento in cui c'è accelerazione
%	\begin{define}[Sistema di riferimento inerziale, SRI]
%		Sistema in cui corpi non soggetti a forza si muovono di moto rettilineo uniforme
%			m\dv[2]{\vba r_i}{t}=\vba 0 \implies \vba r_i(t)=\vba r_{i,0}+\vba u_i t
%	\end{define}
%	in realtà neanche il sistema Terra è inerziale: ade esempio possiamo misurare la forza di Coriolis
%	anche il Sole è in movimento rispetto alla Galassia, che a sua volta è in rotazione
%	sistema inerziale è limite asintotico di qualcosa che conosciamo
%	Principio di relatività Galileiana
%		si può scrivere in due modi:
%			\begin{itemize}
%				\item le leggi fisiche sono le stesse in tutti i SRI, che si può utilizzare come principio
%				\item le leggi fisiche sono invarianti in forma (cioè invarianti, covarianti o controvarianti) rispetto alle trasformazioni del gruppo di Galileo
%%CIT: dopo le leggi di Newton potete costruire macchine e fare la rivoluzione industriale				
%			\end{itemize}
%\begin{digression}[Pasticcio filosofico]
%	si stavano rompendo molte cose, non ci sono più dogmi. La teoria della relatività è l'esatto contrario: so come funziona la fisica da un'altra parte anche se sono da un'altra parte, dice come si trasformano le cose
%	sarebbe più una teoria dell'assolutezza che della relatività
%\end{digression}
%%CIT: erano entrambi anziani e li perdoniamo
%%CIT: dovete conoscere le equazioni di Maxwell come una volta si conoscevano le preghiere: a memoria e con profondità
%	Questo però non funzione con le leggi di Maaxwell, che ricordiamo come
%		\div\vba E=4\pi\rho	[Gauss]
%		\div\vba B=0	[no monopoli]
%		\curl\vba E+\frac{1}{c}\pdv{\vba B}{t}=0	[Faraday]
%		\curl\vba B-\frac{1}{c}\pdv{\vba E}{t}=\frac{4\pi}{c}\vba j	[Ampèere-Maxwell]
%	la relatività galileiana non si applica a queste equazioni che contengono una velocità \textit{fissa} $c$
%	possiamo provare a fare le trasformazioni galileiana, vengono equazioni complicate che dipendono da $\vba v$
%	problema messo sotto al tappeto per 30 anni perchè si cercava di risolverlo per onde meccaniche, le uniche che si riuscirono a concepire
%%CIT: Hollywood se n'è accorta 10 anni fa che non c'è rumore nello spazio, finroa era tutto BOOM
%%CIT: solo le onde elm lo percepiscono, e qui già capiamo che siamo nella fisica fantasmatica
%	si cercava di risolvere tutto con l'etere, così leggero da essere percepito solo dalle onde elettromagnetiche, dunque le equazioni di Maxwell varrebbero solo per il sistema in cui l'etere è a riposo, che sarebbe il sistema di riferimento privilegiato
%	possiamo però anche riscrivere la meccanica buttando la relatività galileiana buttando 'etere
%	è l'esperimento a decidere, in questo caso quello di Michelson-Morley
%
%	Iniziamo il conto
%		\curl\vba E+\frac{1}{c}\pdv{\vba B}{t}=0
%		\curl\curl\vba E= \grad\left(\div \vba E \right) -\laplacian\vba E= -\frac{1}{c}\pdv{t} \curl\vba B
%	usiamo le altre equazioni di Maxwell e scegliamo di metterci nel vuoto, dove $\rho\vba j=0$
%		\curl\curl\vba E= \underbrace{\grad\left(\div \vba E \right)}_{=0} -\laplacian\vba E= -\frac{1}{c}\pdv{t} \curl\vba B=\frac{1}{c}\pdv{\vba E}{t}
%	otteniamo così l'equazione delle onde che si propagano alla velocità $c$
%		\frac{1}{c^2}\pdv[2]{\vba E}{t}-\laplacian\vba E=0
%	