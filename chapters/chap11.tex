% SVN info for this file
\svnidlong
{$HeadURL$}
{$LastChangedDate$}
{$LastChangedRevision$}
{$LastChangedBy$}

\chapter{Relatività ristretta}
\labelChapter{relristretta}

\begin{introduction}
	‘‘La relatività ristretta è tutta la Fisica.''
	\begin{flushright}
		\textsc{Albert Einstein,} cercando di vendere i suoi libri di Relatività ristretta agli ignari studenti di Fisica 1.
	\end{flushright}
\end{introduction}

%%LEZ 26, 02/05/2022
%
%%LIBRI:
%%relatività: Enzo Barone, ha molte più cose ma è un bel libro, anche se non sufficiente per tutto
%
%introduzione alla fisica moderna
%relatività ristretta
%passaggio da fisica classica a quantistica
%
%finora siamo arrivati fino al 1860, vediamo un po' il contesto storico
%
%
%onde elettromagnetiche
%
%la fisica sembrava padroneggiare tutti i fenomeni accessibili all'esperienza quotidiana
%momento storico: "ormai sappiamo quasi tutto"
%
%1894 Michelson, sperimentale americano
%	progresso solo nella precisione decimale
%1900	Lord Kelvin, in realtà William Thomson, uno dei tre, si occupa di termodinamica e meccanica statistica, gli altri due sono JJ Thompson che corpre l'elettrone è una particella, George Thompson scopre l'elettrone ed è un'onda, padre e figlio
%	soltanto due nuvole (problemi) sull'orizzonte della fisica teorica:
%		moto della materia attraverso l'etere
%		teorema di equipartizione in meccanica statistica
%		
%poi successe l'esatto contrario, problemi che portano a soluzioni: relatività ristretta e meccanica quantistica (in ordine)		
%
%1900 (dicembre)	Planck: ipotesi dell'energia quantizzata, introduce il quanto
%1905	Einstein: propone la relatività ristretta
%1913	modello atomico di Bohr
%1915	relatività generale
%1925	sviluppo completo della meccanica quantistica ad opera di Heisenberg e Shroedinger
%
%c'è stata un'intera rivoluzione concettuale
%
%spezziamo una lancia per Michelson: scoperte nei decimali profondi in realtà è vera: quello che si è scoperto sono perfezionamenti della classica, che la riproduce dopo un certo limite
%	relatività ristretta: velocità della luce "a infinito", ma in realtà è una quantità dimensionale, dipende dall'unità di misura che metto, in realtà considero fenomeni più piccoli della velocità della luce, in particolare prendo $\frac{v}{c}\ll 1$
%	quantistica: dopo costante di Planck	mando a 0 $h$, in realtà prendo quantità molto più grandi di $h$, cioè preso $S$ con le dimensioni di un'azione e $\frac{S}{h}\gg 1$
%
%
%RELATIVITà
%Prima di Einstein c'era già la relatività, la \textit{relatività Gelileiana}
%cambiamento di atteggiamento mentale: bisogna pensare alle proprietà di trasformazione, a volte ci sono dei principi id invarianza
%
%A) sistema di riferimento
%	 sistema di riferimento, cioè un sistema di coordinate: sono le coordinate dello spazio ed una del tempo
%		sono del tipo $(x,y,z,t)=(\vba r, t)$
%	abbiamo già fatto delle scelte: 
%		dov'è l'origine degli assi
%		l'orientazione
%		l'origine dei tempi, cioè quando inizio a contare il tempo
%% TODO: IMMAGINE
%	ma le leggi della fisica non dovrebbero dipendere da queste scelte!
%	sono invarianze che ci aspettiamo, ma dobbiamo capire in che senso sono delle invarianze
%	
%	Guardiamo le equazioni di Newton, la seconda legge, per un singolo punto materiale
%		m\dv[2]{\vba r}{t}=\vba F \implies \vba r(t)
%	cioè posso determinare $\vba r(t)$ questo perchè è equazione differenziale del secondo ordine, basta avere condizione iniziale
%	abbiamo così 3 equazioni
%		la scelta dell'origine degli assi non fa variare le equazioni
%			\vba r\rightarrow \vba r'=\vba r+\vba r_0
%	dobbiamo però fare attenzione: spesso ragioniamo in termini di potenziali, lì dobbiamo fare la trasformazione anche sul campo di forze, come ad esempio spostare la molla per un oscillatore armonico
%	in generale la forza ci viene già data in un particolare sistema di coordinate
%	quindi è meglio fare un esempio concreto e realistico di quello che succede con le leggi fisiche: sistemi che interagiscono, punti materiali che gravitano o cariche elettriche
%	se ho tanti punti ho tante equazioni
%		m\dv[2]{\vba r_i}{t}=\vba F, i=1,\dots ,n
%	tipicamente succede che la forza che agisce sulla particella i dipende da dove sono tutte le altre, cioè da tutte le possibili differenze
%		m\dv[2]{\vba r_i}{t}=\vba F_i(\vba r_j -\vba r_k), i=1,\dots ,n
%	esempio:
%		gravità \vba F_i=\sum_{i\neq j} Cm_im_j\frac{\vba r_i-\vba r_j}{\abs{\vba r_i-\vba r_j}^3}
%	questo avremmo potuto usarlo come principio per dedurlo, cioè la fisica non dipende da dove siamo, quindi non è possibile che ci sia il segno + o qualsiasi altra combinazione possibile tranne il - avremmo dei problemi in questo senso
%	i principi di simmetria limitano la forma delle leggi fisiche
%	\begin{enumerate}
%		\item	quelle equazioni sono invarianti in forma quando facciamo
%		\vba r_i\rightarrow \vba r_i'=\vba r_i+\vba r_0
%		che è a 3 parametri
%		\item possodefinire l'origine dei tempi
%			t\rightarrow t'=t+t_0
%		che è ad 1 parametro
%		\item orientazione
%			\vba r\rightarrow \vba r'=R \vba r, R\in O(3)
%%TODO: guardare comando per gruppo ortogonale
%		cioè $R$ matrice ortogonale che ha 3 parametri 
%		l'equazione non è invariata ma sappiamo come si trasforma, perchè stiamo lavorando come i vettori, che possiamo ruotare e sappiamo come fanno
%	\end{enumerate}
%	Otteniamo così 
%		\vba r'=R\vba r		\vba F'=R\vba F
%		m\dv[2]{\vba r_i}{t}=\vba F'_i(\vba r_i'-\vba r_j')
%	oltre che invarianz viene detta \textit{covarianza}
%	non rimangono le stessa ma si trasformano allo stesso modo
%	è un modo intelligente di ridefinire un vettore dopo le frecce e coordinate, cioè sotto azione di rotazioni
%	vettore è roba che sta nella rappresentazione fondamentale di $O(3)$
%	quindi le equazioni di Newton sono invarianti sotto le trasformazioni di cui sopra, che formano un gruppo non abeliano con 7 parametri con pezzi abeliani (traslazioni)
%	poniamoci il problema di vedere se abbiamo contato tutto
%	radice fisica da Galileo, c'è un'ulteriore invarianza
%		\vba r_i \rightarrow\vba r'_i=\vba r_i +\vba vt
%	che è a 3 parametri
%	un esempio è quello dei due treni che viaggiano parallelamente e non si sa chi parte
%	le leggi della dinamica non vedono il moto rettilineo uniforme!
%	ed è la \textit{relatività galileiana}
%	è vero per lo stesso motivo di prima
%		\vba r_i-\vba r_j=\vba r'_i-\vba r'_j
%		\dv[2]{\vba r_i}{t}=\dv[2]{\vba r'_i}{t}
%	questo funzione perchè la forza è proporzionale all'accelerazione, cioè la derivata seconda e non la prima come per Arisostele
%	
%	Abbiamo così 10 parametri, che costituiscono il \textit{gruppo di Galileo}
%	
%	
%	
%B) Sistemi di riferimento inerziali
%	domanda: in quale set di sistemi di riferimento sono valide le leggi di Newton?
%	sono interne o esterne?
%	esempio: giostra: forza centrifuga: non c'è effettivamente nessuno che sta tirando, ma la percepiamo perché siamo in un sistema di riferimento in cui c'è accelerazione
%	\begin{define}[Sistema di riferimento inerziale, SRI]
%		Sistema in cui corpi non soggetti a forza si muovono di moto rettilineo uniforme
%			m\dv[2]{\vba r_i}{t}=\vba 0 \implies \vba r_i(t)=\vba r_{i,0}+\vba u_i t
%	\end{define}
%	in realtà neanche il sistema Terra è inerziale: ade esempio possiamo misurare la forza di Coriolis
%	anche il Sole è in movimento rispetto alla Galassia, che a sua volta è in rotazione
%	sistema inerziale è limite asintotico di qualcosa che conosciamo
%	Principio di relatività Galileiana
%		si può scrivere in due modi:
%			\begin{itemize}
%				\item le leggi fisiche sono le stesse in tutti i SRI, che si può utilizzare come principio
%				\item le leggi fisiche sono invarianti in forma (cioè invarianti, covarianti o controvarianti) rispetto alle trasformazioni del gruppo di Galileo
%%CIT: dopo le leggi di Newton potete costruire macchine e fare la rivoluzione industriale				
%			\end{itemize}
%\begin{digression}[Pasticcio filosofico]
%	si stavano rompendo molte cose, non ci sono più dogmi. La teoria della relatività è l'esatto contrario: so come funziona la fisica da un'altra parte anche se sono da un'altra parte, dice come si trasformano le cose
%	sarebbe più una teoria dell'assolutezza che della relatività
%\end{digression}
%%CIT: erano entrambi anziani e li perdoniamo
%%CIT: dovete conoscere le equazioni di Maxwell come una volta si conoscevano le preghiere: a memoria e con profondità
%	Questo però non funzione con le leggi di Maaxwell, che ricordiamo come
%		\div\vba E=4\pi\rho	[Gauss]
%		\div\vba B=0	[no monopoli]
%		\curl\vba E+\frac{1}{c}\pdv{\vba B}{t}=0	[Faraday]
%		\curl\vba B-\frac{1}{c}\pdv{\vba E}{t}=\frac{4\pi}{c}\vba j	[Ampèere-Maxwell]
%	la relatività galileiana non si applica a queste equazioni che contengono una velocità \textit{fissa} $c$
%	possiamo provare a fare le trasformazioni galileiana, vengono equazioni complicate che dipendono da $\vba v$
%	problema messo sotto al tappeto per 30 anni perchè si cercava di risolverlo per onde meccaniche, le uniche che si riuscirono a concepire
%%CIT: Hollywood se n'è accorta 10 anni fa che non c'è rumore nello spazio, finroa era tutto BOOM
%%CIT: solo le onde elm lo percepiscono, e qui già capiamo che siamo nella fisica fantasmatica
%	si cercava di risolvere tutto con l'etere, così leggero da essere percepito solo dalle onde elettromagnetiche, dunque le equazioni di Maxwell varrebbero solo per il sistema in cui l'etere è a riposo, che sarebbe il sistema di riferimento privilegiato
%	possiamo però anche riscrivere la meccanica buttando la relatività galileiana buttando 'etere
%	è l'esperimento a decidere, in questo caso quello di Michelson-Morley
%
%	Iniziamo il conto
%		\curl\vba E+\frac{1}{c}\pdv{\vba B}{t}=0
%		\curl\curl\vba E= \grad\left(\div \vba E \right) -\laplacian\vba E= -\frac{1}{c}\pdv{t} \curl\vba B
%	usiamo le altre equazioni di Maxwell e scegliamo di metterci nel vuoto, dove $\rho\vba j=0$
%		\curl\curl\vba E= \underbrace{\grad\left(\div \vba E \right)}_{=0} -\laplacian\vba E= -\frac{1}{c}\pdv{t} \curl\vba B=\frac{1}{c}\pdv{\vba E}{t}
%	otteniamo così l'equazione delle onde che si propagano alla velocità $c$
%		\frac{1}{c^2}\pdv[2]{\vba E}{t}-\laplacian\vba E=0
%	

%%%%%%%%%%%%%%%%%%%%%%%%%%%%%%%%%%%%%%%%%%%%%%%%%%%%%%%%%%%%%%%%%%%%%%%%%%%%%%%%%%%%%%%%%%
%LEZ 27, 04/05/2022
%CIT voi siete tutti criminali, perché zitti zitti non avete detto niente
abbiamo constatato che la meccanica e  l'elettrodinamica non vanno d'accordo
la meccanica infatti è covariante per gruppo a 10 parametri: traslazioni spazio temporali e trasformazione galileiane con velocità costante
motivo evidente: velocità nelle equazioni esplicita che non c'è nelle equazioni di Newton
forze in completa generalità come differenze di posizioni
con manipolazioni semplici abbimao trovato le equazioni delle onde
	\frac{1}{c^2}\pdv[2]{\vba E}{t}\laplacian \vba E=0
propagazione onde elettromagnetiche nel vuoto alla velocità della luce
ma in quale sistema di rifereimento si propagano alla velocità della luce?
se fossero vere le galieiane la luce si propagherebbe avelocità c+v o c-v
cosa succede a queste eqauzioni se facciamo una trasformazione gelileiana?
	x\rightarrowx'=x-vt 
	y\rightarrowy'=y
	z\rightarrow '=z
	t\rightarrowt'=t	sempre, incarnazione delle assunzioni filosofiche su spazio assoluto: tutti osservatori misurano lo stesso tempo
	vedremo poi che non è vero
alle coordinate sappiamo esattamente cosa fare, sono derivate
	\pdv{x'}=(catena in 4 variabili)\pdv{x}{x'}\pdv{x}+\pdv{y}{x'}\pdv{y}+\pdv{z}{x'}\pdv{z}+\pdv{t}{x'}\pdv{t}=\pdv{x}
	\pdv{t'}=\pdv{x}{t'}\pdv{x'}+\dots+\pdv{t}{t'}\pdv{t}=\pdv{t}+v\pdv{x}
	
abbiamo così che
	x=x'+vt'
abbiamo fatto una particolare trasformazione con una velocità nella direzione x
tutte le eq sono fatte di vettori, se velocità fosse in direzione qualsiasi cosa dovrei fare?
devo fare l'unica cosa vettoriale che si riduce in quella direzione
	\pdv{t}+\vba v\vdot\grad (prodotto scalare)
possiamo scrivere subito in quel sistema di riferimento con v_x\vdot\pdv{x}
diventa
	\pdv[2]{t}\rightarrow \left( \pdv{t'}-v\vdot\grad '\right) \left(\pdv{t'}-\vba v\vdot \grad '\right)=
ricordiamo che stiamo agendo su un elemento
	=\pdv[2]{t'}-\pdv{t'}\right[\vba v\vdot\grad '\left]-\vba v\vdot\grad '\pdv{t'}+(v\vdot\grad)^2

	\frac{1}{c^2}\pdv[2]{\vba E}{t}\laplacian \vba E=0 \rightarrow \frac{1}c^2 \left(\pdv[2]{\vba E}{t'} -2v\vdot\grad '\pdv{E}{t'} +(\vba v\vdot\grad ')^2 \vba E \right) -\laplacian\vba E=0
l'equazione non è invariante in forma, tutto dipende dalla velocità del sistema di riferimento su cui mi sto spostando
c'è una sottigliezza: abbimao fatto un'ipotesi: ho trasformato coordinate ma non il campo elettrico, cioè $\vba E=\vba E'$, ma se avessi fatto una rotazione avrei dovuto ruotare il vettore $\vba E$
sembra un'ipotesi sensata perché è un vettore nella rappresentazione fondamentale di SO3 e galileianamente mi aspetto sia così perchè non vede il tempo
la soluzione che troveremo sarà un cambiamento dei campi elettrici che si mescoleranno con i cmapi magnetici
galileianamente mi aspetto che si comportino come 3 funzioni scalari 
	f(x,y,z)=f'(x',y',z')
l'equazione che comanda il cambio di coordinate per funzione scalare fornisce lo stesso numero
ad esempio in una stanza in ogni punto c'è un numero scalare che vale qualche cosa
le coordinate possono essere diverse ma nello stesso punto fisico c'è associato lo stesso numero scalare
non è vero però per un cmapo vettoriale, avrei una matrice di rotazione davanti, una legge di trasformazione dentro ed una fuori


perché sono stati 40 anni sereni su questo fatto?
tutte le onde che erano state viste erano propagate in un mezzo, la velocità nell'onda nel mezzo statico
qunidi l'universo deve essere pieno di un mezzo in cui la luce si propaga
sistema dell'etere in cui è fermo è sistema dell'universo e poi si trasforma in altri sistemi
c'è l'etere che deve essere evanescente e di essere la roba che vibra per far propagare le onde elettromagnetiche
analogo al contemporaneo materia oscura ed energia oscura

vediamo ora un paio di esperimenti fatti prima del 1905, con esiti che suggerivano che il quadro di sistema di riferimento dell'etere era problematico
%NB: ho meesso un environment a caso per gli esperimenti, non so se ha senso implementarlo tho
\begin{experiment}[Michelson Morley]
	Scopo: rilevare il moto della Terra attraverso l'etere
	Se l'etere è fermo, qualsiasi orientazione dell'eclittica la velocità della luce sarà diversa, ma di quanto?
%TODO: IMMAGINE
	avremmo un effetto Doppler sulle frequenze delle onde della luce, qunidi vedremmo più rosso o blu
	il raggio della Terra è 150*10^6km, il numero di secondi in un anno è vicino a \pi*10^7 vogliamo la velocità della Terra
		\abs{\vba v}\sim \frac{150*10^6 km 2\pi}{\pi 10^7}\sim 30\frac{km}{sec}
		\frac{v}{c}=\beta\sim 10^{-4}
	per misurare con questa precisione serve un \textit{interferometro}
	esso è costituito da specchi e una sorgente luminosa (lampadina ad incandescenza) ed un punto di osservazione rappresentata da cannocchiale (in realtà la lampadina dovrà essere monocromatico)
	il primo specchio è split, è semitrasparente, una parte del fascio va contro S_1, sbatte e torna indietro, 'altra metà è riflessa dallo specchio S_2 e torna al cannocchiale, quindi ho ricomposto il fascio di luce
	in generale i bracci L_1 ed L_2 hanno sempre lunghezze diverse per il livello di precisione richiesto
	inrealtà l'esperimento viene fatto in tutti i i modi possibili e poi viene fatta una media per avere i risultati
	immaginiamo di essere nella situazione ottimale per cui la Terra si sta muovendo attraverso l'etere e dopo 6 mesi torniamo con velocità -\vba v
	abbiamo che LC e CO sono in comune, quindi non c'è differenza di cammino ottico che li riguardi
	vediamo cosa succede in CS_2 nel sistema di laboratorio, quindi sulla Terra
		siamo in movimento rispetto all'etere, la velocità che si vede è c-v, qunidi quanto tempo ci impiega a fare avanti e indietro?
			CS_2C\colon t_2= L_2\left( \frac{1*{c-v}+\frac{1}{c+v} \right)= \frac{2cL_2}{c^2-v^2}= \frac{2L_2}{c}\frac{1}{1-\beta^2} \text{ con } \beta=\frac{v}{c}
	vediamo cosa succede quando la luce va contro lo specchio S_1, viene meglio fatto nel sistema dell'etere, posso farlo perchè $t=t'$
%TOD: IMMAGINE
	quando arriva su S_1 è passato un po' di tempo, per l'etere il raggio luminoso è diagonale, è metà del tempo che stiamo cercando di misurare
	L si trova con il potente strumwnt matematico che è il teorema di Pitagora, L distanza percorsa dalla luce in t_1/2, nell'etere la luce va a velocità c
		L=\sqrt{L_1^2+\frac{v^2t_1^2}{4}}= 	c\frac{t_1}{2}
	considerando
		CS_1C\colon \frac{t_1^2}{4}(c^2-v^2)=L_1^2
		\implies t_1=\frac{2L_1}{c}\frac{1}{\sqrt{1-\beta^2}}
	i due tempi sono quindi diversi
	prendiamo il cao semplice L_1=L_2
		\frac{t_1}{t_2}=\sqrt{1-\beta^2} <1
	le dimensioni dell'effetto si misurano con lo sviluppo di Taylor
		\sqrt{1-\beta^2}\sim 1-\frac{\beta^2}{2}+O(\beta^4)
	cioè \beta^2/2 con O(10^{-8})
	lo facciamo in tutti i modi possibili: cambio laboratorio, posizione spaziale, estate, inverno
	ma quanto predetto non succede!
	abbiamo in realtà t_1\sim t_2
	qunidi non si riesce a vedere l'effetto dell'etere, o meglio del nostro movimento rispetto ad esso
	si sono così inventati delle spiegazioni: trascinamento parziale dell'etere come le mosche du una nave, ma sorgono problemi: attrito, definizione eterea contraddittoria
	era quello a cui si rifereiva lord Kelvin: il moto attraverso l'etere
		
\end{experiment}

\begin{experiment}[Aberrazione della luce stellare]
%TODO: IMMAGINE	
	la stella manda una luce con una certa velocità, stelle fisse da DAnte
	se ci muoviamo con una velocità, nel nostro sistema di riferimento non la vediamo così
	nel punto $A$ vediamo noi come osservatori, sistema in cu iA è fermo, vediamo la stella muoversi con velocità -v, dobbiamo comporre le vlocità vettorialmente, quindi deve esserci c'
	in $B$ succede l'opposto, vedo la stella che va avanti (effetto ciclista)
	mentre la Terra gira nell'orbiat dell'eclittica, tutte le stelle descrivono delle ellissi nel cielo tolti tutti gli altri movimenti
	effetto detto aberrazione della luce stellare
	il problema è che non lo fa come predetto dal modello
	
	presa una sorgente di luce qualisasi, una sistema di riferimento $\mathcal{S}$, il raggio di luce arriva e forma un angolo $\theta$ con l'asse $x$
	se simo invece in un sistema di riferimento $\mathcal{S'}$ in movimento di velocità v, la stella $S$ non la vediamo più nello stesso posto: in A e B vediamo la stella spostata
	stiamo immaginando luce che arriv anell'orginie, sistemi che coincidono in quel'istante lì
	quindi in $\mathcal{S}$
		c_x=-c\cos\theta
		c_y=-c\sin\theta
	invece in \mathcal{S}' dobbiamo comporre le velocità
		c'_x=-c\cos\theta-v		abbiamo usato regola di composizione delle velocità galieliane= -c\cos\theta'
		c'_y=-c\sin\theta=-c\sin\theta'
	la relazione fra \theta' e \theta dipende dalle trasformazioni galileiane in teoria
		\tan\theta'=\frac{\sin\theta'}{\cos\theta'}=\frac{c_y'}{c_x'}=\frac{\sin\theta}{\cos\theta +\beta}
	questo ci dice che $\tan\theta'<\tan\theta$
	
	è un fenomeno noto da secoli dal cannocchiale, il risultato però è sbagliato!
	in realtà ci sono correzioni dell'ordine di O(\beta^2)
	sviluppando in serie di Taylor è lineare nella velocità della Terra
	
	
\end{experiment}


\begin{experiment}[Velocità della luce in un mezzo in moto di Fizeau, 1850]
	abbiamo n tubo d'acqua con sorgente e rilevatore messo su un treno, ci aspttiamo che la velocità del treno venga sottratta
	la vleocità nel mezzo è la vloeictà della luce nel vuto sull'indice di rifrazione del mezzo n
		v_n=\frac{c}{n}
	nel caso galieliano (moto del sistema con velocità $u$) ci aspettiamo che 
		v'=v_n+u=\frac{c}{n}+u
	in realtà però non succede, i dati seguono la tendenza
		v'_n=\frac{c}{n}+u\left(1-\frac{1}{n^2}\right)
	questo fattore è detto $k$, \textit{coefficiente di trascinamento dell'etere}
	il problema era cercare di capire l'interazione dei corpi materiali con l'etere, che era solo teorico
	
\end{experiment}

l'etere era solo una piccola modificazione necessaria in un sistema in cui funzionava tutto

si risolve tutto ponendo la velocità della luce uguale in tutti i sistemi di riferimento
\begin{experiment}[Decadimento di pioni in volo, circa 1960]
	succede tutto in un acceleratore id particelle, i pioni appartengono ad una categoria di particelle dette \textit{mesoni}, che contengono 2 quark a carica up o down
		\pi^+ \sim u\vba d
		\pi^-\sim\vba ud
		\pi^0\simu\vba u+d\vba d
	\pi^0 decade in una coppia di fotoni, cioè
		\pi^0\rightarrow\gamma+\gamma
	il pione decade in un tempo di 10^{-16} secondi. Facendoli andare più veloce per noi vivono più tempo
	se è fermo l'impulso è conservato, quindi i fotoni che vanno alla velocità della luce succede
%TODO: IMMAGINE
	però noi stiamo lavorando in un acceleratore di particelle, quindi c'è il sistema del laboratorio per cui \pi^0 va molto veloce con velocità v_\pi
	quando decade abbiamo
%TODO: IMMAGINE
	con gelileiana ci aspettiamo che i fotoni abbiano velocità c+\v_\pi e c-v_\pi
	con molta accuratezza però si è trovato che anche quando v_\pi\sim c il fotone che è stato sparato va sempre a velocità c, non importa cosa sta facendo il pione	
	quindi non è vero quando predetto da Galileo, cioè che il vettore sparato a velocità abbia velocità 2c e no sempre c
	
\end{experiment}

andremo a vedere come questo distrugga la nostra intuizione
quindi il giocattolo è rotto, che si fa?

si riguardano i postulati della relatività galileiana vanno riguardati
il problema di chiamarli postulati è che sono suggeriti dai dati sperimentali per usarli come fondamento logico

nella relatività galileiana c'erano due affermazioni mescolate
RG
	\item	 le leggi fisiche assumono la stessa forma, cioè sono invarianti o covarianti, in tutti i sistemi di riferimento inerziali SRI
	\item	 le regole di trasformazione tra SRI sono le trasformazioni di Galileo

Con Maxwell non funziona, cosa si rompe?
abbiamo due possibilità
	\item	a è falso, l'etere falsifica a, oppure è ristretto alla meccanica (è un'arringa rossa, una falsa via d'uscita perché non sono due cose separate)
	quindi esiste un SRI privilegiato, quello dell'etere in cui la velocità della luce è c
	\item	a è vero anche per l'elettrodinamica, quindi b è falso
	cioè non esiste il moto assoluto: non posso vedere se sono fermo o in moto rettilineo uniforme: come su una nave o astronave facendo esperimenti senza accelerazioni, quindi il passo logico è dire che la differenza SRI non ha senso perché sono tutti equivalenti
	
l'evidenza sperimentale favorisce la soluzione 2

Allora la teoria della relatività ristretta indicata con RR accetta che i sistemi inerziali siano tutti equivalenti
	\item	 le leggi fisiche assumono la stessa forma in tutti i SRI
	\item	la velocità della luce è una costante universale che ha lo stesso valore in tutti i SRI ( da esperimento sui pioni)
una possibile conseguenza è che ci aspettiamo che le equazioni di Maxwell sono corrette! Sono giuste così come sono scritte, non è strano che compaia $c$ perché non dipende dal sistema di riferimento
allora però la velocità non è a posto, perché il gruppo di trasformazioni è quello sbagliato
quindi la meccanica newtoniana deve essere modificata sia in termini pratici
deve comunque restare una buona approssimazione per $\beta\rightarrow 0$

ci tocca stabilire quali sono le trasformazioni, trovarne le regole e riscrivere la meccanica
come bonus dovremo riscrivere le equazioni di Maxwell, riscritte in modo bellissimo




\subsection{Relatività della simultaneità}
cambiando sistema di riferimento cambia anche il tempo
dati due eventi tali che $t_1=t_2$ nel sistema di riferimento $\mathcal{S}$ si ha $t'_1\neq t'_2$ in $\mathcal{S}'$ e vale anche il viceversa
%TODO: IMMAGINE
esattamente in mezzo ad un autobus mettiamo una lampadina e due specchi S_1 ed S_2
per l'osservatore sull'autobus i due eventi sono simultanei perché i fotoni arrivano insieme
questo quando tl'autobus è fermo
se invece lo facciamo partire con una velocità $v$ ed osserviamo dalla stazione
la luce viene accesa esattamente davanti a noi e anche per me che sono in stazione la luce continua ad andare a velocità $c$
	t_1'<t'_2	lo specchio sta andando addosso alla luce
	
in un sistema galileiano funzione perché vediamo fotoni a velocità diverse











