% SVN info for this file
\svnidlong
{$HeadURL$}
{$LastChangedDate$}
{$LastChangedRevision$}
{$LastChangedBy$}

\chapter{Il flusso del campo elettrico e la legge di Gauss}
\labelChapter{convergenzafunzioni}

\begin{introduction}
	‘‘BEEP BOOP''
	\begin{flushright}
		\textsc{Lollo BiancoBOT}, dopo aver finito le citazioni. % TO DO: quote 
	\end{flushright}
\end{introduction}
\lettrine[findent=1pt, nindent=0pt]{P}{er}  % TO DO: intro
\section{Flusso di un campo vettoriale}
% TO DO: concezione intuitiva di flusso
\begin{define}[Flusso di un campo vettoriale]
	Il \textbf{flusso di un campo vettoriale attraverso una superficie orientata}\index{flusso di un campo vettoriale} $\Sigma$, parametrizzata da una funzione $\vba{r}(u,v)$, è
	\begin{equation}
		\Phi_{\Sigma}\left(\vba{E}\right)=\int_{\Sigma}\vba{E}\vdot\vbh{u}_nd\Sigma=\int_{\Sigma}\vba{E}\vdot\frac{\partial \vba{r}}{\partial u}\cross\frac{\partial \vba{r}}{\partial v}dudv
	\end{equation}
\end{define}
\begin{intuit}
	Se descriviamo la corrente di un fluido come l'acqua con un campo vettoriale $\vba{F}$, il flusso di $\vba{F}$ rappresenta \textit{quanto fluido} passa \textit{attraverso} una certa superficie per unità di tempo (anche se quest'ultima viene spesso sottointesa).\\
	Con questa interpretazione euristica si può capire anche perché l'integrale presenta nella definizione un \textit{prodotto scalare}: se l'acqua scorre perpendicolarmente alla superficie, molta acqua passerà e il flusso sarà dunque grande; al contrario, se il fluido scorre parallelamente alla superficie l'acqua non l'attraverserà mai e quindi il flusso è nullo. In altre parole, ciò che influisce sul flusso è la componente del flusso perpendicolare alla superficie!
\end{intuit}
Matematicamente parlando, il flusso non è altro che un tipo di integrale superficiale di un campo vettoriale. % TO DO: link ai richiami.\\

Come abbiamo detto, la superficie deve essere \textbf{orientabile}: detto in una maniera suggestiva, intuitiva ma non formale come farebbero i fisici, una superficie con due \textit{facce distinte} e due \textit{orientazioni} possibili che corrispondono alla scelta di un \textit{campo normale} che punta sempre dalla parte di una delle facce.\\
In particolare, la superficie deve essere effettivamente orientata, ossia si deve scegliere uno dei campi normali in modo da definire quando il flusso è positivo e quando è negativo. Generalmente, per convenzione si impone che il vettore normale alla superficie è orientato verso l'\textit{esterno}: quando la componente perpendicolare del campo vettoriale $\vba{E}$ e il vettore normale saranno \textit{concordi}, cioè quando $\vba{E}$ è \textit{uscente} dalla superficie, si ha un flusso \textit{positivo}; se il campo $\vba{E}$ è \textit{entrante} la superficie, allora si ha un flusso \textit{negativo}.
\begin{observe} % TO DO: confrontare mazzoldi e note docente, riformulare che non è molto chiaro
	Data una superficie chiusa $\Sigma$, tracciamo una curva chiusa $\gamma$ su di essa; possiamo scindere $\Sigma$ in due sottosuperfici $\Sigma_1$ e $\Sigma_2$ che hanno in comune una superficie $\Sigma_{1,2}$ delimitata da $\gamma$.
	Il flusso per linearità di scinde in
	\begin{equation*}
		\Phi_{\Sigma}=\Phi_{\Sigma_1}+\Phi_{\Sigma_2}
	\end{equation*}
	In realtà, il flusso non è influenzato da quale sia la superficie  $\Sigma_{1,2}$: infatti, per uno dei sottoflussi il contribuito dato da $ \Sigma_{1,2}$ sarà negativo perché il campo è entrante, ma per l'altro sottoflusso sarà positivo perché il campo è uscente.
	\begin{equation*}
		\Phi_{\Sigma}=\Phi_{\Sigma_1}+\Phi_{\Sigma_2}=\Phi_{\Sigma_1-\Sigma_{1,2}}+\Phi_{\Sigma_{1,2}}+\Phi_{\Sigma_2-\Sigma_{1,2}}-\Phi_{\Sigma_{1,2}}=\Phi_{\Sigma_1-\Sigma_{1,2}}+\Phi_{\Sigma_2-\Sigma_{1,2}}
	\end{equation*}
\end{observe}
\section{Legge di Gauss}
% https://mathinsight.org/surface_integral_vector_field_introduction4
% http://sites.science.oregonstate.edu/math/home/programs/undergrad/CalculusQuestStudyGuides/vcalc/flux/flux.html
% https://mathinsight.org/vector_field_fluid_flow
% https://tutorial.math.lamar.edu/classes/calciii/SurfIntVectorField.aspx
\begin{theorema}[Legge di Gauss]\index{legge!di Gauss}
	Il flusso del campo elettrostatico $\vba{E}$ attraverso un superficie \textbf{chiusa} è uguale alla somma algebrica (\textrm{o nel caso di una distribuzione continua}, dell'integrale) delle cariche contenute all'\textbf{interno} della superficie, comunque siano distribuite, divisa per $\epsilon_0$.
	\begin{itemize}
		\item \textbf{Caso discreto:}
		\begin{equation}
			\Phi_\Sigma\left(\vba{E}\right)=\frac{\left(\sum_i q_i\right)_{int}}{\epsilon_0}\label{LeggeGaussDiscreta}
		\end{equation}
		\item \textbf{Caso continuo:}
		\begin{equation}
			\Phi_\Sigma\left(\vba{E}\right)=\frac{1}{\epsilon_0}\int_{V}\rho\left(x,y,z\right)dV\qquad\text{tale che}\ \partial V=\Sigma\label{LeggeGaussContinua}
		\end{equation}
	\end{itemize}
\end{theorema}
Lo dimostreremo per una \textit{singola} carica contenuta nella superficie - dato che il caso per molteplici cariche e per una distribuzione continua seguono praticamente immediatamente - ma prima di farlo in modo formale, vediamo una derivazione più ‘‘fisica''.
\paragraph{Angolo solido}
Per far ciò, ci servirà la nozione di \textit{angolo solido}.
\begin{define}[Angolo solido]
	L'\textbf{angolo solido}\index{angolo solido} è una generalizzazione a tre dimensioni dell'angolo piano e dà una misura della parte di spazio compresa entro un fascio di semirette uscenti intorno ad un punto $P$. In termini matematici, esso è definito come l'area sulla sfera unitaria intorno a $P$ individuata dalla superficie (finita) $\Sigma$:
	\begin{equation}
		\Omega=\int d\Omega =\int \frac{d\Sigma_0}{r^2}=\int\frac{cos\theta d\Sigma}{r^2}
	\end{equation}
	dove
	\begin{itemize}
		\item $d\Omega$ è l'angolo solido infinitesimo.
		\item $d\Sigma:0$ è la \textit{proiezione ortogonale} al raggio dell'elemento infinitesimo di superficie $d\Sigma$.
		\item $\theta$ è l'\textit{angolo polare} delle coordinate sferiche.
	\end{itemize}
% https://tutorial.math.lamar.edu/classes/calciii/SurfIntVectorField.aspx
% https://betterexplained.com/articles/flux/
\end{define}
Poiché $d\Sigma_0$ è un elemento infinitesimo della calotta sferica, data una parametrizzazione in coordinate sferiche vale
\begin{equation*}
	d\Sigma_0=r^2\sin\theta d\theta d\phi
\end{equation*}
da cui segue che
\begin{equation}
	d\Sigma=\sin\theta d\theta d\phi
\end{equation}
Integrando $\theta$ da $0$ a $\pi$ e $\phi$ da $0$ a $2\pi$, si ottiene l'angolo solido sotto cui dal centro $P$ è vista tutta la superficie:
\begin{equation}
	\Omega=\int d\Omega=\int_0^{2\pi}d\phi\int_0^\pi d\theta\sin\theta=4\pi
\end{equation}
Questo risultato è valido per una qualunque superficie \textit{chiusa} che racchiuda $P$ - e ne corrisponde al valore massimo dell'angolo solido.\\
\begin{units}~\\
	\textbf{\textsc{Angolo solido:}} Steradiante  ($\unit{\steradian}$).\\
	\textit{Dimensioni:} $[\Omega]=\mathsf{1}$.
\end{units}
\paragraph{Derivazione fisica della legge di Gauss}
Dato il campo di Coulomb $\vba{E}$ generato dalla carica $q$, vogliamo determinare l'elemento di flusso infinitesimo $d\Phi\left(\vba{E}\right)$, ossia il flusso tramite l'elemento d'area infinitesimo $d\Sigma$.

Innanzitutto, si noti che l'angolo tra il versore radiale $\vbh{u}_r$ uscente dalla carica $q$ e il versore normale $\vbh{u}_n$ alla superficie coincide con un possibile angolo polare $\theta$ che parametrizza un punto della calotta sferica unitaria centrata in $q$.
\begin{equation*}
\vbh{u}_r\vdot\vbh{u}_n=\cos\theta
\end{equation*}
Il flusso infinitesimo diventa
\begin{equation*}
	d\Phi_{\Sigma}\left(\vba{E}\right)=\vba{E}\vdot\vbh{u}_nd\Sigma=\frac{q}{4\pi\epsilon_0}\frac{\vbh{u}_r\vdot\vbh{u}_n}{r^2}d\Sigma=\frac{q}{4\pi\epsilon_0}\frac{\cos\theta}{r^2}d\Sigma=\frac{q}{4\pi\epsilon_0}\frac{d\Sigma_0}{r^2}=\frac{q}{4\pi\epsilon_0}d\Omega
\end{equation*}
\begin{observe}
	Il flusso del campo $\vba{E}$ generato da una carica puntiforme dipende solo dall'angolo solido e \textit{non} dalla superficie o dalla distanza dalla carica: il flusso è lo stesso per qualunque superficie il cui bordo si appoggi sul cono individuato dall'angolo solido.\\
	Questo è una \textit{diretta} conseguenza che il campo di Coulomb presenta un fattore $\nicefrac{1}{r^2}$; se la relazione fosse stata anche solo leggermente diversa non varrebbe tale dipendenza.
\end{observe}
Per una superficie (finita) e chiusa che racchiude la carica $q$ si ha
 \begin{equation*}
 	\Phi_\Sigma\left(\vba{E}\right)=\frac{q}{4\pi\epsilon_0}\Omega=\frac{q}{\epsilon_0}
 \end{equation*}
\paragraph{Dimostrazione formale della legge di Gauss}
\begin{demonstration}
	Per semplicità, poniamo l'origine del nostro sistema di riferimento dove è situata la carica. Data la simmetria di carattere radiale fornita dal campo elettrostatico di Coulomb, ci conviene utilizzare le coordinate sferiche
	\begin{equation*}
		\begin{cases}
			x=r\sin\theta\cos\phi\\
			y=r\sin\theta\sin\phi\\
			z=r\cos\theta
		\end{cases}
	\end{equation*}
	Parametrizziamo la superficie $\Sigma$ con l'angolo \textit{polare} $\theta$ e l'angolo \textit{azimutale} $\phi$ delle coordinate sferiche:
	\begin{align*}
		\vba{r}(\theta,\phi)&=x(\theta,\phi)\vba{u}_x+y(\theta,\phi)\vba{u}_y+z(\theta,\phi)\vba{u}_z=\\
		&=r(\theta,\phi)\sin\theta\cos\phi\vba{u}_x+r(\theta,\phi)\sin\theta\sin\phi\vba{u}_y+r(\theta,\phi)\cos\theta\vba{u}_z
	\end{align*}
	Osserviamo che per descrivere una superficie con le coordinate sferiche è necessario fornire la \textit{distanza} $r(\theta,\phi)$ dall'origine nella direzione indicata dagli angoli $\theta$ e $\phi$. Anzi, la parametrizzazione può essere espressa totalmente in termini \textit{radiali}! Infatti, il versore radiale è dato da
	\begin{equation*}
		\vbh{u}_r=\frac{\vbh{e}_r}{\abs{\vbh{e}_r}}=\frac{\pdv{x^i}{r}\vbh{u}_i}{1}=\sin\theta\cos\phi\vbh{u}_x+\sin\theta\sin\phi\vbh{u}_y+\cos\theta\vbh{u}_z
	\end{equation*}
	Raccogliendo $r(\theta,\phi)$ dalla parametrizzazione scritta prima si ottiene quindi
	\begin{equation*}
		\vba{r}(\theta,\phi)=r(\theta,\phi)\vbh{u}_r
	\end{equation*}
	Per definizione, il flusso è
	\begin{equation*}
		\Phi_{\Sigma}\left(\vba{E}\right)=\int_{\Sigma}\vba{E}\vdot\vbh{u}_nd\Sigma=\int_{\Sigma}\vba{E}\vdot\vbh{u}_n\abs{\pdv{\vba{r}(\theta,\phi)}{\theta}\cross\pdv{\vba{r}(\theta,\phi)}{\phi}}d\theta d\phi\squarequal
	\end{equation*}
	Poiché il versore normale è
	\begin{equation*}
		\vbh{u}_n=\frac{\pdv{\vba{r}(\theta,\phi)}{\theta}\cross\pdv{\vba{r}(\theta,\phi)}{\phi}}{\abs{\pdv{\vba{r}(\theta,\phi)}{\theta}\cross\pdv{\vba{r}(\theta,\phi)}{\phi}}}
	\end{equation*}
	il flusso si può calcolare come
	\begin{equation*}
		\squarequal\int_{\Sigma}\vba{E}\vdot\pdv{\vba{r}(\theta,\phi)}{\theta}\cross\pdv{\vba{r}(\theta,\phi)}{\phi} d\theta d\phi=\frac{q}{4\pi\epsilon_0}\int_{\Sigma}\frac{1}{r(\theta,\phi)^2}\vbh{u}_r\vdot \pdv{\vba{r}(\theta,\phi)}{\theta}\cross\pdv{\vba{r}(\theta,\phi)}{\phi} d\theta d\phi
	\end{equation*}
	Per semplificare quel prodotto misto, dobbiamo prima analizzare i termini che partecipano al prodotto vettoriale.\\
	In un generico punto\footnote{Qui indicato tramite le coordinate ad esso associate dalla parametrizzazione.} $\left(\theta,\phi\right)$ della superficie, i vettori della base del piano tangente alla superficie in tal punto sono
	\begin{equation*}
		\begin{cases}
			\pdv{\vba{r}}{\theta}=\pdv{\theta}\left(r(\theta,\phi)\vba{u}_r\right)=\pdv{r(\theta,\phi)}{\theta}\vbh{u}_r+r(\theta,\phi)\pdv{\vbh{u}_r}{\theta}\\
			\pdv{\vba{r}}{\phi}=\pdv{\phi}\left(r(\theta,\phi)\vba{u}_r\right)=\pdv{r(\theta,\phi)}{\phi}\vbh{u}_r+r(\theta,\phi)\pdv{\vbh{u}_r}{\phi}\\
		\end{cases}
	\end{equation*}
	Si nota subito che  le componenti parallele a $\vba{u}_r$ \textit{non} influiscono al flusso. Al netto di costanti moltiplicative, il contribuito di tali componenti è un $\vba{u}_r$ nel prodotto vettoriale del prodotto misto, ma valendo
\begin{equation*}
	\begin{cases}
		\vbh{u}_r\vdot\vbh{u}_r\cross\vba{a}=0\\
		\vbh{u}_r\vdot\vba{a}\cross\vbh{u}_r=0
	\end{cases},\quad\forall \vba{a}\text{ vettore}
\end{equation*}
tali componenti non cambieranno in alcun modo il flusso; ciò che invece lo cambia sono le derivate dei versori radiali. Sviluppando, l'espressione del flusso si ha
\begin{align*}
	\Phi_{\Sigma}(\vba{E})&=\frac{q}{4\pi\epsilon_0}\int_{\Sigma}\frac{1}{r(\theta,\phi)}\vbh{u}_r\vdot \pdv{\vba{r}(\theta,\phi)}{\theta}\cross\pdv{\vba{r}(\theta,\phi)}{\phi} d\theta d\phi=\\
	&=\frac{q}{4\pi\epsilon_0}\int_{\Sigma}\frac{1}{\Ccancel[red]{r(\theta,\phi)^2}}\vbh{u}_r\vdot\left(\Ccancel[red]{r(\theta,\phi)}\pdv{\vbh{u}_r}{\theta}\cross \Ccancel[red]{r(\theta,\phi)}\pdv{\vbh{u}_r}{\phi}\right)d\theta d\phi=\\
	&=\frac{q}{4\pi\epsilon_0}\int_{\Sigma}\vbh{u}_r\vdot\pdv{\vbh{u}_r}{\theta}\cross\pdv{\vbh{u}_r}{\phi}d\theta d\phi
\end{align*}
Poiché
\begin{equation*}
	\begin{cases}
		\pdv{\vbh{u}_r}{\theta}=\cos\theta\cos\phi\vbh{u}_x+\cos\theta\sin\phi\vbh{u}_y-\sin\theta\vbh{u}_z\\
		\pdv{\vbh{u}_r}{\phi}=-\sin\theta\sin\phi\vbh{u}_x+\sin\theta\cos\phi\vbh{u}_y\\
\end{cases}
\end{equation*}
e
\begin{equation*}
	\vbh{u}_r\vdot\pdv{\vbh{u}_r}{\theta}\cross\pdv{\vbh{u}_r}{\phi}=\begin{vmatrix}
		\sin\theta\cos\phi & \sin\theta\sin\phi & \cos\theta\\
		\cos\theta\cos\phi & \cos\theta\sin\phi & -\sin\theta\\
		-\sin\theta\sin\phi & \sin\theta\cos\phi & 0
	\end{vmatrix}=\sin\theta
\end{equation*}
segue che
\begin{equation}
	\Phi_{\Sigma}(\vba{E})=\frac{q}{4\pi\epsilon_0}\int_{\Sigma}\sin\theta d\theta d\phi=\frac{q}{4\pi\epsilon_0}\Omega\label{LeggeGaussGeneralizzata}
\end{equation}
dove $\Sigma$ è l'angolo solido sull'intera superficie.\\
Se la superficie (finita) è chiusa si ha $\Omega=4\pi$, ottenendo quindi il risultato desiderato.
\end{demonstration}
\begin{observe}
	La \eqref{LeggeGaussGeneralizzata} descrive il flusso del campo elettrostatico attraverso una superficie \textbf{qualunque}. La legge di Gauss si potrebbe vedere come un \textit{caso specifico} di questa relazione.
\end{observe}
Il caso per cariche multiple segue dal \textit{principio di sovrapposizione} dei campi elettrici:
\begin{equation*}
	\Phi_{\Sigma}\left(\vba{E}\right)=\int_{\Sigma} \vba{E}\vdot \vbh{u}_nd\Sigma=\int_{\Sigma}\left(\sum_i\vba{E}_i\right)\vdot\vbh{u}_nd\Sigma=\sum_i\int_{\Sigma}\vba{E}_i\vdot\vbh{u}_nd\Sigma=\sum_i\frac{q_i}{\epsilon_0}
\end{equation*}
Da questa si ottiene, passando al continuo, la relazione \ref{LeggeGaussContinua}.
\begin{observe}\label{LeggeGaussMoltoGeneralizzata}
	La dimostrazione della legge di Gauss si basa sul fatto fondamentale che la legge di Coulomb, che descrive l'interazione tra cariche elettriche, è \textit{inversamente proporzionale} a $r^2$. È dunque possibile adattare la legge di Gauss in altri contesti non elettrici, se consideriamo una forza tra due enti inversamente proporzionale a $r^2$.\\
	Ad esempio, esiste una formulazione della legge di Gauss per la \textit{forza di gravitazione} completamente equivalente alla legge di gravitazione universale di Newton: il flusso del campo gravitazionale attraverso una superficie chiusa è pari alla massa inclusa in essa, moltiplicata per $-4\pi G$.
	\begin{itemize}
		\item \textbf{Caso discreto:}
		\begin{equation}
			\Phi_\Sigma(\vba{G})=-4\pi G\sum_i m_i
		\end{equation}
		\item \textbf{Caso continuo:}
		\begin{equation}
			\Phi_\Sigma(\vba{G})=-4\pi G\int_{V}\rho\left(x,y,z\right)dV\qquad\text{tale che}\ \partial V=\Sigma
		\end{equation}
	\end{itemize}
	Si noti, tra l'altro, che non è particolarmente differente dal caso elettrico dato che la legge di Gauss per il campo elettrico si può anche scrivere come \begin{equation*}
		\Phi_\Sigma\left(\vba{E}\right)=4\pi k\sum_i q_i
	\end{equation*}
	o
	\begin{equation}
		\Phi_\Sigma\left(\vba{E}\right)=4\pi k\int_{V}\rho\left(x,y,z\right)dV\qquad\text{tale che}\ \partial V=\Sigma
	\end{equation}
\end{observe}
\paragraph{Flusso tramite una superficie chiusa per una carica esterna}
La legge di Gauss descrive il flusso tramite una superficie chiusa tenendo conto delle cariche \textit{interne} ad essa... e si ci fossero delle cariche \textit{esterne}?

Limitiamoci all'inizio al caso di una singola carica esterna: il campo di Coulomb entra nella superficie chiusa, attraversa lo spazio contenuto da essa e poi esce dall'altro lato. In termini di angolo solido, il cono elementare che sottende l'angolo solido infinitesimo $d\Sigma$ determina sulla superficie chiusa due elementi $d\Sigma_1$ e $d\Sigma_2$. Per la convenzione sul segno del flusso:
\begin{itemize}
	\item $\vba{E}$ \textit{entra} in $d\Sigma_1$: $\vba{E}\vdot \vbh{u}_n d\Sigma_1<0$.
	\item $\vba{E}$ \textit{esce} da $d\Sigma_2$: $\vba{E}\vdot \vbh{u}_n d\Sigma_2>0$.
\end{itemize}
I flussi infinitesimi che otteniamo\footnote{Il procedimento è analogo a quello con cui si ottiene l'equazione \ref{LeggeGaussGeneralizzata}.} sono
\begin{equation*}
	\begin{cases}
		d\Phi_{\Sigma_1}(\vba{E})=\vba{E}\vdot \vbh{u}_n d\Sigma_1=-\frac{q}{4\pi\epsilon_0}d\Omega\\
		d\Phi_{\Sigma_2}(\vba{E})=\vba{E}\vdot \vbh{u}_n d\Sigma_2=\frac{q}{4\pi\epsilon_0}d\Omega
	\end{cases}
\end{equation*}
Integrando sull'intera superficie chiusa otteniamo
\begin{equation}
	\Phi_\Sigma\left(\vba{E}\right)=\int_{\Sigma}\vba{E}\vdot\vbh{u}_nd\Sigma=0
\end{equation}
Il flusso tramite una superficie \textit{chiusa} dipende \textit{solo} dalle cariche interne ad essa.
\begin{observe}
	Cosa cambia dal caso della carica interna? Il campo elettrico in quella situazione risulta essere \textit{entrante} (se la carica è positiva) o \textit{uscente} (se la carica è negativa) da ogni elemento infinitesimo; il flusso avrà quindi sempre lo stesso segno oppure essere nullo, ma sulla superficie intera questo si ha solo se questa è parallela al campo.   
\end{observe}
\section{Applicazioni della legge di Gauss}
La legge di Gauss, in linea di principio, ci descrive solo il flusso tramite una superficie chiusa. Tuttavia, in situazioni di \textit{evidenti simmetrie}, confrontando la definizione di flusso con quello ottenuto dalla legge di Gauss possiamo sorprendentemente calcolare in modo abbastanza facile il campo elettrostatico che genera il flusso.
\begin{attention} % TO DO: dove cazzo lo metto
	Bisogna fare attenzione ad utilizzare la legge di Gauss in assenza di simmetrie. Ad esempio, consideriamo una situazione come in figura.
	
	Qui il flusso è nullo perché ciò che entra esce, ma il campo elettrico non è nullo.
\end{attention}
\paragraph{Filo carico rettilineo (infinito)}
	Si consideri un filo rettilineo infinito con densità lineare costante $\lambda$. Per semplicità, poniamo il sistema di riferimento in modo che il filo carico sia lungo l'asse $z$. Poniamo un cilindro attorno al filo in modo che il filo passi per l'asse del cilindro. Data l'evidente simmetria cilindrica del sistema usiamo, per motivi che dovrebbero essere oramai chiari, le coordinate cilindriche:
\begin{equation*}
	\begin{cases}
		x=R\cos\theta\\
		y=R\sin\theta\\
		z=z
	\end{cases}
\end{equation*}
Oltre ad essere un sistema di riferimento, fissato $R$ abbiamo una parametrizzazione del cilindro di raggio $R$.

Ora, ci è già noto che in questo sistema di riferimento il campo elettrostatico dipende esclusivamente dalla coordinata radiale e ha direzione radiale, ossia $\vba{E}=E(R)\vbh{u}_R$. Per come abbiamo posto il cilindro $\Sigma$, il versore normale alla superficie laterale coincide con quello radiale delle coordinate cilindriche, pertanto
\begin{equation*}
	\Phi_{\Sigma}\left(\vba{E}\right)=\int\vba{E}\cdot\vbh{u}_nd\Sigma=\int\vba{E}\cdot\vbh{u}_Rd\Sigma=\int E(R)\vbh{u}_R\vdot\vbh{u}_Rd\Sigma=E(R)\int d\Sigma\squarequal
\end{equation*}
dove l'ultimo passaggio è lecito in quanto sulla superficie del cilindro il raggio è fissato e quindi anche $E(R)$ è costante.\\
Dato che l'elemento di area è dato da $d\Sigma=d\Phi dz$, si ha
\begin{equation*}
	\squarequal\lim_{L\to+\infty}2\pi R L E(R)
\end{equation*}
Per la legge di Gauss,
\begin{equation*}
	\Phi_{\Sigma}\left(\vba{E}\right)=\frac{q}{\epsilon_0}=\frac{\lambda L}{\epsilon_0}
\end{equation*}
dove $\lambda$ è la densità lineare di carica; per ottenere il flusso per il filo infinito ci basterebbe mandare $L$ all'infinito. Eguagliando i due flussi ottenuti si ricava che
\begin{equation*}
	2\pi R\Ccancel[red]{L}E(R)=\frac{\lambda \Ccancel[red]{L}}{\epsilon_0}
\end{equation*}
e quandi
\begin{equation}
	\vba{E}(R)=\frac{\lambda}{2\pi\epsilon_0 R}
\end{equation}
\paragraph{Superficie carica infinita}
Si consideri una superficie piana $\Sigma$ con densità superficiale costante $\sigma$. Per semplicità, poniamo il sistema di riferimento in modo che la superficie coincida con il piano $x=0$. Come per il caso del filo carico rettilineo, consideriamo un cilindro, questa volta che interseca la superficie ortogonalmente e posto in maniera che le basi siano alla stessa distanza dal piano.

Ci è già noto che il campo elettrostatico dipende esclusivamente dalla coordinata perpendicolare al piano, cioè $x$, e ha direzione $\vbh{u}_x$. In particolare, si osservi che da facce opposte del piano il versore normale $u_n=u_x$ cambia verso e quindi cambia verso anche il campo elettrostatico:
\begin{equation*}
	\vba{E}=
	\begin{cases}
		E(\abs{x})\vbh{u}_x&\text{se}\ x>0\\
		-E(\abs{x})\vbh{u}_x&\text{se}\ x<0
	\end{cases}
\end{equation*}
Il flusso tramite il cilindro $\Sigma$ si può scindere in tre componenti: i due flussi $\Phi_{A_1}$ e $\Phi_{A_2}$ attraverso le basi e il flusso $\Phi_{SL}$ attraverso la superficie laterale; tuttavia, poiché la superficie laterale è sempre ortogonale al campo, quest'ultima componente è nulla. Inoltre, si ha che il campi èsce sempre dalle basi, pertanto i flussi saranno positivi e, per questioni di simmetria, coincidono:
\begin{equation}
	\Phi_{A_1}=\Phi_{A_2}
\end{equation}
Pertanto,
\begin{equation*}
	\Phi_{\Sigma}\left(\vba{E}\right)=2\int_A E(\abs{x})d\Sigma=2 E(\abs{x})\int_{A_1}d\Sigma=2E(\abs{x})A_1
\end{equation*}
Ricordando che la densità di carica superficiale $\sigma$ è costante, la carica interna al cilindro è data da
\begin{equation*}
	q=\sigma A_1
\end{equation*}
Per la legge di Gauss si ha
\begin{equation*}
	\Phi_{\Sigma}=\frac{\sigma A_1}{\epsilon_0}
\end{equation*}
Eguagliando i due flussi ottenuti si ricava che
\begin{equation*}
	2\Ccancel[red]{A_1}E(\abs{x})=\frac{\sigma \Ccancel[red]{A_1}}{\epsilon_0}
\end{equation*}
e quindi
\begin{equation}
	\vba{E}(\abs{x})=\frac{\sigma}{2\epsilon_0}\vbh{u}_x
\end{equation}
\paragraph{Sfera uniformemente carica}
Si consideri una palla sferica di raggio $R$ con densità volumica costante $\rho$. Per semplicità, poniamo il sistema di riferimento in modo che l'origine coincida con il centro della sfera. Come superficie $\Sigma$ per calcolare il flusso scelgo una sfera di raggio $r$ centrata anch'essa nell'origine.

Ci è già noto che il campo elettrostatico dipende esclusivamente dalla coordinata radiale e ha direzione radiale, ossia $\vba{E}=E(r)\vbh{u}_r$. Il versore normale alla superficie $\Sigma$ è $\vbh{u}_n=\vbh{u}_r$.
A questo punto distinguiamo il calcolo quando la superficie sferica ha raggio maggiore della palla ($r>R$) o quando ha raggio minore ($r<R$).
\subparagraph{Il caso esterno: $r>R$}
In questo caso la superficie sferica \textit{contiene} la sfera uniformemente carica. Il flusso dunque è
\begin{equation*}
	\Phi_{\Sigma}\left(\vba{E}\right)=\int E(r)d\Sigma=E(r)4\pi r^2
\end{equation*}
dato che $E(r)$ è costante sulla sfera di raggio $r$.
La superficie sferica contiene \textit{tutta} la carica della palla al suo interno, quindi per la legge di Gauss
\begin{equation*}
	\Phi_{\Sigma}(\vba{E})=\frac{q}{\epsilon_0}=\frac{4 \pi R^3\rho}{3\epsilon_0}
\end{equation*}
e quindi
\begin{equation}
	\vba{E}(r)=\frac{\rho R^3}{3\epsilon_0r^2}\vbh{u}_r=\frac{q}{4\pi\epsilon_0 r^2}\vbh{u}_r
\end{equation}
\begin{observe}
	Una qualunque distribuzione di carica a simmetria \textit{sferica} dipendente dalla distanza radiale, ossia $\rho(x,y,z)=\rho(r)$, genera al suo esterno un campo uguale a quello di una carica puntiforme.
\end{observe}
\subparagraph{Il caso interno: $r<R$}
In questo caso la superficie sferica contiene al suo interno solo una parte della carica complessiva:
\begin{equation*}
	q_{interna}=\frac{4}{3}\pi r^3\rho
\end{equation*}
Se il flusso, calcolato secondo la definizione, non cambia espressione (ma valore sì!)...
\begin{equation*}
	\Phi_{\Sigma}\left(\vba{E}\right)=E(r)4\pi r^2
\end{equation*}
... quello per la legge di Gauss diventa
\begin{equation*}
	\Phi_{\Sigma}(\vba{E})=\frac{q_{interna}}{\epsilon_0}=\frac{4 \pi r^3\rho}{3\epsilon_0}
\end{equation*}
e quindi
\begin{equation}
\vba{E}(r)=\frac{q_{interna}}{4\pi\epsilon_0 r^2}\vbh{u}_r=\frac{\rho r}{3\epsilon_0}\vbh{u}_r
\end{equation}

Riassumendo, il campo elettrico generato da una sfera uniformemente carica di raggio $R$ a distanza $r$ dall'origine è
\begin{equation}
	\vba{E}(r)=\begin{cases}
		\frac{\rho r}{3\epsilon_0}\vbh{u}_r&\text{se}\ r\leq R\\
		\frac{q}{4\pi\epsilon_0 r^2}\vbh{u}_r &\text{se}\ r\geq R
	\end{cases}
\end{equation}