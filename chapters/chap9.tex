% SVN info for this file
\svnidlong
{$HeadURL$}
{$LastChangedDate$}
{$LastChangedRevision$}
{$LastChangedBy$}


\subsection{Digressione sulla serviazione della legge generale di Ampére}
riprendiamo la legge di Ampere: la circuitazione del campo magnetico nel caso stazionario è data da:
\begin{equation*}
	
	
	\end{equation*}

riscrivendo 
ottengo relaziona vettoriale in forma differenziale\\
ERa nel caso di un filo rettilineo infinito e calcoliamo la circuitazione e $\gamma$ curva qualsiasi\\
Perché abbiamo lavorato un filo rettilineo \textit{infinito}?\\
Prendo un circuito qualsiasi , ci sarà un certo campo magnetico. Data un'altra curva $\gamma$ potrei calcolarmi la circuitazione attraverso la curva $\gamma$ qualunque del campo magnetico
\begin
	
\end
ds sarà lungo $\gamma$. Se le curve chiuse si intersecano la circuitazione dovrà essere $\mu_0 i$ per Ampere.
Parametrizzo $\gamma$ come $\r(\phi_1)$ e parametrizzo il circuito con un'ltra curva $\gamm_2$ . $ds_1$ è 
Uso la legge di Laplace
\begin{equation*}
	B=\frac{\mu_0 i}{4\pi}\int_{\gamma_2}
\end{equation*}
è il campo magnetico B calcolato nel vettoer posizione $r$

Quindi il campo magnetico B è valutato sull'altra curva $B(r(\phi_1))$. 
poso cambiare l'ordine perché è un determinante: prodotto vettoriale per scalare
Riconosciamo lintegrale ...
Dalla legge di Ampere dovrebbe essere $\mu_0 i$, quindi dovrebbe essere 1. Riconosciamo il \textit{Gauss linking number} \index{Gauss!linking number} che è un invariante topologico rilevante con teoria dei nodi: conta il numero di volte che le curve si intersecano con segno %lezioni di topologia yt
Il linking number che conta le intersezioni è $\pm 1$, gazie all'orientamento delle curve h oil segno\\
Si ha che il linkingnumber è 
\begin{equatino*}
	n=\frac{}{}
\end{equation*}
La mappa di Gauss è una mappa $G$ che mappa una qualsiasi sup in R^3 sulla sfera di raggio1, ma in questo caso mappa un toro sulla sfera prendendo la differenza delle parametrizazioni
$\funztot{S^1\times S^}{S^2}{(\phi_1, \phi_2)}{\frac{r(\phi_1)}{} }$, che è sicuramente sulla fera di raggio 1 perché ha norma 1. La mappa ricopre la sfera più volte: non è univoca la funzione! Questo integrale $n$ è l'area dell'immagine della mappa di Gauss. Mappo il toro in una sfera e l'immagine della mappa non è biunivoca: se le due curve si intersecano una volta la mappa ricopre al sfera due volte: conta l'area delle immagini della mappa di Gauss divisa per $4\pi$, cioé quante volte la mappa ricopre la sfera in base a quante volte si intersecano. è un fatoo di teoria dei nodi. La legge di Ampere emerge in questo modo dal caso generale.
Si nota che $n$ è sempre intero, se le curve sono "separate" l'integrale sarà $0$\\

%CIT: ci sono domande a cui io sappia rispondere su questo?

Uso della legge di Ampére per ricavare il campo magnetico, esattamente come abbiamo ricavato il potenziale di Coulomb con la legge di Gauss.
\begin{example}[Filo rettilineo infinito]
	Per simmetria il campo magnetico dipende solo dal raggio, lungo $u_\phi$ che indica lo spostamento angolare.\\
	La circuitazione è $\mu_0 i$. Prendo una superficie $\gamma$   per portare fuori il campo magnetico dall'integrale, come per la legge di Gauss per non fare il flusso. Scelgo una superficie per cui l'integrale è costante: circonferenza di raggio $R$. Siccome l'integrale è in d\phi ma B dipende solo da $R$, ottengo
	\begin{gather*}
		
		
	\end{gather*}
	Abbiamo così derivato la legge di Biot- Savart (?). è così emplice perché ho una simmetria particolare dietro, in generale non è vero. Cooscere la circuitazione o il rotore di solito non è sufficiente per trovare il campo vettoriale
\end{example}

\begin{example}[Solenoide infinito]
	Per simmetria so che il solenoide infinito produce un campo magnetico all'interno del solenoide e nessun campo magnetico all'esterno del solenoide stesso.\\
	Prendo una curva $\gamma$ rettangolare e calcolo la circuitazione lungo quella curva. Il campo magnetico potrebbe dipendere dal raggio ed è diretto lungo u_z, asse z diretto verso l'alto.\\
	La circuitazione lungo la cruva $\gamma$
		\begin{gather*}
			
			
		\end{gather*}
	L'unico tratto che contribuisce è l'integrale fra A e B, BC ds è ortogonale a B(idem AD), CD invece fa 0 al campo magnetico perché è esterno al solenoide. Sto integrando lungo la verticale, quindi  u_z\dot ds=z
	DEvo uguaglierlo a \mu_0i totale. i è la corrente che attraversa la spira. i_\gamma è la corrente che attraversa al curva $\gamma$: è data dal numero di spire contenute nella curva. Avevamo definito il numero di spire come l'integrale della densità lineare di spire. La corrente che interseca $\gamma$ è data dall'intensità di corrente nella spire per il numero di spire lì dentro.\\
	Ma quante spire ci sono dentro $\gamma$? FAccio l'integrale fra z_1 e z_2 della densità di spire in dz. Assumo che la densità di spire del solenoide sia costane.
	La $i$ è la densità di corrente che sta girando nel solenoide, quindi di ogni singola spira. Ricavo che B non dipende neanche da $R$ ed è solo $B=\mu_0 i n$. \\
	Recap: Il rettangolo è scelto in modo accurato: solo un lato parallelo al campo magnetico contribuisce all'integrale. Lungo z $B$ è costante. Densità di corrente da numero di spire, ogni spira ha intensità di corrente $i$. Uguaglio alla circuitazione e ritrovo il campo magnetico
	
\end{example}

è l'equivalente dell'elettrostatica del trovare il campo elettrico in condizioni di simmetria grazie alla legge di Gauss.
campo magnetico in caso di simmetrie semplici /elttrostatica: caso della sfera con densità ci carica uniforme e le due piastre che hanno corrente (?) costante


\subsection{Elettrostatica e magnetostatica a confronto}
L'elettrostatica e la magnetostatica si riassumono in 4 equazioni.
%DOUBT: mettiamo un array? alla lavagna è una tabella

Le equazioni sono ancora indipendenti, non sono influenzate una dall'altra: è solo nel caso statico e stazionario in cui non dipendono dal tempo i campi


Siccome il gradiente del campo elettrico è nullo sostituendo nella prima equazione ottengo l'equazione di Poisson \index{Poisson!equaione} \index{equazione! di Poisson}


Siccome la divergenza di B è nulla, B si esprime il rotore di un potenziale vettore $A$. Prendo l'equazione e la inserisco nella seconda.

Nella prima lezione %MANCA REFERENCE
abbiamo visto che il rotore del rotore è ...


Dobbiamo fare vedere che il potenziale $A$. Con $V$ era definito a meno di potenziale, per cui conta solo la differenza di potenziale: la sua classe di equivalenza sarà $V\tilde V+a$. Nel caso del potenziale vettore $A$ c'è qualcosa in più: se definisco $A=A+\nabla\varphi$ allora questo non cambia il campo magnetco, $B=\nabl\times A)=..$ ma il rotore del gradiente è nullo. Qunidi il potenziale vettore $A$ è definito a meno di una costante e a meno di un gradiente. è nota come invarianza di Keich (?). Tale invarianza ci permette di Fissando V all'infinto pari a 0 posso fissare la costante. Posso fissare l'invarianza (scelta rappresentante nella classe di equivalenza) in modo che la divergena di A sia nulla. Se prendo un campo vettoriale A la cui divergenza non è nulla, lo mappo sotto la trasformazione $A\tile Aè\nabla\varphi$. Uso $A'$ con $\nablaA'=0$.... SClego lo scalare $\varphi$ in modo tale che il suo laplaciano sia...
$\varphi$ sarà determinato da un 'equazione differenziale del secondo ordine, faccio la trasformazione A' per avere
è detta scelta di Cage (?).

con questa scelta mando a zero $\nabal(\nabla A)$.

Quindi sia l'elettrostatica sia la magnetostatica sono soluzioni dell'equazione di Poisson, infatti componente per componente i laplaciani di $A$ sono 

Si risuce tutto a 4 eq di Poisson per i potenziali: una per il potenziale scalare che determinano

Data una certa configurazione di $\rho$ e $j$, cioé intensità di carica e corrente.. Determino così i potenziali e da essi i campi.\\

Possiamo far vedere che la soluzione più genreale, date condizioni al contorno fisiche, cioé $V,A$ vanno a $0$ ad infinito, allora la soluzione di equazioni differenziali è data da
\begin{gather*}
	A\\
	V(r)=\frac{}{}\\
	\text{con } dV
	
\end{gather*}
La soluzione che trovo sono 
Controllo con il laplaciano di V(r) e dovrei trovare $\frac{\rho}{\epsilon_0}$.

\\

Esempio pratico di come si usa questa soluzione: esempio ritrito visto in modo diverso. 
\begin{example}[Sfera carica uniformemente]
	Rideriviamo il campo magnetico di una sfera di raggio $R$ carica uniformemente con $\rho(r)..$. Da quella soluzione deriviamo il campo sia interno sia esterno.\\
	Ci mettiamo in coordinate sferiche $x= y= z=$ e facciamo l'integrale $V(r)$. Calcoliamo in anticipo $dV'$. Se scelgo punto in maniera furba ho $|r-r'|=\sqrt{}$. L'integrale sarebbe fra o e $\infty$, am $\rho$ è definito solo all'interno della sfera, qunidi mi limito a r_0. Siccome \rho è costante lo porto via dall'integrale. L'integrale in d\phi'=2\pi perché non dipende da \phi'. Siccome y=\cos\theta'  cambio segni di integrazione, sposto da -1 a 1, lo mangia e va in y. Attenzione alla derivata dell'argomentFacendo la derivata della radice ho 1/2rad per derivata dell'argomento rispetto a y, semplifico il 2, r esce dall'integrale, r' si semplifica e viene $\frac{\rho}{2\epsilon r}$. Valuto in 1, valuto in -1 r+r'  (ho moduli di numeri, r è senza vettore)\\
	Il modulo di numeri positivi (raggi r+r') è positivo, tolgo il modulo. è qui che distinguo il caso dentro la sfera e fuori dalla sfera,	\\
	Fuori dalla sfera:  cioé r>r_0. siccome rì è integrato fra r e r_0, allora sto fuori dal range di integrazione di r': r>r'
	Siccome $\rho$ è la carica sul volume della sfera, sostituisco in $V$ e trovo il potenziale di Coulomb solito. \\
	Il campo elettrico invece
	\\
	Dentro la sfera: r<r_0 facendo l'integrale devo spezzarlo fra 0 ed r e poi da r a r', esco con segno positivo o negativo in base al posizionamento dei punti e quale modulo è più grande (maggiore  o minore di r su sui sto integrando)		$V=\frac{\rho}{2\epsilon_0}\left( r_0^2 - \frac{r^2}{3} \right)$
	Notiamo che il potenziale è continuo per r=r_0, inoltre coincide anche la loro derivata rispetto a r 		campo che cresce linearmente e decade come 1/r^2
	
	%GRAFICI
	
	Potenziale:	In R è continua ed è continua anche la derivata prima: non ho cuspidi. 
	Il campo elettrico invece è lineare fino ad R e poi scende come 1/r^2
	
	Siamo riusciti a fare l'integrale perché la configurazione è simmetrica.
	L'ho visto però partendo dall'equazione di Poisson
\end{example}

\section{Campi elettrici e magnetici variabili nel tempo}
Parte da sperimentazioni di Faraday ed Henry sul fenomerno dell'induzione elettromagnetica. Maxwell a livello teorico per consistenza delle equazioni, ha scoperto che un campo elettricoe variabile nel tempo produce campo magnetico.


Esperimenti da cui possimao dedurre l'induzione elettromagnetica:

\begin{itemize}
	\item spira con struento che misura la crrente che sta passando nel circuito. Prendo un magnete qualunque che produca un campo magnetico di dipolo (come sempre). Avvicino il magnete alla spira con una certa velocità $v$. Lo faccio entrare nella spira circolare. Lo strumento a sinistra misura la corrente diretta in una data direzione. Nel circuito gira corrente in assenza di generatori.\\
	Stesso esperimento ma allontanando il magnete (cambio il verso della velocità, ma il campo magnetico è lo stesso). La corrente che circola è diretta nel verso opposto
	Ho produzione di differenza di potenziale (o forza elettromotrice)che produce un campo elettrico.
	\item spira circolare a cui collego un generatore. La avvicino ad una spira che è collegata a uno strumento che misura corrente. La spira con generatore in movimento si comporta esattamente come un campo magnetico, quindi avvicinandola con una certa velocità v all'altras pira, lo strumento attaccato ad essa misura una corrente. Se la allontano la corrente va in senso opposto
	
	\item Esperimento di Faraday: dato un cilindretto di ferro su cui avvolgo tantissime spire (solenoide con l'anima in ferro), metto interruttore e generatore. Con l'interruttore aperto non ho campo magnetico. Metto una spira circolare di fianco a cui attacco circuito con il solito circuito. A interruttore aperto non c'è campo magneitco né corrente generata. Alla chiusura dell'interruttore si crea un campo magnetico che genera corrente segnalata dallo strumento in una certa direzione, che dura per qualche istante, e poi si ferma. Appena chiudo l'interruttore osservo una corrente. Quando il cmapo magnetico si stabilizza e non vria più nel tempo allora non osservo più corrente.\\
	Riaprendo l'interruttore non ho più campo magnetico, ma èer un istante ossero una corrente diretta nella direzione opposta. Siccome il campo magnetico va dal massimo a 0, quindi avendo una variazione dello stesso nel tempo ho corrente
\end{itemize}

Da queste osservazioni sperimentali si deuce la legge di Fraday

La forza elettromotrice generata (detta indotta, perché è indotta dal campo magnetico) è $\mathcal{E}=$ non conta il campo magnetico uniforme costante nel tempo, la fem è prodotto da campo (anzi di più flusso) magnetico variabile nel tempo. Avvicinando la spira invece del campo magnetico cambia il flusso e si produce corrente. Quello che misuro negli esperimenti è l'intensità di corrente, non la fem. Ma grazie alle leggi di Ohm ricavo la fem per un circuito con resistenza $R$  $i=\frac{  \frac{}{} }{}$.\\

Una fem è un campo elettrico non conservativo. Una fem è prodotta da un campo elettrico non conservativo, in particolare è proprio la def di fem è circuitazione del campo elettrico. Quella del campo elettrostatico è nulla. I contributi non conservativi sono quelli che producono fem. Pila di volta mantiene separazione di carica costante perché energia chimica (esterna che entra nel sistema) è convertita in elettrica. Effetto Haul (?)

Possiamo risrivere la lagge di Faraday come 

Legge di Lenz:
La corrente che viene prodotta si oppone sempre alla causa che l'ha generata.\\
La fem indotta è tale da opporsi alla causa che l'ha generata. 
Pensando all'esperimento delle spire, se una si avvicina all'alta con velocità $v$, il campo magnetico prodotto è fatto , quindi il flusso del campo magnetico avvicinando la spire aumenta (avvicinando le spire iù linee di forza entrano nell'altra spira). La fem produce campo magnetico nella direzione opposta, campo magnetico che si opppone a quell che l'ha generata, cerca di diminuirlo, ed è questo il motivo del segno -
il fem si oppone alla variazione del flusso del campo magnetico


La legge è formulata in modo tale che è la variazione del \textit{flusso} del campo magnetico a dare fem. Come faccio a generare una forza elettromotrice indotta?
La derivata del flusso. Cosa contribuisce al flusso?

Cosa posso cambiare nel tempo? 
\begin{itemize}
	\item $B$
	\item lìangolo della spira rispetto al campo magnetico, questo perché ho un prodotto scalare: muovere una spira all'interno di un campo magnetico non uniforme
	\item la superficie: cambia l'area
	
\end{itemize}


Vedremo che la seconda e la terza causa sono effetti della forza di Lorentz. Quindi solo la variazione di B nel temopo  non è previsto dalle leggi usate finore