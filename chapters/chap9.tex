% SVN info for this file
\svnidlong
{$HeadURL$}
{$LastChangedDate$}
{$LastChangedRevision$}
{$LastChangedBy$}

%LEZ 16 28/03/2022
\subsection{Digressione sulla derivazione della legge generale di Ampère}
Riprendiamo la legge di Ampere, per cui la circuitazione del campo magnetico nel caso stazionario è data da:
\begin{equation*}
	\oint_\gamma \vba B\vdot d\vba s = \mu_0 i	
\end{equation*}
Riscrivendola opportunamente
%TODO: capire che stregoneria ha fatto, temo sia un'integrazione
si ottiene una relaziona vettoriale in forma differenziale quale
\begin{equation*}
	\grad\cross\vba B= \mu_0 \vba j
\end{equation*}
%TODO: inserie disegno del filo infinito: 16.1

Questo era nel caso di un filo rettilineo infinito con la circuitazione calcolata su una curva qualsiasi $\gamma$. Ma perché abbiamo lavorato un filo \textit{rettilineo infinito}?\\
Consideriamo ora un circuito qualsiasi $\gamma_2$, attraversato da un certo campo magnetico.
%TODO: inserie immagine 16.2
Data un'altra curva $\gamma_1$ potrei calcolarmi la circuitazione del campo magnetico attraverso quest'ultimo.
\begin{equation*}
	\oint_\gamma \vba B\vdot d\vba s
\end{equation*}
con $d\vba s$ lungo $\gamma_1$. Se le curve chiuse si intersecano, la circuitazione dovrà essere $\mu_0 i$ per la legge di Ampère.\\
Parametrizzo $\gamma_1$ come $\gamma_1 \colon \vba r(\phi_1)$ e parametrizzo il circuito con un'altra curva $\gamma_2 \colon \vba r(\phi_2)$ Inoltre $d\vba s= \dv{\vba r (\phi_1)}{\phi_1} d\phi_1$.\\
Usando la prima legge di Laplace il campo magnetico generato dal circuito calcolato nel vettore posizione $\vba r$ è
\begin{equation*}
	\vba B (\vba r)=\frac{\mu_0 i}{4\pi}\int_{\gamma_2} d\vba{s}_2 \cross \frac{\vba r - \vba r (\phi_2)}{\abs{\vba r_ - \vba r (\phi_2)}^3}
\end{equation*}

Valutando invece il campo magnetico sull'altra curva $B(r(\phi_1))$ si ha:
\begin{gather*}
	\Gamma_{\gamma_1}(\vba B)= \oint_{\gamma_1}\frac{\mu_0i}{4\pi} \oint_{\gamma_2} d\vba s \cross \frac{\vba r - \vba (\phi_2)}{\abs{\vba r - \vba (\phi_2)}^3} \vdot d\vba{s}_1=\\
	=\frac{\mu_0i}{4\pi}\oint_{\gamma_1} \oint_{\gamma_2} d\vba{s}_1\cross d\vba{s}_2 \vdot \frac{\vba r_1(\phi_1) - \vba r_2 (\phi_2)}{\abs{\vba r_1 - \vba r_2}^3}	
\end{gather*}
Notiamo che possiamo cambiare l'ordine fra prodotto vettoriale e scalare perché si tratta di un determinante.\\
Dalla legge di Ampere dovremmo ottenere $\mu_0 i$, quindi l'integrale dovrebbe essere $1$. Per riottenere effettivamente questo risultato riconosciamo il \textit{Gauss linking number} \index{Gauss!linking number} \index{linking number} che è un invariante topologico rilevante con teoria dei nodi: esso conta il numero di volte che le curve si intersecano con segno. %lezioni di topologia yt
Il linking number che conta le intersezioni è $\pm 1$, con il segno deciso grazie all'orientamento delle curve.\\
%TODO: inserire immgini 16.3 e 16.4
Si ha che il linking number è 
\begin{equation*}
	n=\frac{1}{4\pi} \frac{\vba r_1 - \vba r_2 }{\abs{\vba r_1 - \vba r_2}^3}	
\end{equation*}
Inoltre esso è collegato anche alla \textit{mappa di Gauss} \index{Gauss!mappa di}, una mappa $G$ che mappa una qualsiasi superficie in $\realset^3$ sulla sfera di raggio $1$, ma in questo caso mappa un toro sulla sfera prendendo la differenza delle parametrizzazioni e normalizzandola:
$\funztot[G] {S^1\times S^1} {S^2} {(\phi_1, \phi_2)}{ \frac{\vba r_1(\phi_1) - \vba r_2 (\phi_2)}{\abs{\vba r_1(\phi_1) - \vba r_2(\phi_2)}^3}	}$, che è sicuramente sulla sfera di raggio $1$ perché ha norma $1$. \\
Questo integrale $n$ è l'area dell'immagine della mappa di Gauss. Notiamo che la mappa non è biunivoca: se le due curve si intersecano una volta la mappa ricopre la sfera due volte: essa conta l'area delle immagini della mappa di Gauss divisa per $4\pi$, cioé quante volte la mappa ricopre la sfera in base a quante volte si intersecano. Inoltre $n$ è sempre un \textit{numero intero} e se le curve non si intersecano è nullo. Le considerazioni appena fatte provengono dalla teoria dei nodi. \\
Abbiamo così dedotto la legge di Ampère dal caso generale.


%CIT: ci sono domande a cui io sappia rispondere su questo?

Adesso useremo la legge di Ampére per ricavare il campo magnetico, esattamente come abbiamo ricavato il potenziale di Coulomb con la legge di Gauss.
\begin{examplewt}[Filo rettilineo infinito]
%TODO: inserire immagine 16.5
	Per simmetria cilindrica il campo magnetico dipende solo dal raggio $R$ ed è diretto lungo $\vbh{u}_\phi$ che indica lo spostamento angolare, quindi $\vba B=B(R)\vbh{u}_\phi$.\\
	La circuitazione è $\mu_0 i$. Prendo una superficie $\gamma$ per portare fuori il campo magnetico dall'integrale, come per la legge di Gauss per non calcolare il flusso. Scelgo una superficie per cui l'integrale è costante: una circonferenza di raggio $R$, in questo caso $d\vba s=Rd\phi$. Siccome l'integrale è in $d\phi$ ma $\vba B$ dipende solo da $R$, ottengo
	\begin{gather*}
		\oint\vba B\vdot d\vba s= \mu_0 i \implies \vba B(R)R\int^{2\pi}_0 d\phi=\mu_0 i \implies B=\frac{\mu_0 i}{2\pi R}
	\end{gather*}
	Abbiamo così derivato la legge di Biot- Savart. 
\end{examplewt}
In questo caso è stato così semplice perché ho una simmetria particolare dietro, in generale però non è vero. Infatti conoscere la circuitazione o il rotore di solito non è sufficiente per trovare il campo vettoriale. Vediamo un altro esempio.

\begin{examplewt}[Solenoide infinito]
	Per simmetria sappiamo che il solenoide infinito produce un campo magnetico all'interno del solenoide e nessun campo magnetico all'esterno del solenoide stesso. Proviamo a ricavarlo di nuovo usando una particolare curva.\\
%TODO: inserire immagine 16.6
	Prendo una curva $\gamma$ rettangolare e calcolo la circuitazione lungo essa. Il campo magnetico potrebbe dipendere dal raggio ed è diretto lungo $\vbh{u}_z$, con l'asse z diretto verso l'alto, quindi $\vba B=B(R)\vbh{u}_z$.\\
	La circuitazione lungo la curva $\gamma$ sarà
		\begin{gather*}
			\oint_\gamma \vba B\vdot d\vba s=\left( \int^B_A + \int_B^C + \int^D_C + \int^A_D  \right) \vba B\vdot d\vba s= \int^B_A \vba B\vdot s=\\
			=\int^{z_2}_{z_1} B(R)dz=B(R)(z_2-z_1)
		\end{gather*}
	Questo perché l'unico tratto che contribuisce è l'integrale fra $A$ e $B$, infatti i tratti $\overline{BC}$ e $\overline{AD}$ sono ortogonali al campo magnetico, invece lungo il tratto $\overline{CD}$ è nullo perché il campo magnetico è esterno al solenoide. Inoltre siccome stiamo integrando lungo la verticale si ha $\vbh{u}_z\vdot ds=dz$.\\
	Per la legge di Ampère $B(R)(z_2-z_1)=\mu_0 i_\gamma$, dove $i$ è la corrente che attraversa la spira, mentre $i_\gamma$ è la corrente che attraversa al curva $\gamma$ ed è data dal numero di spire contenute nella curva. Avevamo definito il numero di spire $N$ come l'integrale della densità lineare di spire $n$ (supposta costante), mentre la corrente che interseca $\gamma$ è data dall'intensità di corrente nella spire per il numero di spire lì dentro:
		\begin{gather*}
			N=\int^{z_2}_{z_1}ndz = n(z_2-z_1) \implies i_\gamma=Ni=n(z_2-z_1)i
		\end{gather*}
	La $i$ è la densità di corrente che sta girando nel solenoide, quindi di ogni singola spira. Ricavo quindi che $\vba B$ non dipende neanche da $R$, infatti:
		\begin{gather*}
			B(R)\cancel{(z_2-z_1)}=\mu_0 i_\gamma= \mu_0n\cancel{(z_2-z_1)}i \implies B=\mu_0 i n
		\end{gather*}
%DOUBT: mettiamo il recap o no? è molto sintetico ma potrebbe confondere
	Ricapitolando, il rettangolo è stato scelto in modo accurato, infatti solo un lato parallelo al campo magnetico contribuisce all'integrale e lungo $z$ si ha $\vba B$ costante. Consideriamo poi la densità di corrente data dal numero di spire, ed ogni spira ha intensità di corrente $i$. Infine uguagliamo alla circuitazione e ritroviamo il campo magnetico.	
\end{examplewt}

\begin{observe}
Gli esempi appena visti sono l'equivalente in elettrostatica del trovare il campo elettrico in condizioni di simmetria grazie alla legge di Gauss: 
	\begin{align*}
		\text{sfera con densità di carica uniforme} & \leftrightsquigarrow  \text{filo percorso da corrente}\\
		\text{solenoide infinito} & \leftrightsquigarrow  \text{piastre}
	\end{align*}
\end{observe}
%TODO: verificare esempi elettrostatica, sia per denominazione sia per reference

\subsection{Elettrostatica e magnetostatica a confronto}
L'elettrostatica e la magnetostatica si riassumono in 4 equazioni.
%DOUBT: mettiamo un array? alla lavagna è una tabella
\begin{gather*}
	\begin{array}{c|c}
		\div{\vba E}=\frac{\rho}{\epsilon_0} & \div{\vba B}=0\\
		\\
		\hline
		\\
		\curl{\vba E}=0 & \curl{\vba B}=\mu_0\vba j
	\end{array}
\end{gather*}
%TODO: bisogna migliorare la spaziatura, senza spazi \\ è tutto appiccicato
%	possibile alternativa: multicol
Notiamo che le equazioni sono ancora indipendenti, questo perché siamo nel caso statico e stazionario in cui i campi non dipendono dal tempo.


Siccome il gradiente del campo elettrico è nullo, esiste un campo scalare $V$ tale che $\vba E=-\grad V$: sostituendo nella prima equazione ottengo la legge di Poisson \index{Poisson!equazione} \index{equazione! di Poisson} 
\begin{equation*}
	\laplacian V=-\frac{\rho}{\epsilon_0}
\end{equation*}


Siccome la divergenza di B è nulla, esiste un potenziale vettore $\vba A$ tale che $\vba B=\curl{\vba A}$. Sostituendo questa equazione nella prima e sviluppando otteniamo
\begin{gather*}
	\curl\left(\curl\vba A\right)=\mu_0\vba j\\
	\grad \left( \div{\vba A}\right) - \laplacian{\vba A}=\mu_0 \vba j
\end{gather*}

%OLD
%Nella prima lezione %MANCA REFERENCE
%abbiamo visto che il rotore del rotore è ...
%Dobbiamo fare vedere che il potenziale $A$. Con $V$ era definito a meno di potenziale, per cui conta solo la differenza di potenziale: 
IL potenziale $V$ è sempre definito a meno di costanti, infatti quello che conta è la differenza di potenziale: in termini più rigorosi, la sua classe di equivalenza sarà $V\sim V+a$. \\
Nel caso del potenziale vettore $A$ c'è qualcosa in più: se definisco $\vba A=\vba A+\grad\oldphi$ allora questo non cambia il campo magnetico:
\begin{gather*}
	\vba B=\curl\vba A=\curl \left( \vba A+\grad\oldphi\right)= \curl \vba A + \cancel{\curl\grad\oldphi}=\curl\vba A		
\end{gather*} 
perché il rotore del gradiente è nullo. Quindi il potenziale vettore $A$ è definito a meno di una costante e a meno di un gradiente. Questo è un caso dell'\textit{invarianza di Gauge} \index{Gauge!invarianza di} \index{invarianza!di Gauge}.\\
%Tale invarianza ci permette di 
Analogamente al caso dell'elettrostatica, in cui fissando $V=0$ all'infinto possiamo determinare una costante, fissiamo l'invarianza di Gauge in modo tale che la divergenza di $\vba A$ sia nulla, cioé $\div \vba A=0$. 
%OLD 
%Posso fissare l'invarianza (scelta rappresentante nella classe di equivalenza) in modo che la divergenza di $\vba A$ sia nulla, cioé $\div \vba A=0$. 
Se consideriamo un campo vettoriale $\vba A$ la cui divergenza non è nulla, lo mappiamo sotto la trasformazione $\vba A\sim \vba A+\grad\varphi$. Usiamo $\vba A'$ con $\div \vba A'=0$ e lo scriviamo come
\begin{gather*}
	\div\vba A+\div\grad\oldphi=0 \implies \laplacian\oldphi= -\div\vba A
\end{gather*}
Ovvero scegliamo lo scalare $\oldphi$ in modo tale che il suo laplaciano sia come sopra. In questo modo$\oldphi$ sarà determinato da un'equazione differenziale del secondo ordine ed è detta \textit{scelta di Gauge} \index{Gauge!scelta di}.\\
%OLD
%faccio la trasformazione $\vba A'$ per avere
%è detta scelta di Cage (?).
%
%con questa scelta mando a zero $\grad(\div A)$.

Abbiamo così ottenuto che sia l'elettrostatica sia la magnetostatica sono soluzioni dell'equazione di Poisson, infatti componente per componente i laplaciani di $\vba A$ sono 
\begin{gather*}
	\laplacian\vba A=-\mu_0\vba j\\
	\laplacian A_x=\mu_0 j_x\\
	\laplacian A_y=\mu_0 j_y\\
	\laplacian A_z=\mu_0 j_z\\
	\hline
	\laplacian V=-\frac{\rho}{\epsilon_0}
\end{gather*}

Così si riduce tutto a 4 equazioni di Poisson per i potenziali: una con il potenziale scalare per l'elettrostatica, e tre per la magnetostatica. Data una certa configurazione di $\rho$ e $\vba j$, cioé intensità di carica e corrente posso determinare i potenziali e da essi i campi elettrici e magnetici.\\

Possiamo far vedere che la soluzione più generale, date condizioni al contorno fisiche, cioé $V$ e $\vba A$ vanno a $0$ ad infinito, allora la soluzione di equazioni differenziali è data da
\begin{gather*}
	\vba A(\vba r)=\frac{\mu_0}{4\pi}\int\frac{\vba j(\vba r)}{\abs{\vba r-\vba r'}} dV' \text{ dove } dV'=dx'dy'dz'\\
	V(\vba r)=\frac{1}{4\pi\epsilon_0}\int \frac{\rho(\vba r')}{\abs{\vba r-\vba r'}}dV'
\end{gather*}
%OLD
%La soluzione che trovo sono 
%Controllo con il laplaciano di V(r) e dovrei trovare $\frac{\rho}{\epsilon_0}$.
%TODO: va aggiustata la formattazione delle formule, oltre agli environment

Vediamo ora nella pratica come usare queste soluzioni con delle situazioni che abbiamo già incontrato
\begin{examplewt}[Sfera carica uniformemente]
	Ritroviamo il campo elettrico sia interno sia esterno di una sfera di raggio $R$ carica uniformemente con $\vba\rho(\vba r)=\begin{cases}
		\rho & r<r_0\\
		0 & r>r_0
	\end{cases}$. 
	Ci mettiamo in coordinate sferiche$\begin{cases}
		x=r\cos\phi\sin\theta\\
		y=r\sin\phi\sin\theta\\
		z=r\cos\theta
	\end{cases}$
	e dopo qualche considerazione calcoliamo l'integrale $V(r)$ dato dall'equazione di Poisson:
	\begin{itemize}
		\item Calcoliamo in anticipo $dV'=(r')^2\sin\theta 'dr'd\theta 'd\phi '$
		\item Se scegliamo il punto in maniera furba abbiamo $|\vba r-\vba r'|=\sqrt{r^2+(r')^2-2rr'\cos\theta '}$.
		\item L'integrale sarebbe definito fra $0$ e $+\infty$, ma siccome $\rho$ è definito solo all'interno della sfera ci limitiamo a $r_0$
		\item $\rho$ è costante quindi lo portiamo via dall'integrale
	\end{itemize}
%NOTA: l'itemize è una mia idea per riassumere tutte le varie considerazioni, in un testo unico sarebbe stato un po' dispersivo e sembrava un essay in cui si cercano 100 sinonimi di since
	Otteniamo così:
	\begin{gather*}
		V(r)=\frac{\rho}{4\pi\epsilon_0}\int^{r_0}_0dr' \int^\pi_0 d\theta' \int^{2\pi}_0 d\phi' \frac{(r')^2 \sin\theta '}{\sqrt{r^2+(r')^2 -2rr'\cos\theta '}}\squarequal	
	\end{gather*}
	Per i passaggi successivi invece
	\begin{enumerate}
		\item operiamo un cambio di variabile $\begin{cases}
		u=\cos\theta'\\
		du=-\sin\theta'
	\end{cases}$
		\item l'integrale $\int^{2\pi}_0 d\phi'=2\pi$ perché $V$ non dipende da $\phi'$.
		\item siccome $y=\cos\theta'$ cambiamo i segni di integrazione spostando l'intervallo di integrazione su $[-1,1]$%OLD lo mangia e va in $y$. 
		\item prestiamo attenzione alla derivata dell'argomento: %OLD facendo la derivata della radice abbiamo 1/2rad per derivata dell'argomento rispetto a y, 
		semplifichiamo il 2, $r$ esce dall'integrale, $r'$ si semplifica ed otteniamo $\frac{\rho}{2\epsilon_0 r}$. %OLD Valutiamo $r+r'$ in $1$ e in $-1$ 
		\item il modulo di numeri positivi (i raggi $r+r'$) è positivo, quindi togliamo il modulo. 
	\end{enumerate}	
	\begin{gather*}
		\stackrel{(1,2,3)}{\squarequal}\frac{\rho}{2\epsilon_0}\int^{r_0}_0 \negthickspace dr'\int^1_{-1}\negthickspace du \frac{-(r')^2}{\sqrt{r^2+(r')^2-2rr'u}} \stackrel{(4)}{=}
		\frac{\rho}{2\epsilon_0}\int^{r_0}_0 \negthickspace dr' \eval{\frac{-(r')^2}{rr'} \sqrt{r^2+(r')^2-2rr'u}}^1_{-1}=\\
		=\frac{\rho}{2\epsilon_0r}\int^{r_0}_0 dr' (-r')\left( \abs{r-r'}-\abs{r+r'} \right)
		\stackrel{(5)}{=}\frac{\rho}{2\epsilon_0r}\int^{r_0}_0 dr' r'\left( r+r' + \abs{r-r'} \right)		
	\end{gather*}
%NOTA: soluzione artigianale del \negthickspace perché usciva fuori l'ultimo =
Ed è qui che distinguiamo il caso dentro la sfera e fuori dalla sfera:
	\begin{itemize}
		\item Fuori dalla sfera:  $r>r_0$.\\
		Siccome $r'$ è integrato fra $r$ e $r_0$, allora siamo fuori dal range di integrazione di $r'$, quindi consideriamo $r>r'$.\\
		Siccome $\rho$ è la carica sul volume della sfera, cioé $\rho=\frac{q}{V_S}=\frac{q}{\frac{4}{3} \pi r_0^2}$, sostituisco in $V$ e trovo il potenziale di Coulomb solito:
		\begin{gather*}
			V(r)=\frac{\rho}{\cancel 2\epsilon_0 r}\int^{r_0}_0 \cancel 2 (r')^2 dr'= \frac{q r_0^3}{3\epsilon_0 r}=\frac{qr_0^3}{\frac{4}{3}\pi r_0^3 r}=\frac{q}{4\pi \epsilon_0 r}
		\end{gather*}
		Il campo elettrico allora è:
		\begin{gather*}
			\vba E=-\grad V=\pdv{V}{r}\vbh{u}_r =\frac{q}{4\pi\epsilon_0 r^2}\vbh{u}_r		
		\end{gather*}
		\item Dentro la sfera: $r<r_0$.\\
		Facendo l'integrale dobbiamo spezzarlo fra $0$ ed $r$ e poi da $r$ a $r_0$, per poi uscire con segno positivo o negativo in base al modulo più grande: %OLD (maggiore  o minore di r su sui sto integrando)		
		\begin{gather*}
			V=\frac{\rho}{2\epsilon_0}\left( \int^r_0 dr'(2r')^2 + \int^{r_0}_r dr' 2rr' \right)= \frac{\rho}{2\epsilon_0 \cancel r} \left( \frac{2}{3} r^3 + \cancel r (r_0^2 -r^2) \right)=\frac{\rho}{2\epsilon_0}\left( r_0^2 -\frac{r^2}{3} \right)\\
			\vba E=-\grad V= -\pdv{V}{r}\vbh{u}_r =\frac{\rho r}{3\epsilon_0}\vbh{u}_r
		\end{gather*}
	\end{itemize}
	Notiamo che il potenziale è continuo per $r=r_0$, inoltre coincide anche la loro derivata rispetto a $r$
%NOTA: forse l'itemize non è la scelta migliore, per ora ho lasciato questa per distinguere meglio i casi
	%GRAFICI 16.8 e 16.9
	
	Potenziale:	In $R$ è continua ed è continua anche la derivata prima: non ho cuspidi. 
	$V(r)=\begin{cases}
		\frac{\rho}{2\epsilon_0}(r_0^2 -\frac{r^2}{3}) & r<r_0\\
		\frac{q}{4\pi\epsilon_0 r} &r>r_0
	\end{cases}$\\
	Il campo elettrico invece è lineare fino ad $R$ e poi decade come $1/r^2$
	$\vba E(r)=\begin{cases}
		\frac{\rho r}{3\epsilon_0} & r<r_0\\
		\frac{q}{4\pi\epsilon_0 r^2} & r>r_0
	\end{cases}$
	Siamo riusciti a fare l'integrale perché la configurazione è simmetrica. Abbiamo così ricavato un risultato già noto partendo dall'equazione di Poisson.
\end{examplewt}
%NOTA: non avendo i grafici ho lasciato questa non impaginazione, poi andranno fatti i riquadri



\section{Campi elettrici e magnetici variabili nel tempo}
Lo studio dei campi elettrici e magnetici variabili nel tempo parte da sperimentazioni di Faraday ed Henry sul fenomeno dell'induzione elettromagnetica. Maxwell lavorando a livello teorico sulla consistenza delle equazioni ha scoperto che un campo elettrico e variabile nel tempo produce un campo magnetico.\\
Esperimenti da cui possiamo dedurre l'induzione elettromagnetica:

\begin{itemize}
	\item Consideriamo una spira con strumento che misura la corrente che sta passando nel circuito. Prendiamo un magnete qualunque che produca un campo magnetico di dipolo (come sempre). Avviciniamo il magnete alla spira con una certa velocità $\vba v$. Lo facciamo entrare nella spira circolare. Lo strumento a sinistra misura la corrente diretta in una data direzione. Nel circuito gira corrente in assenza di generatori.\\
%TODO: immagine 16.10
	Riproduciamo lo stesso esperimento ma allontanando il magnete (cambiamo il verso della velocità, ma il campo magnetico è lo stesso). Notiamo che la corrente che circola è diretta nel verso opposto.\\
	Abbiamo così una produzione di differenza di potenziale (o forza elettromotrice )che produce un campo elettrico.
%TODO: immagine 16.11
	\item Consideriamo una spira circolare a cui colleghiamo un generatore. La avviciniamo ad una spira che è collegata a uno strumento che misura la corrente. La spira con generatore in movimento si comporta esattamente come un campo magnetico, quindi avvicinandola con una certa velocità $\vba v$ all'altra spira, lo strumento attaccato ad essa misura una corrente. Se la allontaniamo la corrente gira in senso opposto.
%TODO: immagine 16.12
	\item Esperimento di Faraday: dato un cilindretto di ferro su cui avvolgiamo tantissime spire (solenoide con l'anima in ferro), mettiamo un interruttore ed un generatore. Con l'interruttore aperto non abbiamo campo magnetico. Mettiamo una spira circolare di fianco a cui attacchiamo un circuito con il solito rilevatore. Ad interruttore aperto non c'è campo magnetico né corrente generata. Alla chiusura dell'interruttore si crea un campo magnetico che genera corrente segnalata dallo strumento in una certa direzione, che dura per qualche istante, e poi si ferma. Appena chiudiamo l'interruttore osserviamo una corrente. Quando il campo magnetico si stabilizza e non varia più nel tempo allora non osserviamo più corrente.\\
	Riaprendo l'interruttore non abbiam più campo magnetico, ma per un istante osserviamo una corrente diretta nella direzione opposta. Siccome il campo magnetico va dal massimo a $0$, abbiamo una variazione dello stesso nel tempo e quindi abbiamo generato corrente.
%TODO: immagine 16.13
\end{itemize}

Da queste osservazioni sperimentali si deduce la \textit{legge di Faraday}. \index{Faraday!legge di} \index{legge!di Faraday}
\begin{define}[Legge di Faraday]
	La forza elettromotrice generata (detta \textit{indotta}, perché è indotta dal campo magnetico) è 
	\begin{equation*}
		\mathcal{E}_i=-\dv{\Phi_\Sigma (\vba B)}{t}
	\end{equation*}
	Cioé la forza elettromotrice è prodotta da un campo (anzi di più flusso) magnetico variabile nel tempo. Avvicinando la spira invece del campo magnetico cambia il flusso e si produce corrente. Quello che misuro negli esperimenti è l'intensità di corrente, non la forza elettromotrice. Ma grazie alle leggi di Ohm ricavo la $\mathcal{E}$ per un circuito con resistenza $R$:
	\begin{equation*}
		i=-\frac{1}{R}\dv{\Phi_\Sigma(\vba B)}{t}
	\end{equation*}
\end{define}

\begin{observe}
	Una forza elettromotrice è prodotta da un campo elettrico \textit{non} conservativo, in particolare è proprio la definizione di $\mathcal{E}$ come circuitazione del campo elettrico, possiamo così riscrivere la legge di Faraday come
	\begin{equation*}
		\mathcal{E}=\oint\vba E\vdot d\vba s \implies \oint \vba E_i\vdot d\vba s =-\dv{\Phi(\vba B)}{t}		
	\end{equation*}
	Quella del campo elettrostatico però è nulla: infatti sono i contributi non conservativi a produrre la forza elettromotrice.\\
	Ad esempio la pila di Volta mantiene separazione di carica costante perché l'energia chimica (esterna che entra nel sistema) è convertita in elettrica. %OLD Effetto Haul (?)
\end{observe}

\begin{define}[Legge di Lenz]
	La forza elettromotrice indotta è tale da opporsi alla causa che l'ha generata. 
\end{define}
\begin{observe}
	Pensando all'esperimento delle spire, se una si avvicina all'altra con velocità $\vba v$, il flusso del campo magnetico aumenta perché più linee di forza entrano nell'altra spira. La forza elettromotrice quindi si oppone alla variazione del flusso del campo magnetico ed è questo il motivo del segno $-$ nella legge di Faraday.
\end{observe}
%TODO: inserire immagine 16.14

Notiamo che la legge è formulata in modo tale che sia la variazione del \textit{flusso} del campo magnetico a generare $\mathcal{E}$. Come facciamo a generare una forza elettromotrice indotta? Ricordiamo che il flusso è dato da:
\begin{equation*}
	\Phi_\Sigma(\vba B)=\int_\Sigma \vba B\vdot \vbh{u}_n d\Sigma	
\end{equation*}
Cosa possiamo cambiare nel tempo? 
\begin{itemize}
	\item $\vba B$ nel tempo
	\item l'angolo della spira rispetto al campo magnetico, questo perché abbiamo un prodotto scalare: muovere una spira all'interno di un campo magnetico non uniforme
	\item la superficie: cambia l'area
\end{itemize}


Vedremo che la seconda e la terza causa sono effetti della forza di Lorentz. Quindi solo la variazione di $\vba B$ nel tempo  non è prevista dalle leggi usate finora.\\


%%%%%%%%%%%%%%%%%%%%%%%%%%%%%%%%%%%%%%%%%%%%%%%%%%%%%%%%%%%%%%%%%%%%%%%%%%%%%%%%
%LEZ 17 30/03/2022
%NOTA: sta riprendendo quanto fatto nella lezione precedente, lascio entrambe le formulazioni per poi scegliere la migliore
Ricordiamo la legge di Faraday: esiste una forza elettromotrice $\mathcal{E}$ indotta da una variazione del flusso magnetico, ad esso proporzionale attraverso un circuito. Essa esisterebbe anche senza un circuito, ma per vederla serve metterne uno.\\

%DOUBT: adottiamo fem o f.e.m. come notazione invece di scrivere 100 volte forza elettromotrice?
Motivi che potrebbero portare alla generazione della f.e.m. $\mathcal{E}$ sono:
\begin{itemize}
	\item variazione forma circuito, che cambia la superficie $\Sigma$
	\item variazione dell'angolo tra $\vba B$ e $\Sigma$
	\item variazione di $\vba B$ nel tempo
	\item spostamento rigido del circuito
\end{itemize}
Andremo a vedere che nell'ultimo caso la f.e.m. non è un effetto nuovo: è una conseguenza della forza di Lorentz,%OLD: la sua esistenza dice che la fem deve essere prodotto
l'unico effetto veramente nuovo è la variazione di $\vba B$ nel tempo perché non è previsto dalle leggi che già conosciamo.\\
Questa dimostrazione ci permetterà di derivare in forma differenziale legge di Faraday.
\begin{demonstration}
%TODO: inserire immagine 17.1
	Immaginiamo di avere un circuito in un campo magnetico non uniforme. Immaginiamo che si sposti leggermente: all'istante $t+dt$ è in una posizione diversa da quella precedente con velocità $\vba v$. Il piccolo intervallo su cui si è spostato è $d\vba r=\vba vdt$.\\
	Sul filo ci sono delle cariche libere che subiscono una forza per il fatto di essere immerse in un campo magnetico $\vba B$. Ogni singola carica libera subisce la forza di Lorentz: $\vba F=e\vba v\cross\vba B $ e il campo elettrico indotto per ogni particella che sta lì sopra è $\vba E_i =\frac{\vba F}{e}=\vba v\cross\vba B$ e tende a far girare le cariche.\\
	Questo è simile all'effetto Haul(?),perché è un esempio di campo elettrico indotto da forza di Lorentz.\\
%TODO: da aggiustare motivazione seguente sui vari prodotti usati
	Se le cariche si stanno muovendo con velocità ortogonale al circuito	punto per punto guardiamo come sono diretti il campo magnetico e la velocità della spira	il campo elettrico generato fa girare intorno le cariche.\\
	La f.e.m. indotta è pari alla circuitazione del campo elettrico, ma non sarà conservativo, poi siccome stiamo assumendo $\vba B$ non uniforme ma costante nel tempo (esso dipende dalla posizione, le derivate sulla posizione non sono nulle).\\
	la derivata rispetto al tempo può uscire dall'integrale, ma $d\vba s$ è parallelo e rimane uguale, infatti è il vettore spostamento lungo la spira punto per punto tangente al circuito, invece	$d\vba r $ è ortogonale alla spira e indica lo spostamento che sta facendo il circuito nello spazio	
	\begin{gather*}
		\mathcal{E}_i=\Gamma_\gamma(\vba E_i)=\oint\vba E_i\vdot d\vba s=\oint \vba v\cross\vba B d\vba s=\\
		=\oint d\vba s\cross\vba v\vba B=\oint d\vba s\cross\dv{\vba r}{t}\vdot\vba B= \dv{t}\oint_\gamma d\vba s\cross d\vba r\vdot\vba B\squarequal
	\end{gather*}
%DOUBT: mettiamo un itemize come nella leione precedente visto che è una lista di considerazioni da fare prima/durante il gather?
	Adesso immaginiamo la superficie laterale: abbiamo una sorta di cilindro sbilenco dallo spostamento nel tempo del circuito.
%OLD dr=vdt	d\Sigma\vdot un
	Se consideriamo $d\vba s\cross d\vba r$, esso punta in direzione ortogonale alla superficie laterale ed integrandolo lungo tutta la spira otteniamo tutta la superficie laterale, possiamo infatti considerarlo come l'elemento infinitesimo di superficie $d\Sigma_l\vdot\vbh u_n=d\vba s\cross d\vba r$. Quindi data $\gamma$ curva bordo di una base.\\
	Tornando all'integrale 
	\begin{gather*}
		\dv{t}\oint_\gamma d\vba s\cross d\vba r\vdot \vba B=\dv{t}
	\end{gather*}
\end{demonstration}
%ricordiamo che l'area laterale del cilindro con base curva qualunque è data dall'integrale lungo la coordinata su cui costruisco il cilindro di ds vettoriale vettore spostamento lungo la direzione verticale in cui sto costruendo il cilindro e di modulo l'altezza del cilindro
%nel nostro caso dr è piccolo, quindi abbiamo d\Sigma_l
%
%
%la fem indotta è il flusso del campo magnetico attraverso la superficie laterale infinitesima
%$\mathcal{E}_i=\dv{t}\Phi_{\Sigma_i}(\vba B)$ flusso tagliato
%Recap: tutte le cariche subiscono un campo elettrico. per fem devo integrare sul circuito. riesco a portare avanti derivata rispetto al tempo. è lungo sup ds\crossdr che è la sup laterale infintesima del cilindretto
%non è ancora quello che vogliamo, dobbiamo far vedere che la differenza di flusso fra la spira nella posizione 1 e 2, con $\Sigma_1$ sup del circuito che finisce sul circuito prima dello spostamento, 2 è dopo lo spostamento: sono le superfici di base prima e dopo lo spostamento
%$\Delta\Phi=\Phi_{\Sigma_2}(\vba B) - \Phi_{\Sigma_1}(\vba B)$
%so che il flusso attraverso una superficie chiusa del campo magnetico è nullo $\Phi_\Sigma=0$
%	siccome $\vbh{u}_n punta fuori dalla superficie chiusa /lo abbiamo già visto altre volte
%$....$
%la forza di Lorentz per uno spostamento rigido in un campo magnetico uniforme ma non costante produce un \textit{flusso tagliato}
%cosa simile a spira immersa in campo magnetico non uniforme
%siccome deve essere nullo allora
%flusso tagliato $..=-\dv{t}\Phi_{\Sigma}(\vba B)$ con $\Sigma$ superficie che finisce sul circuito
%quando calcolo la differenza
%aggiungo limite		abbiamo calcolato che è il flusso attraverso la sup laterale e sostituisco
%
%ed è quello a cui volevamo arrivare
%
%Recap: abbiamo dimosrato che per uno spostamento rigido la forza elettrmotrice indotta $\mathcal{E}$ è una conseguenza della forza di Lorentz. La stessa cosa vale per variazioni della forma o dell'angolo
%per spiegare la fem indotta quando la spira si muove o cambia angolo o cambia forma non servono leggi nuove perché sono effetti della forza di Lorentz
%quello che non è previsto è la fem per effetto di variazione del campo magnetico $\vba B$, cioé $\dvp{\vba B}{t}\rightarrow \vba E_i$
%per spiegarlo vanno modificate le nostre leggi
%
%
%riscriviamo la forza elettromotrice indotta
%$\mathcal{E}= -\dv{t}\Phi_\Sigma(\vba B)=\int\vba E_id\vba s=-\dv\int$
%con curva \gamma su cui il campo elettrico ha circuitazione non nulla, \Sigma che finisce sulla curva \gamma e abbiamo certo campo magnetico che attraversa \Sigma e produce flusso
%campo magnetico che cambia nel tempo, la derivata che ci interessa è quella che agisce sul campo magnetico, non forma: consideriamo circuito fisso e non deformato
%$=\int_\Sigma \partial\vba B/$
%usiamo teorema del rotore sul membro di sinistra
%$\int_\Sigma\curl\vba E\vdot\vbh{u}_n d\Sigma=-$
%deve valere per ogni \Sigma, quindi per campo non conservativo è $\curl\vba E=-\partial/\partial$	modifica quando il campo magnetico dipendente dal tempo: produce campo elettrico non conservativo, fem indotta
%Questa è la 2/3 legge di Maxwell o forma differenziale della legge di Faraday
%%CIT: è la seconda o terza, dipende come contate
%è \textit{la} legge fondamentale
%tutti i fenomeni sperimentali che abbiamo visto derivano da questa, che è più specifica di quella integrale che contiene cose che derivano dalla legge di Lorentz
%Faraday combina Lorentz e Maxwell: tiene conto di fem da Lorentz e campo magnetico variabile rispetto al tempo
%
%La legge fondamentale è quella di Maxwell
%
%
%
%riscriviamo la legge anche per i potenziali
%una legge che non subisce variazioni è che $\div\vba B=0$, cioé $\vba B$ può essere scritto come il rotore di un campo $\vba A$   e campo elettrico -grad potenziale V
%Ora non è più vero, infatti $\curl\vba E=-\partial (\curl\vba A)=$
%scambio le derivate
%$\grad(\vba E+\partial \vba A)=0$
%è questo che si può scrivere come $-\grad V$
%quindi $\vba E=-\partial \vba A -\grad V$
%nel caso non statico ho un termine in più: derviata di $\vba A$ rispetto al tempo
%riottengo i campi dai potenziali con $\vba E=...$ e $\vba B=...$
%resta che li posso determinare univocamente ma per $\vba E$ devo coinvolgere anche la derivata
%vedremo che sarà più elegante con la relatività ristretta
%
%
%questa è l'elettrodinamica, che mischia elettricità e magnetismo, che si influenzano a vicenda, un quadro generale si vedrà nella relatività ristretta invariante per trasformazioni di Lorentz	tutto contenuto in  potenziali vettore $\vba $ e potenziale scalare $V$
%
%
%
%Applicativo
%\section{induttanza}
%Autoflusso
%Sappiamo un circuito, spira in cui circola corrente produce campo magnetico di dipolo
%%IMMAGINE
%il capo magneitco prodotto dalla spira produce un flusso non nullo attraverso la spira stessa
%è detto autoflusso
%$\Phi_\Sigma(\vba B)=\int\vba B\vdot$
%ma B è il campo magnetico stesso generato dalla spira, dato dalla 1 legge elementare di Laplace: $\vba B=\frac{}{}\ointd\vba s\cross\vbh{u}_r\r^2$
%$=\frac{}{}\int_\Sigma\oint_\gamma\frac{}{}$
%mi interessa che questi integrali dipendono esclusivamente dalla geometria, l'unica cosa che non dipende dalla geometria è l'intensità di corrente $i$ che posso regolare
%la quantità $L=\frac{}{}\int\oint$ dipende solo dalla forma del circuito perché il campo magnetico è autogenerato dalla spira stessa
%$L$ può dipendere eventualmente dal materiale contenuto nella spira /magnetismo nella materia
%l'autoflusso sarà quindi $\Phi_\Sigma(\vba B)=Li$ con $L$ detta \textit{induttanza}
%
%l'autolusso di un qualsiasi circuito è proporzionale all'intensità di corrente con costate di proporzionalità dipendente solo dalla geometria
%è molto simile alla situazione del condensatore: differenza di potenziale, capacità dipendente solo da forma
%
%\begin{examplewt}[Solenoide (non) infinto]
%	si usano i solenoidi per creare l'induttanza
%	sappiamo che il campo magnetico in un solenoide infinto è $B=\mu_0 in$ con $n$ dentià lineare di spire
%	prendo il solenoide di una lunghezza $d$ molto maggiore del diametro $R$ ($d>>R$) per avere l'approssimazione che sia infinto trascurando gli effetti di bordo
%	siccome il campo magnetico è costante il flusso attraverso \Sigma che taglia il solenoide è dato dal flusso sulla singola spira di superficie \Sigma sup della spira per il numero totale di spire N
%	\Phi(B)=\Sigma BN=
%	per un solenoide circolare \Sigma=\pi R^2
%	=\mu_0i
%	quindi L=\mu_0 n^2\Sigma d	
%	posso farla per unità di lunghezza dividendo per (?)
%	
%	un po' come il caso delle piastre del condensatore: per produrre campo elettrico costante usavamo due piastre e trascuravamo gli effetti di bordo, ma qui abbiamo il solenoide
%		
%\end{examplewt}
%
%Unità di misura dell'induttanza
%lo leggo dalla definizione 
%%TODO: mettere unità di misura, environment?
%H di Henry
%%CIT: come al solito l'ordine di grandezza di Henry è cannato
%\begin{example}[Ordine di grandezza dell'Henry]
%	Solenoide rettilineo $n=10^3$ spire al metro, superficie $\Sigma=100$ cm^2
%	induttanza per unità di lunghezza $L=4\pi 10^{-2}$
%	
%	in laboratorio riempiamo di materiale ferromagnetico, come il ferro che aumenta notevolemnte il campo magnetico, poi il flusso e di conseguenza l'induttanza
%	
%\end{example}
%
%L'induttanza è importante per l'autoinduzione
%
%Autoinduzione
%ogni volta che c'è una variazione di flusso del campo magnetico attraverso sup \Sigma ho fem $\mathca{E}_i=$
%posso anche avere variazione dell'auoflusso, ed è detta autoinduzione
%siccome $\Phi_\Sigma(|vba B)=Li$
%sostituisco sopra
%$=d/dt=$
%$\mathcal{E}_i=-L$
%ogni corrente che varia nel tempo ha fem che fa cambiare la corrente stessa
%%CIT c'era un mio prof che diceva un loopo che si morde la coda
%
%
%
%
%applicazione induttanza in laboratori
%Circuito RL
%$\mathcal{E}_i=-L\dd$
%avevamo visto i cricuiti RC, che avrann delle similitudini
%Fisicamente prendiamo il circuito aperto e chiuderlo, generatore che produce fem, cariche cominciano a girare, circolano nel solenoide, che produce B che produce variazione di flusso, che tende ad opportsi a quello che l'ha generato, ho fem nella dierzione opposta /effetto legge di Lentz/
%L si oppone alla circolazione di corrente, quindi ci impiega un po' ad arrivare a regime
%è l'opposto di quello che succedeva nei condensatori: la corrente tende a partire più lentamente a causa dell'induttanza, che agisce come una sorta di inerzia
%%CIT l'induttanza tende a non far cambiare le cose, come il Gattopardo
%$\mathcal{E}+\mathcal{E}_i=Ri$
%$\mathca{E}-L\dd=RI$
%ritroviamo equazione differnziale come nel caso del condensatore
%risolviamola
%$\mathca{E}-Ri=L\dd\\
%di/\mathca{E}-Ri=dt/L$\\
%integriamo da t=0 e fino al tempo t
%$\int_0^{i(t)}=\int_0^t\\
%-1/R\log=\\
%=\mathcal{E}e^{\frac{}{L}}\\
%\implies i(t)=\frac{\mathcal{E}}{R}\left( 1-e^{} \right)$
%
%in termini di dimensioni L/R=s
%R/L=\tau
%%grafici
%all'inizio l'intensità di corrente è forte, la fem che ho è grande all'inizio
%man mano che la corrente cresce tende ad arrivare a valore costante, derivata tende a 0 e fem tende a 0
%la fem indotta $\mathcal{E}_i=L+\frac{R}{LR}$\mathcal{E}e^{-\frac{}{}}$
%
%nei circuiti RC faceva l'opposto,sia per intensità di corrente sia per differenza di potenziale
%RL \tau=RC	si oppone subito
%RC \tau=R/L	 si oppone asintoticamente
%
%i(t)=\frac{}{}()
%è etta \textit{extracorrente di chiusura}
%
%sarà più interessante nel caso della corrente alternata, circuiti RLC
%
%
%Considerazioni energetiche del circuito RL
%per i condensatori c'è il bilancio energetico: il generatore ha certa potenza, energia dissipata nella resisteza, nel condensatore parte dell'energia era immagazzinata come diff en potenziale fra le due piastre $V_C=\frac{1}{2}\frac{q^2}{C}$ e da qui avevamo dedotto una densità di energia di campo elettrico, immagazzinata fra le due piastre $u_E=\frac{}{}\epsiloon$
%in un volume $V$ è $U_C=\int_V  dV$
%dal caso specififoc del condensatore avevamo dedotto energia associata a campo elettrico, vale in gerneale la def di densità di energia
%
%equivalente per il caso degli "inuttori"
%la potenza generata dal generatoe $P=Vi$,qunidi $P_\mathcal{E}=\mathcal{E}i$ è la potenza erogata dal generatore
%potenza dissipata dalla resistenza $P_R=i^2R$
%potenza associata all'induttanza $V$ fem indotta
%$P_L=\mathcal{E}_ii=-Lidi/dt$ si accumula in modo proporzionale alla derivata dell'intensità di corrente
%$\mathcal{E}_i=Ri^2+Lidi/dt$
%da $\mathal{E}=Ri...$
%ottengo P_=P_+P_
%per energia immagazzinata nel'induttanza devo fare l'integrale ra 0 è \infty perché ho comportamento asintotico
%U_L=\int^{+\infty}_0P_Ldt=
%ma i_\infty=\frac{\mathcal}{R} corrente che circola a regime
%l'energia immagazzinata dall'induttore è $U_L=\frac{}{}Li_{\infty}^2$
%analogo ai condensatori
%
%per completare parallelismo con condensatore
%l'energia è accumulata nell'induttore, in cui c'è campo magnetico che si porta l'energia
%l'energia è associata l fatto che prima non c'era un campo magnetico e adesso c'è
%com'è immagazzinata? prendo solenoide infinito
%	L=\mu_0\Sigma n^2d
%inserisco nella formula di U_L
%U_L=\frac{1}{2}
%	ma B in un solenoide è \mu_0in
%	\Sigma d è voleume del solenoide
%=B^2/
%
%quindi per induttori 
%U_L=\frac{}{}Li^2
%densità di energia del campo magnetico $u_B=\frac{}{}\frac{B^2}{\mu_0}$
%U_B=\int_V
%
%è una cosa valida in generale: ogni volta che ho E eB si crea densità di energia u_E e u_B
%se ci sono entrambi allora si sommano
%energia del campo elettromagnetico U=\frac{}{2}\left( \epsilon_0 E^2 + \frac{B^2}{\mu_0} \right)
%vera per qualsiasi capo elettrico e magnetico
%si può anche derivare da formalismo hamiltoniano


%%%%%%%%%%%%%%%%%%%%%%%%%%%%%%%%%%%%%%%%%%%%%%%%%%%%%%%%%%%%%%%%%%%%%%%%%%%%%%%%%%%%%%%%%%%%%%%%%%%%%%%%%%%%%%%%%%%%%%%%%%%%%%
%LEZ 18 31/03/2022
abbiamo visto che campi varabili nel tempo producono altri campi: abbiamo visto che tramite l'analisi della legge di Faraday un campo magn variabile nel tempo produce fem e quidi campo el indotto
Maxwell si è accorto (per unire interazione elettrica e magnetica in una sola ) 
se n'è accorto con la \textit{corrente di spostamento}
come nome è misleading, ma il modo di accorgersene è semplice
partiamo dalla legge di Ampère, che abbiamo derivato nel caso statico 
andremo a derivare legge di Ampère-Maxwell
\oint\vba Bd\vba s=\mu_0 i
\curl\vba B=\mu_0\vba j
non è compatibile con la conservazione della carica: prendiamo la divergenza della seconda equazione: \grad\curl\vba B=\mu_0\grad\vba j
a sx è immediatamente 0, quindi è compatibile con densità di corrente elettrica solenoidale \div\vba j=0
Quindi la legge di Ampère è consistente solo se la corrente \vba j è solenoidale, cioé ha divergenza nulla
in realtà vorremmo l'equazione di continuità: \div\vba j+\pdv{\rho}{t}=0	cioé se in un certo volume di spazio c'è carica, tutt quella che passa nella superficie esce e produce carica
non si crea nè si distrugge: se non è zero ho corrente elettrica, quindi flusso di corrente elettrica
se ho un volume con carica el, se esce allora la corrente j ha un flusso non nullo nella sup che determina il volume: modo locale di formulare la conservazione della carica

tale equazione ci suggerisce come correggere la legge di Ampère
sappiamo che \div\vba E=\frac{}{}\epsilon_0
se questa è vera allora possiamo scrivere \rho=\epsilon_0\div\vba E
\pdv{\rho}{t}=\epsilon_0
vorremmo quindi aggiungere al termine \div\vba j \pdv{\rho}{t}, ma abbimo appena vist che è pari a \epsilon_0\div\pdv{\vba E}{t}
Quindi aggiungo un pezzo tale che dopo la divergenza ho \pdv{\rho}{t}
\begin{equation}
	\curl\vba B=\mu_0 \vba j+\mu_0\epsilo_0\pdv{\vba E}{t}
\end{equation}
arriviamo quindi alla legge di Ampère Maxwell appena scritta
\grad\curl\vba B=\mu_0\div\vba j+\mu_0\epsilon_0\pdv{\div\vba E}{t}
per ottenere la forma integrale della legge dobbiamo scrivere i flussi attraverso una superficie \Sigma
ricordiamo che \mu_0\epsilon_0=\frac{}{c^2}
\int_\Sigma\curl\vba B\vdot\vbh{u}_nd\Sigma=\mu_0\int_\Sigma\vba j\vdot\vbh u_n d\Sigma +\frac{1}{c^2}\pdv{t}\int_\Sigma \vba E\vdot\vbh u_n d\Sigma
uso il teo rotore sul membro a sinistra
\oint_\Sigma \vbaB\vdot d\vba s= \mu_0 i +
è l'analoo della legge di Faraday: un cambiamento del fuusso del campo elettrico genera un contributo addizionale alla circuitazione
quindi il termine i_S=\epsilon_0\pdv{\Phi_\Sigma}{t} è detta \textit{corrente di spostamento}
allo stesso modo la densità di corrente j_S=\epsilon	è detta \textit{densità di corrente di spostamento}
è detta corrente di spostamento perché è un contributo pari a j, potrei combinarli e definire una j totale
\vba j_{TOT}=\vba j+\vba j_S
quindi \curl\vba B=\mu_0\vba j_{TOT}
combinando gli effetti è un campo solenoidale: 0=\div\vba j_{TOT}=\div\vba j+\div\vba j_S(=\pdv{\rho}{t})=
vediamo perché deve essere necessaria questa corrente: prendiamo un circuito RC: per la consistenza della legge di Ampere, qualsiasi corrente metta a destra deve essere la stessa (?)
Maxwell ha risolto i lproblema considerando due superfici aperte: \Sigma_1 bolla che entra fra le due piastre del condensatore e finisce sulla stessa curva \gamma
\Sigma_2 anche finisce su \gamma ma va solo sul filo. si ha \partial\Sigma_1=\partial\Sigma_2
accendo la corrente e circola, il flusso di j attraverso \Sigma_2=i: \int_{\Sigma_2}\vba j\vdot =i
ma fra le piastre del condenstaore non circola carica! \int=0
però sono 
se fosse solenoidale questi due contributi dovrebbero essere uguali: è un altro modo per accorgersi che la corrente \vba j da sola non è solenoidale
sembra che fra le piastre ci sia un'interruzione
in realtà ho il contributo addizionale. è stato detto che nelle piastre ho campo elettrico che sta variando, diventa più intenso man mano che si carica, se varia ho contributo non nulla dalla corrente di spostamento, quindi il flusso in \Sigma_1 non è 0 ma è pari alla corrente di spostamento
è ripristinato che la corrente totale: corrente+ corrente di spostamento sia solenoidale
il nome è dovuto a questo fatto: c'è un condensatore ma gira lo stesso corrente
il fenomeno vero è che la variazione del campo el produce campo mag
l'equazione fisica è quella di continuità: \div\vba j=\pdv{\rho}{t}
la corrente j deve compensare variazione di carica

perché non ci si era mai accorti di sta cosa: sperimentalmente il contributo aggiuntivo è molto difficile da derivare: \mu_0\epsilon_0=\frac{1}{c^2}, quindi il contributo è molto piccolo
è stato un grande successo della fisica teorica per avere consistenza interna delle equazioni


EPILOGO
siamo arrivati a formulare l'ultima legge mancante
ora possimao scrivere le equazioni di Maxwell
\section{Equazioni di Maxwell}
Le equazioni di Maxwell sono
\begin{itemize}
	\item LEgge di Gauss
	\item
	\item \div\vba B=0 /\vba B campo solenoidale per qualsiasi dipendenza dal tempo
	\item legge di Ampèere Maxwell \curl\vba B=\mu_0\vba j+ \frac{}{c^2}\pdv{\vba E}{t}
\end{itemize}
l'elettrodinamica classica nel vuoto è data da queste 4 eq

abbiamo come _conseguenza_ l'equazione di continuità:\div\vba j+\pdv{\rho}{t}
risolvendo possiamo trovare campo elettrico e magnetico, per trovare dinamica serve come agiscono i campi su una carica: serve forza di Lorentz: \vba F=q(\vba E+\vba v\cross\vba B)
densità di energia dei campi: U=\frac{}{2}(\epsilon_0 e^2+\frac{B^2}{\mu_0})

queste sono le eq fondamentali dell'elettrodinamica classica nel vuoto

abbiamo già visto molte proprietà
dobbiamo ancora vedere le proprietà delle eq. di Maxwell nel vuoto
la prima cosa che ci chiediamo è: coinvolgono dei campi e le sorgenti \rho e j (sono dat, esterne, sono gli input: date \rho e \vba j determino campo el e magn) se spegnessi le sorgenti quale sarebbe la soluzione nel vuoto?
per vuot intendiamo senza sorgenti
\div\vba E=0
\div\vba B=0
\curl\vba E=-\pdv{\vba B}{t}
\curl\vba B= \frac{1}{c^2}\pdv{\vba E}{t}
hanno una certa simmetria: abbiamo già studiato della soluzioni statiche: carica che si comporta come sfera carica
posso avere soluzioni meno statiche: uno dei due può dipendere dal tempo e si scambiano fra loro
sarà una conferma cruciale delle eq di Maxwell è che ci vediamo: una soluzione delle equazioni è che sono sol anche la luce, che è onda elettromagnetica

la sol generale è onda elettromagnetica che si propaga alla velocità della luce
abbiamo così capito che la luce è in realtà un'oscillazione di campi elettrici e magnetici: onda elmagn con data frequenza che siamo in grado di rilevare
lo vedremo dalla prossima settimana

\subsection{Potenziali}
guardando queste eq possiamo riscriverl con dei potenziali
per camp magn è semplice \vba=\curl\vba A
nel caso dinamico il rotore el non è nullo: dobbiamo aggiungere termine \vba E=-\pdv{\vba A}{t}-\grad V	prendo il rot ed è nullo: ritrovo 2 eq M
sono compatibili coneq
abbiamo detto che il potenziale vettore \vba A è definito solo a meno di gradiente: \vba A\sim\vba A+\grad\oldphi
posso traslaral di qun qualsiasi potenziale gradiente ed è lo stesso
ma se \oldphi dipende dal tempo non è più una simmetria del campo elettrico: \vba E\rightarrow -\pdv{\vba A}{t}-\grad\pdv{\oldphi}{t}-\grad V
non è più da sola una simmetria, ma abbiamo anche un suggerimento perché sia simmetrico: basta definire V'=\grad\pdv{\oldphi}{t}-\grad V
quindi definiamo la classe d'eq come V\simV-\pdv{\oldphi}{t}
coinvolgiamo 2 trasformazioni contermporanee per la classe d'equivalenza (vedi i \sim)
son dette invarianza di Gauge
vorrei usare le equ di M per ricavare i potenziali: esattamente come nel caso statico con le eq di Poisson
ora lo facciamo nel caso dinamico

possiamo usare l'invarianza per fissare \div\vba A +\frac{}{c^2}\pdv{V}{t}=0 è anche detta scelta di Gauge (modo fisico di dire scelta di un rappresentante in una classe di equivalenza)	(caso statico \div\vba A=0)
possiamo farlo grazie all'invarianza: se \div\vba A +\frac \pdv \neq 0
allora definaimo \vba A'=\vba A+\grad\oldphi
V'=V-\pdv{\oldphi}{t}
sostituiamo ed otteniamo
\div \vba A'-\laplacian\oldphi +\frac{}{c^2}\pdv[2]{\oldphi}{t}=D
fisso il valore del campo \oldphi in modo che \div \vba A'+\frac{}{c^2}\pdv{V'}{t}=0
\laplacian\oldphi-\frac{}{c^2}\pdv[2]{\oldphi}{t}=-D
definiamo operatore box, operatore dalambertiano (da D'Alambert)
\qed=\laplacian -\frac{}{}\pdv[2]{t}
quindi \qed\oldphi=-D

?
partiamo dal \curl\vba B=\grad(\curl\vba A)=
ma rotore del rotore è grad della divergenza - laplaciano
=\grad(\div\vba A)-\laplacian \vba A
ma \vba B=\curl\vba A e \vba E=-\pdv{\vba A}{t}-\grad V
quindi  scritta in termini di potenziali è
\curl\vba B-\frac{1}{c^2}\pdv{\vba E}{t}=\mu_0\vba j \implies \grad(\div\vba A)-\laplacian\vba A +\frac{}{} \pdv[2]{A}{t} +\frac{}{}\grad\pdv{V}{t}=\mu_0\vba j

racoclgo il gradiente (gradiente della somma) e vediamo che all'interno è nullo
\cancel{\grad(\div\vba A +\frac{1}{c^2}\pdv{V}{t})}-(\laplacian -\frac{}{}\pdv[2]{t})(\vba A)=\mu_0\vba j
(\vba A) significa che applico (?) in A
otteniamo così \qed\vba A=-\mu_0\vba j
generalizzazione al caso dipendente dal tempo delle eq di Poisson: sono delle eq di D'Alambert
rimpiazziamo laplaciano con dalambertiano
spoiler: dalambertiano generalizzazione lacplaciano allo spazio tempo


ricaviamo eq di V partendo dalla \div \vba E
\div\vba E=-\pdv{t}\div\vba A-\laplacian V=\frac{\rho}{\epsilon_0}
usiamo che \vba E=-\pdv{A}{t} -\grad V e \div\vba A+\frac{}{}\pdv{V}{t}=0
abbiamo così \frac{}{}\pdv[2]{V}{t} -\lplacian V=\frac{\rho}{\epsilon_0}
quindi \qed V=-\frac{\rho}{\epsilon_0}
equazioni del moto dell'elettrodinamica classica
eq diff di \vba A e V che date certe distribuzioni di carica \rho e di corrente \vba j posso riacavare poteziali e dalì campi


\subsection{Circuiti RLC}

studio dei circuiti RLC
una loro caratteristica: possiamo studiarli addirittura senza inserire un generatore
%IMMAGINE
ci si aspetta che non ci sia corrente
immaginiamo che al tempo t=0 il condensatore sia carico, quindi ho differenza di potenziale
chiudo l'interruttore le cariche si muovono per compensare diff pot, avrò corrente in una certa direzione: all'interno dell'induttanza ho variazione di corrente: vuole creare corrente opposta e rispedisce indietro le cariche
questo sistema dà origine ad un'oscillazione, simile a quella di una molla, la resistena smora le oscillazioni, quindi si perde l'intensità di corrente
vediamo le eq del circuito
diff po ai capi del condesatore V_C=\frac{q}{C}
indotta V_L=-L\dv{i}{t}
legge di Ohm V_C+V_L=Ri
quindi \frac{q}{C}-L\dv{i}{t}=Ri
ricordiamo che i=-\dv{q}{t}
prendimo una derivata rispetta al tempo dell'eq
-\frac{i}{C} -L\dv[2]{i}{t}=R\dv{i}{t}
\dv[2]{i}{t}+\frac{R}{L}\dv{i}{t} +\frac{i}{LC}=0
è un'equazione differenziale 2 grado che corrisponde ad oscillatore armonico smorzato
cominciamo dal caso R==, quindi circuito LC (realisticamente difficile da realizzare, dovrei avere superconduttori per resistenze molto piccole)
\dv[2]{i}{t} +\omega_0^2 i=0
con \omega_0=\frac{1}{\sqrt{LC}} detta \textit{frequenza caratteristica} del circuito LC
la soluzione dell'eq di 2 grado è una funzione sinusoidale i(t)=A\sin(\omega_0 t+\oldphi)
fase iniziale \oldphi da determinare, anche A
A e \oldphi sono costanti di integrazione da determinare
quindi l'intensità di corrente oscilla
sappiamo che V_C=L\dv{i}{t}
quindi V_C(t)=AL\omega_0\cos(\omega_0 t+\oldphi)
posso fissare delle condizioni iniziali:
se a t=0, allora V_C=V_0 e i(0)=0 e \dv{i}{t}(0)=\frac{V_0}{L}
imponendo queste condizioni iniziali viene \oldphi=0
i(0)=A\sin\oldphi
V_C(0)=AL\omega_0 \cos\oldphi=V_0
possiamo così fissare \oldphi=0
quindi A=\frac{V_0}{L\omega_0}
sostituendo
i(t)=\frac{V_0}{L\omega_0}\sin\omega_0 t
V_C(t)=V_0\cos\omega_0 t
quindi sono in quadratura di fase
%IMMAGINE
parto che la diff pot ai condensatore è massima e corrente è minima, poi si scambiano come quando sono alla pos di eq della molla ma ho una velocità
t/4 e t/2
tutto questo è dovut alla presenza dell'induttanza
il bilancio energetico è fra campo elettrico e magnetico (energia intrappolata nel campo magn nell'induttanza)
oscillazione di energia fra campo elettrico e magnetico E\rightarrow B\rightarrow E\rightarrow B
il moto così non è smorzato e andrebbe avanti all'infinto, esattamente come una molla in assenza di attrito
scriviamo il bilancio energetico
E_{TOT}=\frac{1}{2}CV_C^2 +\frac{1}{2}Li^2=\frac{1}{2}CV_0^2=\frac{1}{2}L\frac{V_0^2}{L^2\omega_0^2}
energie accumulate all'interno del condensatore e dell'induttanza
questa eq vale a qualsiasi istante di tempo
nell'istante iniziale t=0 ho solo l'energia del condensatore, che poi passa fra induttanza e condensatore
quello che succede in caso di presenza di una resistenza: essa fa sì che l'energia si disperda: le oscillazioni invece di andare avanti all'infinito smorzano
\dv[2]{i}{t}+\frac{R}{L}\dv{i}{t}+\frac{1}{LC}i=0
rispetto all'eq del moto armonico abbiamo un termine in più di smorzamento proporzionale alla velocità, di solito da liquido viscoso o aria
\dv[2]{x}{t}+2\gamma\dv{x}{t} +\omega^2x=0
ricordiamo che abbiamo ricavato l'eq della molla come m\dv[2]{x}{t}=-kx -\color{\mu\dv{x}{t}}
\dv[2]{x}{t}+\frac{k}{m}x= +\color{\frac{u}{m}\dv{x}{t}}=0
\color{2\gamma=\frac{\mu}{m}}
\gamma=\frac{R}{2L}
\omega_0^2=\rfac{1}{LC}
abbiamo quindi \dv[2]{i}{}+2\gamma\dv{i}{t}+\omega_0^2i=0
risolviamo l'eq con
sfrutto eq caratteristica dell'eq diff 2 ordine
i(t)=Ae^{i\alpha_1 t}+Be^{i\alpha_2t}
\alpha^2 +2\gamma\alpha +\omega_0^2=0
\alha_1=-\gamma+\sqrt{\gamma^2-\omega_0^2}
\alpha_2=-\gamma+\sqrt{\gamma^2+\omega_0^2}
distinguiamo i casi
\begin{enumerate}
	\item smorzamento forte \gamma^2>\omega_0^2, in termini di RLC si traduce come R^2>\frac{4L}{C}, quindi imponiamo come \textit{resistenza critica} R_C=2\sqrt{\frac{L}{C}}
	i(t)=Ae^{\gamma t+t\sqrt{\gamma^2-\omega_0^2}} + Be^{-\gamma t -t\sqrt{}}
	nella radice ho termine positiv, quniid minore di \gamma, contributo smorzante è quello che domina
	=e^{-\gamma t}\left( Ae^{-t\sqrt{\gamma^2-\omega_0^2}} + Be^{-t\sqrt{\gamma^2-\omega_0^2}}\right)
	A e B vanno fissate a seconda delle condizioni iniziali
	%IMMAGINE
	\item smorzamento critico R=R_C, \gamma^2=\omega_0^2
	le sol sono lin dip, quindi \alpha_1=\alpha_2=-\gamma
	i(t)=e^{-\gamma t}(A+Bt)
	se i(0)=0 allora i(t)=Bte^{-\gamma t}
	non ho ancora oscillazione perché non c'è fase
	\item smorzamento debole \gamma^2<\omega^2	essendo negativo ottengo fase + fase di segno opposto
	la sol si riscrive i(t)=De^{-\gamma t} \sin(\omega t+\oldphi)
	con \omega=\sqrt{\omega_0^2+\gamma^2} e D non termine di fase
	riscrivendo come sin e cos ottengo questa eq, è sol equivalente ma scritta in un altro modo
	ottengo così delle oscillazioni che poi si smorzano
	smorzamento con profilo di e^{-\gamma t} che è il fattore smorzante
	più è piccola a resistenza più riesce ad oscillare
	



\end{enumerate}


vedremo che se aggiungo generatore che dà onde sinusoidali posso compensare lo smorzamento ed ottenerne uno no smorzato


