% SVN info for this file
\svnidlong
{$HeadURL$}
{$LastChangedDate$}
{$LastChangedRevision$}
{$LastChangedBy$}

\chapter{Dielettrici}
\labelChapter{integraledilebesgue}

\begin{introduction}
	‘‘Si dovrebbe sempre generalizzare.''
	\begin{flushright}
		\textsc{Carl Jacobi,} prima che Teoria delle Categorie gli facesse cambiare idea.
	\end{flushright}
\end{introduction}
\lettrine[findent=1pt, nindent=0pt]{D}{opo}
\section{Materiale dielettrici e condensatori}
Consideriamo un \textit{condensatore} alle cui armature è collegato un \textit{elettroscopio}: anche lo abbiamo introdotto come strumento per misurare la carica, può essere usato (e qui lo useremo in questo secondo modo) anche per misurare la differenza di potenziale e/o il campo elettrico. Ricordiamo che, per il condensatore piano con armature distanti $d$ e densità di carica uniforme $\sigma$, il campo elettrico interno è
\begin{equation*}
	E_0=\frac{\sigma}{\epsilon_0}
\end{equation*}
e la differenza di potenziale è
\begin{equation*}
	V_0=\frac{\sigma d}{\epsilon_0}=E_0d
\end{equation*}
Se colleghiamo (in parallelo) l'elettroscopio, tale differenza di potenziale corrisponde ad un certo angolo di separazione delle foglioline d'oro. 
% TO DO: inserire immagine
\paragraph{Potenziale e capacità di un condensatore con conduttore all'interno}
Se introduciamo una \textit{lastra conduttrice} di spessore $s$ nello spazio tra le due piastre, osservando l'elettroscopio ci accorgiamo che l'angolo tra le foglioline è minore rispetto al caso precedente. Effettivamente avviene un calo di potenziale, ma perché?\\
Il campo elettrico del condensatore induce una separazione di carica nella lastra conduttrice, formando una distribuzione superficie di carica positiva da un lato e negativa dall'altra. Ciò induce un campo elettrico di verso opposto a quello già presente, in modo che all'interno della lastra \textit{non} ci sia campo elettrico, ma questo comporta una diminuzione del campo elettrico. In termini di potenziali, si noti che la $\ddp$
\begin{itemize}
	\item tra la prima piastra e il conduttore è $V_1=E_0d_1$.
	\item tra il conduttore e la seconda piastra è $V_2=E_0d_2$.
\end{itemize}
dove $d_i$ è la distanza tra la piastra $i$-esima e il conduttore; poiché $d_1+d_2=d-s$, la differenza di potenziale complessiva è
\begin{equation*}
	V_{C}=V_1+V_2=E_0\left(d-s\right)=V_0-V_{lastra}<V_0
\end{equation*}
In altre parole, la \ddp \textit{diminuisce} di un fattore $E_0s$. Se occupassi l'intera intercapedine con un materiale conduttore, è evidente che si avrebbe $V=0$: il condensatore diventerebbe un unico conduttore e le cariche si disporrebbero sulla superficie esterna.

Al contrario, la capacità \textit{aumenta}. Se chiamiamo la nuova capacità $C_C$, noto che $V_{C}=\dfrac{q_0}{C_C}$, si ha
\begin{equation*}
	V_C=\frac{q}{C_C}=E_0\left(d-s\right)=\frac{q}{\Sigma \epsilon_0}\left(d-s\right)\implies \frac{1}{C_C}=\frac{d-s}{\Sigma \epsilon_0}C_C=\frac{\Sigma \epsilon_0}{d-s}>\frac{\Sigma \epsilon_0}{d}=C_0
\end{equation*}
\begin{equation}
	C_C=\frac{\Sigma \epsilon_0}{d-s}>C_0
\end{equation}
\paragraph{Potenziale e capacità di un condensatore con isolante all'interno. Costante dielettrica relativa}
Ripetendo l'esperimento con una lastra di \textit{materiale isolante}\footnote{In realtà quello che affrontiamo in questo paragrafo è vero solo alcuni tipi di isolanti, i cosiddetti \textbf{dielettrici (lineari)}; nella sezione XXX approfondiremo la differenza tra i due.}, ci accorgiamo che la differenza di potenziale $V$ (e quindi il campo elettrico) è comunque \textit{minore} del caso col vuoto nell'intercapedine, ma tale \ddp è maggiore del caso con il materiale conduttore  - a parità di spessore.
\begin{equation*}
	V_{C}<V<V_0
\end{equation*} % TO DO: disegno?

Se riempissi tutto lo spazio intermedio con una lastra di isolante, si avrebbe $V_{\kappa}\neq0$. In particolare, si osserva che il potenziale tra le piastre in presenza di un isolante di spessore $s$ è
\begin{equation*}
	V(s)=\left(V_{\kappa}-V_0\right)\frac{s}{d}+V_0
\end{equation*} 
dove $V_{\kappa}$ indica il potenziale per $s=d$ (condensatore pieno di isolante) e $V_0$ quello per $s=0$  (condensatore vuoto).\\
Sperimentalmente, si trova che il rapporto tra la \ddp $V_0$ misurata con il condensatore vuoto e quella $V_{\kappa}$ con il condensatore riempito di isolante è \textit{caratteristico} del \textit{tipo} di materiale, ma non dipende dalla geometria o dalla carica delle armature.
\begin{define}[Costante dielettrica relativa e suscettibilità elettrica del dielettrico]
	La \textbf{costante dielettrica relativa}\index{costante!dielettrica!relativa} è il rapporto adimensionale
	\begin{equation}
		\kappa=\frac{V_0}{V_k}>1
	\end{equation}
	La grandezza
	\begin{equation}
		\chi=\kappa - 1>0
	\end{equation}
	viene detta \textbf{suscettibilità elettrica del dielettrico}\index{suscettibilità!elettrica del dielettrico}.
\end{define}
Maggiore è $\kappa$, maggiori sono le capacità conduttive del materiale: formalmente, un materiale è un \textit{conduttore perfetto} se $\kappa=+\infty$, ossia se $V_{\kappa}=V_C=0$.

La seguente tabella presenta alcuni materiali e le loro costanti dielettriche relative.
\begin{center}
	\begin{tabular}{ll}
		\textbf{Materiale}      & \textbf{Costante dielettrica relativa} $\kappa$\\\hline
		Aria           & \num{1,00059}                                                                    \\
		Acqua          & \num{80}                                                                         \\
		Alcool etilico & \num{28}                                                                         \\
		Ambra          & \num{2,5}                                                                        \\
		Bachelite      & \num{4,9}                                                                        \\
		Carta          & \num{3,7}                                                                        \\
		Polistirolo    & \num{2,6}                                                                        \\
		Porcellana     & \num{6,5}                                                                       
	\end{tabular}
\end{center}
Come per il caso del conduttore, la capacità \textit{aumenta}:
\begin{equation*}
	C_{\kappa}=\frac{q}{V_{\kappa}}=\frac{q\kappa}{V_0}=\kappa C_0
\end{equation*}
\begin{equation}
	C_{\kappa}=\kappa C_0>C_0
\end{equation}
\paragraph{Costante dielettrica assoluta}
Nel \refChapterOnly{leggediCoulomb} abbiamo definito la \textit{costante dielettrica del vuoto} $\epsilon_0$. Come era prevedibile dal nome, non è l'unica costante dielettrica: per trattare dei fenomeni elettromagnetici nei materiali ci conviene definire delle costanti, basate su $\epsilon_0$, che incorporano l'informazione sulla conducibilità elettrica data dalla costante dielettrica relativa $\kappa$.
\begin{define}[Costante dielettrica assoluta]
	La \textbf{costante dielettrica assoluta}\index{costante!dielettrica!assoluta} è definita come
	\begin{equation}
		\epsilon=\kappa\epsilon_0
	\end{equation}
\end{define}
Le formule che descrivono fenomeni elettrici nei materiali\footnote{Come già detto, questo vale solo per i dielettrici (lineari), che approfondiremo nella sezione XXX.} possono essere ottenute facilmente dal caso nel vuoto sostituendo a $\epsilon_0$ la costante assoluta $\epsilon$.
\begin{example}
	Per un condensatore piano nel vuoto si ha
	\begin{equation*}
		C_0=\frac{\epsilon_0\Sigma}{d}
	\end{equation*}
	Per un condensatore con un isolante (dielettrico) all'interno si ha
	\begin{equation*}
		C_{\kappa}=\kappa C_0=\frac{\kappa\epsilon_0\Sigma}{d}=\frac{\epsilon\Sigma}{d}
	\end{equation*}
\end{example}
\begin{observe}
	In generale, si può calcolare la costante dielettrica \textit{relativa} misurando la costante dielettrica \textit{assoluta} del materiale con un qualche \textit{esperimento opportuno} e dividendo per la costante dielettrica nel vuoto.
	\begin{equation}
		\kappa=\frac{\epsilon}{\epsilon_0}
	\end{equation}
\end{observe}
\paragraph{Campo elettrico nel condensatore con isolante all'interno}
Ritornando al condensatore completamente riempito di isolante, se il potenziale è minore il campo elettrico è minore del caso nel vuoto; si vede, infatti, che
\begin{equation*}
	E_{\kappa}=\frac{V_k}{d}=\frac{V_0}{\kappa d}=\frac{E_0}{\kappa}\leq E_0
\end{equation*} 
La variazione del campo elettrico dovuta alla presenza del materiale è
\begin{equation*}
	E_0-E_{\kappa}=\frac{V_0}{d}-\frac{V_0}{\kappa d}=\frac{V_0}{d}\frac{\kappa-1}{\kappa}=E_0\frac{\kappa - 1}{\kappa}
\end{equation*}
Se la densità di carica è $E_0=\dfrac{\sigma_0}{\epsilon_0}$, si osserva che 
\begin{equation}\label{campoelettricodielettrico}
	E_{\kappa}=E_0-\frac{\kappa-1}{\kappa}E_0=\frac{\sigma_0}{\epsilon_0}-\frac{\sigma_p}{\sigma_0}
\end{equation}
dove
\begin{equation}
	\sigma_p=\frac{\kappa-1}{\kappa}\sigma_0<\sigma_0
\end{equation}
La \eqref{campoelettricodielettrico} mostra come il campo elettrico all'interno del dielettrico si possa vedere come la \textit{sovrapposizione} di due campi elettrici nel vuoto, uno $E_0$ dovuto dalle cariche libere (distribuzione di carica $\sigma$) sulle armature, l'altro $E_p$ di intensità minore e generato da una distribuzione di carica $\sigma_p$. Le cariche che generano quest'ultimo si possono \textit{immaginare} come depositate sulle facce della \textit{lastra dielettrica}, con segno opposto a quello della carica libera sull'armatura continua - in modo per certi versi simile a quanto succede con i conduttori, ma in maniera ridotta.
% TO DO: insert immagine Quest'ultimo si può immaginare come
\section{Polarizzazione}
È noto che gli isolanti sono caratterizzati da una scarsa presenza di cariche libere, a differenza dei conduttori. Ciò farebbe presupporre che le cariche sulle facce che abbiamo immaginato nella sezione precedente come potenziale spiegazione del campo $E_p$ siano, per l'appunto, un \textit{lavoro di fantasia} della fervida immaginazione di un fisico.\\
In realtà tali cariche non sono per nulla fittizie, ma non sono lo stesso tipo di cariche libere presenti nei conduttori, bensì sono il risultato macroscopico di \textit{processi microscopici}, alla cui base stanno i fenomeni di \textit{polarizzazione}.

Negli isolanti, come appena detto, le cariche non sono particolarmente libere: quasi tutti gli elettroni sono legati agli atomi e non possono allontanarsi spontaneamente. Si può comunque, con l'azione di \textit{agenti esterni}, separare \textit{localmente} cariche positive e negative all'interno degli atomi - senza romperne quindi i legami - in modo da indurre una separazione di carica.

Il fenomeno della \textbf{polarizzazione}\index{polarizzazione} consiste proprio in questo: molto brevemente, esso consiste nel rendere degli atomi normalmente neutri in microscopici \textit{dipoli} sotto l'effetto di un campo elettrico esterno, in modo da causare il comportamento osservato in precedenza nei dielettrici. Non esiste un \textit{unico} modo di polarizzare atomi o molecole; noi ci occuperemo della \textit{polarizzazione elettronica} e della \textit{polarizzazione per orientamento}.
\paragraph{Polarizzazione elettronica}
Approfondiamo ora il primo tipo. Un atomo, secondo il modello non quantistico, consiste in un \textit{nucleo} positivo immerso in una \textit{nube} di elettroni negativi. In assenza di un campo elettrico esterno, il nucleo è neutro e la distribuzione degli elettroni attorno al nucleo è mediamente simmetrica, in modo che il centro di massa della nube coincida con la posizione del nucleo.

Introducendo il campo elettrico, la nube negativa subisce uno spostamento \textit{contro} il campo elettrico, mentre il nucleo positivo si sposta in senso \textit{concorde} al campo fino a raggiungere una nuova posizione di equilibrio in cui il campo elettrico è \textit{controbilanciato} dall'attrazione di \textit{dipolo} tra cariche di segno opposto.
% TO DO: add immagine
All'equilibrio, tra i due centri di cariche c'è una distanza $\vba{x}$, con cui definiamo il \textbf{momento di dipolo elettrico} della configurazione ottenuta.
\begin{equation}
	\vba{p}_a=q\vba{x}=Ze\vba{x}
\end{equation}
dove $Z$ è il numero di cariche nell'atomo e $\vba{x}$ va del centro di carica negativa a quello positiva - ossia nella direzione del campo elettrico.
\begin{observe}
	Si noti che nel singolo atomo tale spostamento è dell'ordine di $\num{10d-15}$, pari circa alle dimensioni del nucleo e quindi il momento di dipolo è \textit{davvero piccolo}. Tuttavia, poiché gli atomi per unità di volume sono un numero estremamente elevato, l'effetto complessivo in un materiale è invece \textit{misurabile}.
\end{observe}
La \textbf{polarizzazione per elettrizzazione}\index{polarizzazione!per elettrizzazione} funziona sinteticamente così: un atomo soggetto ad un campo elettrico esterno $\vba{E}$ acquista un momento di dipolo $\vba{p}_a$ elettrico microscopico, parallelo e concorde al campo $\vba{E}$.
\begin{observe}
	Per creare e mantenere questa distanza tra i centri di carica è necessaria dell'energia, fornita dal campo elettrico e che viene immagazzinata nel dipolo.
\end{observe}
\paragraph{Polarizzazione per orientamento}
Sebbene abbiamo visto come polarizzare degli atomi, ci sono alcune sostanze le cui molecole presentano già un \textit{momento di dipolo intrinseco}: questo avviene nel caso di molecole poliatomiche (che, non sorprendentemente, sono dette molecole \textbf{polari}\index{molecola!polare}) come l'\textit{acqua} $\left(\mathrm{H}_2\mathrm{O}\right)$ o l'\textit{anidride carbonica} $\left(\mathrm{CO}_2\right)$ in cui la distribuzione delle cariche - dovuta ai legami elettrostatici - è tale che il centro delle cariche negative \textit{non} coincide con quello positivo.

Tuttavia, in assenza di campo elettrico esterno i momenti di dipoli molecolari sono puramente \textit{casuali} a causa dell'agitazione termica che distrugge con urti eventuali configurazioni ordinate; il momento di dipolo medio è nullo.
\begin{equation*}
	\left<\vba{p}\right>=0
\end{equation*}
% TO DO: add image
In presenza di un campo elettrico $\vba{E}$ esterno, i momenti di dipoli si allineano con il campo a causa del momento delle forze, facendo sì che il momento di dipolo medio risulti non nullo.
\begin{equation*}
	\left<\vba{p}\right>\neq0
\end{equation*}
Ciò nonostante, l'orientamento delle molecole è soltanto \textit{parziale} perché disturbato dall'agitazione termica: se la temperatura è bassa e il campo è intenso allora aumentano le molecole allineate.

La \textbf{polarizzazione per orientamento}\index{polarizzazione!per orientamento} è quindi una polarizzazione basata sul fatto che le molecole polari sono intrinsecamente dei dipoli.
\paragraph{Dielettrici e isolanti}
Prima di spiegare come queste due polarizzazioni agiscono a livello macroscopico nei materiale dielettrici, dobbiamo effettivamente spiegare cosa sia un \textit{materiale dielettrico}.

Fino ad ora abbiamo utilizzato abbastanza interscambiabilmente il termine ‘‘isolante'' e ‘‘dielettrico'', ma \textit{non} sono sinonimi.
\begin{itemize}
	\item Gli isolanti non hanno (molti) elettroni liberi che si muovono spontaneamente. Di conseguenza, sono materiali che hanno un'alta \textit{resistività} e non scorre praticamente alcuna corrente se soggetto ad un campo esterno. Inoltre, òa costante dielettrica è minore per gli isolanti.
	\item I dielettrici sono materiali isolanti le cui particelle (atomi, molecole) sono facilmente soggette a fenomeni di polarizzazione. Pertanto, immagazzinano facilmente energia nei dipoli formati.
\end{itemize}
Nei fatti, sebbene tutti i dielettrici sono isolanti, \textit{non} tutti gli isolanti sono dielettrici. Se non specificato differentemente, quando parliamo di isolanti consideriamo sempre \textit{isolanti dielettrici}.
%https://www.quora.com/What-is-difference-between-insulator-and-dielectric-substanc
\paragraph{Polarizzazione del dielettrico}
I momenti dipoli dei singoli atomi o molecole sono \textit{microscopici}. Tuttavia, l'elevato numero di particelle per unità di volume e l'alta suscettibilità alla polarizzazione fa sì che nei dielettrici questi effetti si \textit{sovrappongono} e si abbia un risultato misurabile a livello \textit{macroscopico}.

In termini espliciti, in presenza di un campo elettrico esterno $\vba{E}$ ciascun atomi o particelle in un unità di volume $\Delta V$ del dielettrico acquistano un momento di dipolo $\left<\vba{p}\right>$, parallelo e concorde con $\vba{E}$.
\begin{equation*}
	\left<\vba{p}\right>=\frac{1}{N}\sum_{i=1}^N\vba{p}_i
\end{equation*}
Qui $N$ è il numero di atomi nel volume $\Delta V$. La \textbf{densità di polarizzazione}\index{densità!di polarizzazione} è quindi
\begin{equation}
	\vba{P}=\lim_{\Delta V\to 0}\frac{1}{\Delta V}\sum_{i=1}^{N}\vba{p}_i=n\left<\vba{p}\right>
\end{equation}
dove
\begin{equation*}
	n=\lim_{\Delta V\to 0}\frac{N}{\Delta V}
\end{equation*}
è la densità di particelle. Il vettore $\vba{P}$ è anche detto \textbf{vettore polarizzazione!del dielettrico} e caratterizza l'effetto di formazione dei momenti di dipolo indotti dal campo esterno.

La maggior parte dei \textit{dielettrici} soddisfano una legge di proporzionalità lineare tra la densità di dipolo e il campo elettrico:
\begin{equation}
	\vba{P}=\epsilon_0\left(\kappa-1\right)\vba{E}=\epsilon_0\chi\vba{E}
\end{equation}
I dielettrici che seguono tale legge sono detti \textbf{lineari}\index{lineari}: sono sostanze \textit{amorfe} con simmetria spaziale in tutte le direzioni (\textbf{isotropia spaziale}); in altre parole, \textit{non} ci sono direzioni preferenziali dovute \textit{a priori} dal materiale. I dielettrici \textit{non lineari}, come certi cristalli, sono invece anisotropi: $\vba{P}$ e $\vba{E}$ non sono necessariamente paralleli, ma sono su direzioni particolari dette \textit{assi cristallografici}. La suscettibilità elettrica, di conseguenza, non potrà essere rappresentata da un semplice numero, ma è rappresentata da un \textit{tensore}.
\begin{observe}
	Ecco spiegato il perché del termine ‘‘suscettibilità elettrica'': un materiale come acqua e alcol etilico hanno alta suscettibilità elettrica e sono proni a polarizzarsi fortemente, mentre altri come la carta o il polistirolo che hanno bassa suscettibilità tendono a polarizzarsi di meno. % TO DO: check se è vero con quei materiali
\end{observe}
\begin{example}
	Ricordiamo che nel caso del condensatore si aveva
	\begin{align*}
		E_{\kappa}=\frac{E_0}{\kappa}&\text{con } E_0=\frac{\sigma}{\epsilon_0 }
	\end{align*}
	% TO DO: image
	Allora, la densità di polarizzazione, in modulo, è
	\begin{equation}
		P=\epsilon_0\frac{\kappa-1}{\kappa}E_0=\sigma_0\frac{\kappa-1}{\kappa}=\sigma_p
	\end{equation}
Il vettore polarizzazione corrisponde alla densità (vettoriale) di cariche ‘‘fittizia'' definita precedentemente.
\end{example}
\subsection{Campo elettrico generato dalla polarizzazione}
Dopo aver visto come il vettore di polarizzazione sia legato ad un campo elettrico esterno, ci interessa capire come funziona il campo \textit{generato dal dielettrico polarizzato} e quale sia il legame con il vettore di polarizzazione.

Consideriamo un dielettrico \textit{polarizzato uniformemente}, ossia tale per cui $\vba{P}=\text{const}$ e supponiamo di suddividerlo in prismi infinitesimi di base $d\Sigma$, altezza $dh$ e volume $dV=d\Sigma dh$. Le cariche formano tanti dipoli elettrici, ciascuno pari a
\begin{equation*}
	d\vba{p}=\vba{P}dV=\abs{\vba{P}}d\Sigma d\vba{h}=dqd\vba{h}
\end{equation*}
dove $d\vba{h}$ è concorde con $\vba{P}$ e $dq$ è la carica interna al prisma. Ricordiamo che distribuzioni di cariche \textit{differenti} ma con stesso momento di dipolo sono esternamente \textit{indistinguibili} l'una dall'altra; è dunque perfettamente \textit{equivalente} rimpiazzare l'effetto di moltissimi dipoli microscopici interni al prisma $dV$ con un sistema costituito da due distribuzioni di cariche
\begin{equation*}
	\pm dq_p=\pm \abs{\vba{P}}d\Sigma
\end{equation*}
poste \textit{nel vuoto}, distanti $dH$ e distribuite sulle basi del prisma con densità
\begin{equation*}
	\pm\sigma_p=\frac{\pm dq_p}{d\Sigma}=\pm\frac{\abs{\vba{P}}\Ccancel[red]{d\Sigma}}{\Ccancel[red]{d\Sigma}}=\pm\abs{P}
\end{equation*}
% TO DO: immagine
Siccome supponiamo $\vba{P}$ uniforme su tutto il dielettrico, il vettore di polarizzazione è lo stesso per due prismi contigui. Di conseguenza, sulle superfici di contatto le cariche sono uguali e contrarie; se ripetiamo questo ragionamento con altri prismi contigui alle basi, le uniche cariche rimanenti che \textit{non} sono compensate sono solo quelle sulle basi dei primi che \textit{appartengono} alla superficie del dielettrico.\\
Quello che stiamo facendo è supporre che le cariche nel dielettrico, spostate \textit{localmente} dalle posizioni di equilibrio in quanto il materiale è polarizzato uniformemente, si compensino all'interno ma \textit{non} all'esterno dato che la superficie di bordo non permette ulteriori compensazioni. Le cariche si distribuiscono sulla superficie con densità
\begin{equation}
	\sigma_p=\vba{P}\vdot \vbh{u}_n
\end{equation}
dove $\vbh{u}_n$ è il versore normale alla superficie $\Sigma$ del materiale.
% TO DO: immagine
\begin{attention}
	Sebbene a tratti ciò possa ricordare il comportamento dei conduttori, il funzionamento è \textit{fondamentalmente} differente. Le cariche di polarizzazione non sono libere come nei conduttori e quelle che notiamo sulla superficie non sono elettroni che si sono raccolti lì da altre parti del materiale, ma sono gli elettroni \textit{già presenti superficialmente}: li notiamo solo in virtù degli \textit{spostamenti locali} negli atomi e nelle molecole.\\
	Questo è il motivo per cui non possiamo \textit{asportare un pezzo} di dielettrico e misurare le cariche superficiali, come potremmo fare ad esempio con un conduttore - il funzionamento è più vicino a quello che studieremo dei \textit{magnete}, da questo punto di vista.
\end{attention}
\noindent La carica - che avevamo definito ‘‘fittizia'' - in una particolare porzione di superficie $\Sigma_0$ è
\begin{equation}
	q=\int_{\Sigma_0}\sigma_pd\Sigma=\int_{\Sigma_0}\vba{P}\vdot \vbh{u}_nd\Sigma
	\end{equation}
\begin{observe}
	Se la polarizzazione è uniforme non si manifestano cariche all'interno del dielettrico, quindi la carica totale sulla superficie \textit{deve} essere nulla:
	\begin{equation*}
		0=\int_{\Sigma} \sigma_p d\Sigma=\int_\Sigma \vba{P}\vdot\vbh{u}_nd\Sigma
	\end{equation*}
	Applicando il teorema della divergenza si ottiene che
	\begin{equation}
		\int_{V}\div{\vba{P}}dV=0
	\end{equation}
\end{observe}
Se il vettore di polarizzazione \textit{non} è uniforme, la carica \textit{non} si distribuisce solo sulla superficie. Consideriamo sempre la suddivisione in prismi infinitesimi e studiamo il valore della carica sulla base comune a due prismi contigui, con asse parallelo all'asse $z$ e area di base $d\Sigma=dxdy$.
La carica su una superficie infinitesima è
\begin{equation*}
	dq(z)=\vba{P}\vdot \vbh{u}_nd\Sigma
\end{equation*}
Ricordiamo che il versore $\vbh{u}_n$ lo prendiamo orientato verso l'esterno della superficie; nel nostro caso, il versore
% TO DO: guardare il mazzoldi che preferisco quella spiegazione
Si ha quindi che la densità di carica nel dielettrico dovuta alla polarizzazione è
\begin{equation}
	\rho_p=-\div{\vba{P}}
\end{equation}
Nel caso di $\vba{P}$ uniforme si ha
\begin{equation*}
	0=q_to=\int_V=\rho_pdV=-\int_V\div{\vba{P}}dV+\int_{\partial V}\vba{P}\vdot\vbh{u}_n d\Sigma=0
\end{equation*}
\section{Equazioni dell'elettrostatica nel dielettrico}
Siamo ora in grado di formulare le equazioni dell'elettrostatica nei materiali dielettrici. Consideriamo un campo elettrostatico $\vba{E}$ che attraversa un materiale dielettrico, inducendo un vettore di polarizzazione $\vba{P}$.\\
Mentre il rotore del campo elettrico rimane nullo...
\begin{equation}
	\grad{E}=0
\end{equation}
... la sua diverga risulta pari alla carica complessiva nel materiale, diviso per $\epsilon_0$ - ma tale carica è pari alla somma della carica $\rho$ già presente e della carica da polarizzazione $\rho$:
\begin{equation*}
	\div{vba{E}}=\frac{\rho+\rho_p}{\epsilon_0}=\frac{\rho}{\epsilon_0}-\frac{\div{\vba{P}}}{\epsilon_0}\implies \div{\epsilon_0\vba{E}+\vba{P}}=\rho
\end{equation*}
Definito il \textbf{campo elettrostatico di induzione dielettrica}\index{campo!elettrostatico!di induzione dielettrica}
\begin{equation}
	\vba{D}=\epsilon_0\vba{E}+\vba{P}
\end{equation}
otteniamo la legge
\begin{equation}
	\div{\vba{D}}=\rho\label{SecondaMaxwellDIelettriciUno}
\end{equation}
Nel caso dei dielettrici lineari, ricordiamo che
\begin{equation*}
	\vba{P}=\epsilon_0\left(\kappa-1\right)\vba{E}=\epsilon_0\chi\vba{E}
\end{equation*}
da cui
\begin{equation*}
	\vba{D}=\kappa\epsilon_0\vba{E}=\epsilon\vba{E}
\end{equation*}
ed equivalentemente, la legge \eqref{SecondaMaxwellDIelettriciUno} si può riscrivere come
\begin{equation}
	\div{\vba{E}}=\frac{\rho}{\epsilon}
\end{equation} 
\begin{observe}
 Come abbiamo osservato in altri casi lavorando con i dielettrici \textit{lineari}, l'ultima legge è pari all'analoga equazione dell'elettrostatica nel vuoto a cui abbiamo sostituito a $\epsilon_0$ la costante assoluta $\epsilon$.
\end{observe}