% SVN info for this file
\svnidlong
{$HeadURL$}
{$LastChangedDate$}
{$LastChangedRevision$}
{$LastChangedBy$}

\chapter{Elettrodinamica relativistica}
\labelChapter{elettrodinamica relativistica}

\begin{introduction}
	‘‘Chiunque sa cos'è una curva, fino a quando non si ha studiato abbastanza Matematica per confondersi con tutte le innumerevoli eccezioni esistenti.''
	\begin{flushright}
		\textsc{Felix Klein,} dopo aver studiato troppa Matematica.
	\end{flushright}
\end{introduction}
\lettrine[findent=1pt, nindent=0pt]{N}{ella} 

Partiamo dalle equazioni di Maxwell
	\div\vba E=4\pi\rho
	\curl\vba B-\frac{1}{c}\pdv{\vba E}{t}=\frac{4\pi}{c}\vba j
	\curl\vba E+\frac{!}{c}\pdv{\vba B}{t}=0
	\div\vba B=0
	
	\abs{\vba F_{Coul}}=\frac{q_1q_2}{r_{12}^2}
	\vba F_{Lor}=q\left(\vba E+\frac{1}{c}\vba v\times \vba B\right)
	equazione di continuità \pdv{\rho}{t}=-\div\vba j
	
ad occhi non sembrano relativistiche
ispirati dall'ultima equazione, 
\begin{enumerate}
	\item abbiamo un quadrivettore covariante 
		\pdv{x^\mu}=\partial_\mu=\left(\frac{1}{c}\pdv{t}, \grad\right) \implies \partial_\mu\partial\equiv \square =\frac{1}{c^2}\pdv[2]{t}-\laplacian
	
	lorentz invariante: onde che si propagano alla velocità della luce
	\item 	l'elemento di volume dello spazio di Minkowski è Lorentz invariante
		d^4x=dx^0dx^1dx^2dx^3
		d^4x'=\abs{\pdv{(x_0',x_1',x_2',x_3')}{(x_0,x_1,x_2,x_3)}}d^4x=\abs{\det \Gamma}d^4x
		
		il tempo si dilata con fttore \gamma, lo spazio si resttringe con fattore 1/\gamma, assieme non succede nulla
	\item	la carica elettrica è Lorentz invariante: si vede da un fatto sperimentale
	\item 	equazione di continuità
		\pdv{\rho}{t}+\div\vba j=0 \implies (\rho,\vba j)\equiv j^\mu
		fa sospettare che densità di corrente si trasforma come il tempo da 3)
			dq=\rho d^3, ma dq è lorentz invariante, quindi diventa Lorentz invariante anche \rho d^3x se ammetto che \rho si trasforma come il tempo
		abbiamo così un altro quadrivettore j^\mu
		l'equazione di continuità divente: la quadridivergenza della quadricorrente è nulla
			\partial_\mu j^\mu=0
		non ho segno - perché di natura sono già contro e covariante
\end{enumerate}

accumulata l'eq di continuità \partial_\mu j^\mu=0, j^\mu=(c\rho,\vba j)
le equazioni di Maxwell nel gauge di Lorentz per i potenziali \vba A e \phi 
	\begin{cases}
		\square\vba A=\frac{4\pi}{c}\vba j\\
		\square\phi=4\pi\rho 
	\end{cases}
	con \frac{1}{c}\pdv{\phi}{t}+\div\vba A=0
	
\square Lorentz invariante, quello a cui li applico sono quadrivettori controvarianti
il gauge di Lorentz è scritto automaticament in maniera invariante per Lorentz

otteniamo così i quadrivettori 
	A^\mu\equiv(\phi,\vba A) quadrivettore controvariante
	\begin{itemize}
		\item 	\partial_\mu j^\mu=0
		\item 	\square A^\mu=\frac{4\pi}{c} j^\mu
		\item 	\partial_\mu A^\mu=0
	\end{itemize}

vediamo come si mescolano campo elettrico e magnetico, rivedendo come li abbiamo definiti

	(\vba E)^i=\left(-\frac{1}{c}\pdv{\vba A}{t}-\grad\phi \right)^i= -\partial_0 A^i-  \partial_i A^0=
	perché il gradiente nasce covariante, ora però voglio tutti gli indici in alto o in basso
	=-\partial^0 A^i+\partial^iA^0
magnetico
	(\vba B)^i=(\curl\vba A)^i \implies B^1=\partial_2A^3-\partial_3A^2=-\partial^2^3+\partial^3A^3
	B^2=\partial_3A^1 -\partial_1A^3= -\partial^3A^1+\partial^1A^3
	B^3=\partial_1A^2-\partial_2A^1= -\partial^1A^2+\partial^2A^1
	
cioè si organizzano in maniera naturale in un tensore antisimmetrico, cioè T^{\mu\nu}=-T^{\nu\mu}

definiamo F^{\mu\nu}\equiv \partial^\mu A^\nu-\partial^\nu A^\mu, controvariante e antisimmetrico

	F^{0i}=\partial^0A^i -\partial^iA^0 =-E^i
	F^{12}=-B^3
	F^{13}=B^2
	F^{23}=-B^1
	
in rappresentazione matriciale è 
	F=\begin{pmatrix}
		0 & -E^1 & -E^2 & -E^3\\
		E^1 & 0 & -B^3 & B^2 \\
		E^2 & B^3 & 0 & -B^1\\
		E^3 & -B^2 & B^1 & 0
	\end{pmatrix}
	F_{\mu\nu}=g_{\mu\rho}g_{\nu\sigma}F^{\rho\sigma} è lo stesso con $E$ scambiato con $-E$


abbiamo così 
\begin{itemize}
	\item 	\partial_\mu j^\mu=0
	\item 	\square A^\mu=\frac{4\pi}{c} j^\mu
	\item 	\partial_\mu A^\mu=0
	\item	F^{\mu\nu} = \partial^\mu A^\nu -\partial^\nu A^\mu
\end{itemize}
c'è un unico oggetto geometrico nello spazio, trasf lorentz cambia ogni indice
unico oggetto le cui proprietà di trasformazione sono semplici

le equazinoi di Maxwell saranno, iniziando da quelle omogenee (che consentono di scrivere F^{\mu\nu} = \partial^\mu A^\nu -\partial^\nu A^\mu, che sarà la soluzione)
	\begin{itemize}
		\item	\div\vba B=0\\
		\item 	\curl\vba E+\frac{1}{c}\pdv{\vba B}{t}=0 
			 \partial_1B^1+\partial_2B^"+\partial_3B^3=0
			 =-\partial_1F_{23}+\partial_2F_{13}-\partial_3F_{12}==
%CIT ha profumo antisimmetrico
			componente 1: \partial_0B^1+\partial_2E^3 -\partial_3E^2=0
						=\partial_0F_{32}+\partial_2F_{03}+\partial_3F_{20}
					suggerisce che ho permutazioin che sommano su permutazinoi cicliche, in galileiane avevamo \epsilon^{ijk}
					c'è un oggetto quadridimensionale che funziona così? sì, è \epsilon^{\mu\nu\rho\sigma}			 
			==-\epsilon^{0\nu\rho\sigma}\partial_\nu F_{\rho\sigma}
			componente 1
				\epsilon^{1\nu\rho\sigma}\partial_\nu F_{\rho\sigma}=0
			accorpiamo tutto e le equazioni di Maxwell omogenee saranno
				\epsilon^{\mu\nu\rho\sigma}\partial_\nu F_{\rho\sigma}=0
	\end{\itemize}

%TODO: sistemare l'itemiza insignificante
se F_{\mu\nu}=\partial_\mu A_\nu  -\partial_\nu A_\mu\implies \epsilon^{\mu\nu\rho\sigma}\partial_\nu (\partial_\rho A_\rho -\partial_\sigma A^(??))=0

primo termine simmetrico quando tensore antisimmetrico e viceversa, quindi fa 0

\begin{define}
	Tenore elettromagnetico duale contratto con tensore antisimmetrico	
		\tilde F^{\mu\nu}\equiv \frac{1}{2}\epsilon^{\mu\nu\rho\sigma} F_{\rho\sigma}= \begin{pmatrix}
			0 & -B^1 & -B^2 & -B^3 \\
			B^1 & 0 & E^3 & -E^2 \\
			B^2 & -E^3 & 0 & E^1 \\
			B^3 & E^2 & - E^1 & 0
		\end{pmatrix}
	ottnueta scambiando campi elettrici e magnetici
\end{define}


EQUAZIONI NPN OMOGENEE

	\div\vba B=4\pi\rho
	\partial_1 F^{10}+\partial_2F^{20} +\partial_3 F^{30}=\frac{4\pi}{c}j^0
	
	\curl\vba B-\frac{1}{c}\partial{\vba E}{t}=\frac{4\pi}{c}\avb j
	\partial_2F^{21} +\partial_3F^{31}+\partial_0F^{01}=\frac{4\pi}{c}j^1 	componente 1
	
	\partial_\mu F^{\mu\nu}=\frac{4\pi}{c}j^\nu
	
	possiamo riscriverla come \partial_\mu \tilde{F}^{\mu\nu}=0
	da \epsilon^{\mu\nu\rho\sigma}\partial_\nu F_{\rho\sigma}=0
	
se ci fosse una corrente magnetica avremmo simmetria completa fra scambio di campi elettrici e magnetici, ma fisicamente non c'è per assenza di monopoli magnetici


FORZA DI lORENTZ 
	\vba F\equiv\dv{\vba p}{t}=q\left(\vba E+\frac{\vba v}{2}\times\vba B\right)
lineare nei campi, quindi lineare di \vba F contratto a qualcosa: lineare nella velocità quindi quadrivelocità
quindi la quadriforza
	F^\mu\equiv \dv{\p^\mu}{\tau}=\frac{q}{c}F^{\mu\nu}u_\nu
	
	
quarta equazione
	F^0=\frac{q}{c}F^{0i}u_i= -\frac{q}{c}\gamma \vba E\vdot \vba v= \gamma \dv{p^0}{t}=\frac{\gamma}{c}\dv{E}{t}
informazione su lavoro ed energia
la variazione di energia della particella è determinata dal lavoro fatto dal campo elettrico, quindi il campo magnetico non fa lavoro
le componenti spaziali danno la forza di Lorentz, la componente temporale 	lavoro fatto solo dal campo elettrico

l'intera fisica contemporanea è scritta come una generalizzazione di questo




%%%%%%%%%%%%%%%%%%%%%%%%%%%%%%%%%%%%%%%%%%%%%%%%%%%%%%%%%%%%%%%%%%%%%%%%%%%%%%%%%%%%%%%%%%%%%%%5

%LEZ 35 23/05/2022


la densità di energia della scatola per volume della scatola è 
	\rho(\nu)d\nu\equiv 8\pi k_BT\frac{\nu^2}{c^3}d\nu
	RayleighJeans
la costante di Boltzmann è k_B=1.38*10^{-23}JK^{-1}
	\frac{\nu^2}{c^3}d\nu viene dala pura anlisi dimensionale
	\rho(\nu)d\nu=\bar{E}(T)n(\nu)d\nu, per Boltzmann k_BT per il teorema/pricipio di equipartizione della meccanica statistica, e vedremo che è falsa
	
Se siamo in una dimensione il volume di una dimensione è una lunghezza, il numero di onde stacionarie lì dentro N, 
	L=N\frac{\lambda}{2}\implies N(\nu)=\frac{2L\nu}{c}, n(\nu)=\frac{2\nu}{c}
	
in dimensione 2, D=2, siccome stiamo parlando di lunghezze d'onda molto piccole basta considerare cubi n-dimensionali. Qui consideriamo un quadrato con lato di lunghezza $L$. prendiamo una qualunque direzione e lì c'è un'onda stazionaria, cioè che quale che sia la sua direzione ci devono stare un numero intero di semilunghezze d'onda. Dobbiamo farlo per due direzioni perchè abbiamo due angoli \theta_1 e \theta_2
ci sono delle onde piane, quindi ci saranno dei piani che sono perpendicolari e ce ne devono stare un numero intero di semilunghezze d'onda
chiamiamo il primo punto in cui si annulla B, lo proiettiamo sugli assi e le proiezioni saranno B' e B"
impongo la condizione da D=! per due volte
	\bar{AB]=\frac{\lambda}{2} per definizione = per triangoli rettangoli \bar{AB'} \cos\theta_1=\bar{AB"}\cos\theta_2
	\bar{AB'}=\frac{\lambda}{2}\frac{1}{\cos\theta_2}
	\bar{AB"}=\frac{\lambda}{2}\frac{1}{\cos\theta_2}
	
proiettiamo sui due assi
	L=n_1\frac{\lambda}{2}\rfac{1}{\cos\theta_1}=n_2\fracè\lambda}{2}\frac{1}{\cos\theta_2}
troviamo condizione che non dipende da n_1 ed n_2 quadrando
	n_1^2+n_2^2=4\frac{\nu^2}{c^2}\cos^2\theta_1 +4\frac{\nu^2}{c^2}\underbrace{\cos^\theta_2}_{\sin^2\theta_1}= \frac{4\nu^2}{c^2}L^2
adesso si può fare in dimensioni qualsiasi: diventa solo una fila di equazioni con cui si fa lo stesso giochetto
	D=3, n_1^2+n_2^2+n_3^2=\frac{4\nu^2 L^2}{c^2}
	uso i coseni direttori e tutti assieme sommano sempre ad 1, cioè \sum_i\cos^2\theta_i=1
è a frequenza fissata
se mettiamo un d\nu ambo i membri, quanti ce ne sono^ in quanti modi posso costruire quella somma in modo che venga quella roba?
son i numeri di punti a coordinate intere dentro il primo ottante di una corona sferica di raggio \frac{2\nu L}{c} a spessore d\nu
possiamo calcolarlo: 
	N(\nu)d\nu=
il numero di punti a coordinate intere è il volume, tranne per gli effetti di bordo
se però la sfera è gigantesca ed il reticolo a coordinate intere è molto fine ho un'eccellente approssimazione del volume
-> N(\nu)d\nu è il volume del primo ottante di una corona sferica
	N(\nu)d\nu=\frac{1}{8} 4\pi\left(\frac{2\nu L*{c}\right)^2 \frac{2L}{c}d\nu
v bene perché è proporzionale al volume, il fattore 2 mancante è dovuto agli stati di polarizzazione
	2(2polarizzazioni)4\pi\fracé\nu^2*{c^3}L^3d\nu 
	
torniamo così a quanto scritto prima
	\rho_{RJ}(\nu)d\nu=8\pi k_BT\frac{\nu^2}{c^3}d\nu per il teorema di equipartizione
questa è un catastrofe ultravioletta

ogni volta che (:::) è kT abbiamo infiniti gradi di libertà

interpretazione _empirica_ di Plien(?), dipende anche dalla frequenza
	\rho_W(\nu,T)\equiv a\nu^3 e^{\frac{-b\nu}{T}}
è una formula empirica che funziona per grandi frequenze
è stato fatto fittando i dati
è sbagliata!!!!!!!!!!!!!!!!!!!!!!!!!!!!!!!!!!!!!!!!!!!!!!!!


Planck costruisce una formula analitica che fittasse ad alte frequenze e basse frequenze
cercò una derivazione teorica di questa formula che funzionava bene
richiese però un'ipotesi rivoluzionaria, che all'inizio venne spiegata solo come un artificio matematico
%CIT non è detto che siano veri, però sono molto utili. poi scoprirono che erano veri
il problema è che ci sono un numero infinito di modi in cui vivono
possiamo semplificarlo se l'energia dipende dalla frequenza e la sequeunza di energie è discreta

IPOTESI DI PLANCK: $E$ non ha uno spettro continuo, ma solo valori discreti sono possibili
	E\longleftrightarrow E_n(\nu)= nE_1(\nu=nh\nu)
energie che dipensono dalla frequenza, tutti multipli di una singola energia fondamentale che è una costante di proporzionalità per la frequenza: h detta costante di Planck

viene introdotta una nuova costante cje è fondamentale della natura, comparira nella formulazione del corpo nero
	h=6.626*10^{-34}m^2*kg/s che sono unità di azione posizione per impulso o energia per tempo
	
le energie possibili per una radiazione elettromagnetica
%CIT allora quando arriva il tecnico gli diciamo che un fisico teorico se l'è cavata

abbiamo \rho_{RJ}, E=k_BT medio è sbagliato e va rifatto

finché non introduciamo una costante per ragioni dimensionali E non può dipendere da \nu
	E(\nu,T)=\frac{\sum_{n=0}^{+\infty} E_n e^{-\beta E_n} }{\sum_{n=0}^{+\infty} e^{-\beta E_n} }==
ho rinormalizato, al denominatore lo 0 conta perché è un'energia possibile
uso un trucco
	==-\dv{\beta} \log\left(\sum_{n=0}^\infty e^{-\beta E_n}\right) = (E_n è proporzionale a n, so fare la somma) =-\dv{\beta}\log\left(\sum_{n=0}^\infty \left(e^{-\beta h\nu}\right)^n  \right)= -\dv{\beta}\log \frac{1}{1-e^{-\beta h\nu}} =\dv{\beta}\log\left(1-e^[-\beta h\nu]\right)= \frac{h\nu e^{-\beta h\nu}{1-e^{-\beta h\nu}}=\frac{h\nu}{e^{\beta h\nu}-1}
abbiamo anche h\nu enrgia
posso usaro per ricavare dimensionalità
essendo numero puro posso costruire qualunque funzione
in particolare succede che 
	\rho_P(\nu)d\nu=\frac{8\pi\nu^2}{c^3}\frac{h\nu]{e^{\frac{h\nu}{k_BT}}-1}d\nu
scriviamola in funzione della frequenza invece che della lunghezza d'onda usndo \lambda=\fracéc}{\nu}
	\tilde{\rho_P}(\lambda)d\lambda= \frac{8\pi hc}{\lambda^5} \frac{1}{e^{\frac{hc}{k_BT\lambda}}-1}d\lambda 
mando \nu\rightarrow\infty ritrovo Wiern (?)
mando \nu\rightarrow 0 ritrovo RJ


Preso Boltzmann stefan E_{TOT}=\sigma T^4 irradiata basta integrare
	E_{TOT}=\int_=^\infty d\nu I(\nu, T)= \frac{c}{4}\frac{8\pi h}{c^3} \int_0^\infty d\nu \frac{\nu^3}{e^{\frac{h\nu}{k_B T} } -1}
detti A=\fracè2\pi h}c^2, B=\frac{h}{k_BT}
	=A\int_0^\infty d\nu \frac{\nu^3 e^{-B\nu}}{1-e^{-B\nu}} =A\int_0^\infty d\nu \nu^3 e^{-B\nu } \sum_{n=0 }^\infty \left(e^{-B\nu}\right)^n
%CIT: sono cose che ai vostri colleghi fisici non direi: la seria converge, l'integrale converge, è legale tirare fuori la serie 
	=A\sum_{n=1}^\infty \int_0^\infty d\nu \nu^3 e^{-Bn\nu} =
%CIT: lo chiamo con un nome qualunque, tipo x
	=(per la \Gamma di 4 )\frac{6A}{B^4} \sum_{n=1}^\infty \frac{1}{n^4}= \frac{6A}{B^4}\zeta(4)=\frac{6A}{B^4}\frac{\pi^4}{90} \equiv \sigma T^4
ho una nuova costante data solo da costanti della natura
	\sigma=\frac{2}{15}\pi^5 \frac{k_B^4}{h^3c^2}
%CIT credo che sia in quel momento che sentì un brividino, il profumo dell'aria fresca di Stoccolma nel suo ufficio

siamo così arrivati a dire che la radiazione elettromagnetica "si presenta" sotto forma di "quanti" indivisibili di energia, oggi chiamati fotoni che possono essere emessi o riassorbiti dalla materia  solo in multipli interi

perché non ce se ne era accorti prima? h è molto piccolo, lo spettro della luce in generale è ancora continuo perché è continuo lo spettro delle frequenze
me ne accorgo solo se la frequenza è esattamente fissata e l'energia è un multiplo intero
con una luce bianca invece ho molte più frequenze 
capacità di lavorare con fasci di luce monocromatica, ovvero di fissata lunghezza d'onda

come produciamo una luce monocromatica? fascio di luce con prisma di vetro fatto bene 
ogni direzione in uscita dal prisma corrisponde a particolare frequenze d'onda, si mette una fessura per fare uscire solo quella


\section{Effetto fotoelettrico}
Einstein, 1905

data una lastra di metallo, se la irradiamo di radiazioni elettromagnetiche
%TODO IMMAGINE
possiamo vederlo solo con luce monocromatica, se prendo la luce bianca no
usando una radiazione monocromatica a frequenza \nu e lunghezza d'onda \lambda=\frac{c}{\nu}

succede che
\begin{enuemrate}
	\item	il numero di fotone emetti è proporzionale all'intensità della radiazione, il che va d'accordo con la meccanica classica
	\item 	al di sotto di una frequenza critica \nu_0 che dipende dal materiale irradiato, non viene fuori niente! non vengono emessi elettroni
	\item	la massima energia cinetica degli elettroni usciti \textit{non} dipende dall'intensità della radiazione
	quando escono elettroni posso misurarne l'energia, che arriva fino ad un certo punto
\end{enuemrate}

per risolvere basta prendere seriamente l'idea di Planck: multipli di energia per fissata frequenza 

li elettroni in stati legati hanno energia negativa E<0 (metto a 0 l'energia di un elettrone che va ad infinito (?) e vengono catturati
	E=-V_0
tradizionalmente viene detto potenziale d ionizzazione che dipende dal materiale

Solo radiazione con frequenza \nu tale che h\nu>V_0 può estrarre un elettrone dal metallo
un elettrone sottoposto a frequenza può subire solo h\nu 
non è un effetto cumulativo, sono assorbiti uno per volta

stabilito questo, l'energia cinetica massima degli elettroni è l'energia data 
	h\nu -V_0>0
così \nu_0=\frac{v_0}{h}	

è conferma che l'interazione avviene per scambio di pacchetti interi di energia

ha un'interessante rilevanza pratica: c'è nascosto il fatto che per quanto ne sappiamo non esiste nessun meccanismo fisico noto per cui l'utilizzo di telefoni cellulari causi tumori, perché lavorano nella banda delle onde radio (..), i fotoni emessi non hanno energia sufficiente per ionizzare 
una radiazione che non è ionizzante non è ionizzante mai: come una macchina non prende fuoco cse ci butto addosso l'acqua, ma se butto la benzina sì


%%%%%%%%%%%%%%%%%%%%%%%%%%%%%%%%%%%%%%%%%%%%%%%%%%%%%%%%%%%%%%%%%%%%%%%%%%%%%%%%%%%%%%%%
%LEZ 36, 25/05/2022

[...]

1920, esperimenti di Compton osservano \lambda_{fin}=\lambda_{fin}(\theta)
la radiazione luminosa è fatta di particelle: hanno anche un impulso: fotone che sbatte contro elettrone

\gamma come fotone
i fotoni a bassa energia si comportano quasi sempre come maniera semiclassica: onde
se energie sono alte tende e prevalere il loro aspetto particellare
alta erngia era indicata con raggi \gamma, da qui il nome

dall'altra parte arriva un elettrone
se lo considero come un urto fra particella il fotone e l'elettrone saranno riflessi perché c'è un impulso da conservare
l'angolo di scattering del processo di urto è \theta

scrivo la conservazione dell'energia e dell'impulso
scelgo l'energia del fotone (onda monocromatica)
	E_\gamma=h\nu
	\abs{\vba P_\gamma}=(da relazione fra energia e impulso data da relatività ristretta E=\sqrt{m^2c^4+\abs{\vba p}^2c^2}, m_\gamma=0) \frac{E_\gamma}{c}=\frac{h\nu}{c}=\frac{h}{\lambda}

impongo la conservazinoe delle quantità prima e dopo l'urto
\begin{cases}
	E_\gamma+E_e=E_\gamma'+E_e'		\\
	\vba p_\gamma+\vba p_e=\vba p_\gamma'+\vba p_e'
\end{cases}
scelgo il sistema di riposo dell'elettrone prima dell'urto: come se considerassi un elettrone fermo
in questo sistema qui \vba p_e=0 \implies E_e=m_e c^2
è per forza calcolo relativistico perché ho cose che vanno alla velocità della luce
e il sistema diventa
\begin{cases}
	E_\gamma+E_e=E_\gamma'+E_e'		\\
	\vba p_\gamma+\cancel{\vba p_e}=\vba p_\gamma'+\vba p_e'
\end{cases}

riscirvo per togleire la radice quadrata

	E_\gamma-E_\gamma'+m_ec^2= (energia dell'elettrone dopo l'urto) \sqrt{m_e^2c^4+c^2\abs{\vba p_e'}^2}= \sqrt{m_e^2c^4+c^2\abs{\vba p_\gamma-\vba p_\gamma'}^2} = (uso che \abs{\vba p_\gamma}c=E_\gamma e \abs{\vba p'_\gamma}c=E'_\gamma)  \sqrt{m_e^2c^4+ E_\gamma^2 - E_\gamma'^2+2E_\gamma E_\gamma'\cos\theta}
faccio il quadrato dei membri
	\implies 2m_ec^2\left(E_\gamma-E_\gamma'\right) -2E_\gamma E_\gamma' = -2E_\gamma E_\gamma'\cos\theta 
	
finora erano particelle ma sono fotoni ed ho relazione fra energia lunghezza d'onda e frequenza: E_\gamma=\frac{hc}{\lambda}
	hm_ec^3\left(\frac{1}{\lambda} -\frac{1}{\lambda'}\right)= h^2c^2\frac{1}{\lambda\lambda'}(1-\cos\theta)
	\lambda' (lunghezza d'onda del fotone dopo l'urto) -\lambda=\frac{h}{m_ec}(1-\cos\theta)
che corrisponde all'ooservazione sperimentale di Compton
\frac{h}{m_ec} è detta lunghezza d'onda Compton dell'elettrone



le evidenze si accumulano: in una serie di circostanze la radiazione elettromagnetica si può vedere come gas di particelle molto organizzato: tutte stessa direzione ed energia perché hanno la stessa frequenza
fascio di particella ad energia fissata


4° ESPERIMENTO: spettroscopia atomica
qualcosa di non continuo nella struttura della materia: dopo Planck è quantizzato

lo spettro di lunghezze d'onda non è continuo, ho solo determinate frequenze e quindi lunghezze d'onda
la luce emessa da elementi (gas) sottoposti a scariche elettriche ha uno spettro discreto
%CIT che non è un fantasma poco invadente
se proietto sullo schermo quello che esce dal prisma noto che ho delle righe spettrali con posizioni caratteristiche
metto l lunghezza d'onda espresse inunità caratteristiche \lambda(A°)
%tODO: cercare come si fa la A con la ° sopra
%CIT divetò un gigantesco parco giochi per gli sperimentali
%CIT questa cosa è inquietante: i nostri modelli teorici corrispondono perfettamente: non si vincono i Noble spiegando qualcosa spiegato già da altri

Rydberg quantifica questa cosa
per l'idorgeno queste righe spettrali si trovano come
	\frac{1}{\lambda}=\frac{\nu}{c}=R\left(\frac{1}{n_2^2}-\frac{1]{n_1^2}\right)
una serie corrisponde a n_2=1, abbiamo n_1=2,3,4,5	:serie di Lyman
	n_2=2, n_1=3,4,5, 	serie di Balmer
	n_3=3, n_1=4,5,6	serie di Ritz
	
R viene trovato con precisione come R\sim 1,097*10^5 cm^{-1}
questo aveva a che fare come la struttura della materia interagisce con la luce
luce che illumina i gas e vedo cosa assorbono, dall'altra parte gli atomi assorbono luce alle stesse frequenze
ifrogeno assorbe radiazione sulle stesse frequenze alle quali le emette



\section{Modelli atomici}

1907/8
com'è possibile che gli atomi siano neutri e non ci sono respingimenti di forze elettromagnetici forti?
Thompson scopre l'elettrone che ha carica negativa
qunidi suggerisce un modello qualitativo "plum pudding"
distriuzione di carica positiva di roba poco densa con dentro sassolini elettroni
atomo come distribuzione continua di carica positiva con dentro dei puntini, delle particelle puntiformi con carica negativa 

però non è possibile nella fisica classica
gli sperimentali si buttarono a verificare queste cose

Rutherford con gli studenti Gerger e Marsden
i primi anni in cui si lavorava con materiale radioattivo, usano partielle \alpha
fanno un urto di particelle \lpha, che oggia sappimao essere nuclei di elio, qunii due neutroni e due protoni
essendo particella carica si può pillotare con campi elettrici
molti materiali radioattivi decadono come \alpha

caso di assassinio da spionaggio governativo: si sospettano movimenti radioattivi
il polonio è estremamente radioattivo, deve durare abbastanza da poterla trasportare, se estremamente radioattiva tende a decadere: il polonio ha una vita media di 40 giorni. inoltre decade \alpha: è decadimento pericoloso solo se ingerito, perché sono fermate anche da foglio di carta s enon sono molto energetiche: se la ingerisco invece distrugge gli organi interni, dalla pelle è respinto

mettiamo l'oro, elemento pesante per cui la carica positiva fosse rilevabile
oro è malleabile e sottile
mettiamo uno schermo per vedere dove sbattono

se Thompson avesse avuto ragione non sarebbe successo nulla: le particelle \alpha avrebbero tirato dirtto: come sparare nella marmellata


nella stragrande maggioranza dei casi succede come diceva Thompson, ma in altri il proiettile bazooka viene deviato invece di attraversare marmellata
il proiettile viene rimbalzato

quello osservato è consistente con un urto di cariche positive
ma la carica positiva deve essere estremamente concentrata

la carica positiva, oggi detto nucleo atomico, è estremamente concentrato

abbiamo così il modello atomico di Rutherford 1911/13
c'è un nucleo carico positivamente estremamente piccolo e contiene circa tutta la massa dell'atomo
la densità \rho_{nucl} della materia nucleare è dell'ordine di 10^{11}kg/cm^3= 10^8 Tonn/cm^3
questo già ci dice quanto devono essere le dimensioni relative: deve esserci il vuoto
rapporto fra raggio tipico di un atomo e di un nucleo
	\frac{R_{atomo}}{R_{nucleo}}\sim 10^5
nucleo di 1 cm gli elettroni sono ai giardini reali
%CIT perché non sprofondate? gli elettroni dei vostri pantaloni respingono quelli della sedia, siamo su campi elettromagnetici
le cariche elettriche sottoposte ad accelerazione emettono energia

se nella meccanica classica faccio come Rutherford e come i pianeti (modello planetario) il problema è che l'elettrone è soggetto ad accelerazione centripeta, per conseguenza delle eq di Maxwell cariche elettriche accelerate irradiano energia: quindi dovremmo vedere l'energia, quindi l'orbita si restringe e decade
/anche nella gravità ho emissione di onde gravitazionali, che però sono piccole/
quindi l'atomo di Rutherford non è stabile: dopo 15 picosecondi è collassato: proibita dall'elettrodinamica classica

bisogna introdurre delle modificazioni drastiche a livello atomico

nel 1913 il giovane Niels Bohr
scriveremo la versione data da Sommerfield
l'idea di base come postulato è che esistono orbite stabili per gli elettroni (dato sperimentale) nelle quali non possono irradiare, perdere energia
quindi non obbediscono alle equazioni di Maxwell
ma l'energia sotto forma di radiazione elm viene emessa solo nelle transizioni da un'orbita stabile all'altra
bisogna scrivere il vincolo: cosa fissa le orbite stabili?
Sommerfeld scrive nel contesto nella meccanica hamiltoniana
se abbiamo un sistema periodico possiamo integrare il momento coniugato
	\oint p_i dq_i=n h
multiplo intero di una costante

Bohr immagina orbite circolare: la forza elettrostatica l'ambaradan di Keplero funziona
consideriamo orbite circolari
%CIT: io e Giotto siamo parenti - dopo aver disegnato un cerchio perfetto al primo tentativo alla lavagna

l'impuls dell'elettrone per un'orbita circolare è perpendicolare a \vba r
in questo caso dq=rd\phi
	\oint pdq= (velocità costante esce)2\pi pr= 2\pi mvr riconosciuto come valore del momento angolare dl'elettrone = 2\pi L_z
il momento angolare di questo elettrone è quantizzato
	Lz=n\frac{h}{2\pi}= n\hslash 
	
le orbite discrete: ci sono raggi che sono permessi ed altri che non lo sono
	L_z= l_n=mv_nr_n
neanche tutti i raggi e velocità sono permessi
anche le energie saranno dipendenti da n: cinetiche
	E=E_n=\frac{1}{2}mv_n^2-\frac{e^2}{r_n}
funziona per tutti atomi idrogenoidi: gira solo un elettrone


l'accelerazione è puramente centripeta e corrisponde alla forza coulombiana
	\frac{ze^2}{r_n^2}=m\frac{v_n^2}{r_n}
	\implies mv_n^2=\frac{ze^2}{r_n}
sostituendo nell'energia
	= -\frac{1}{2}\frac{ze^2}{r_e}
	
ma L_x=n\hslash 
quindi	 v_n=\frac{n\hslash}{mr_n}
possiamo riscrivere tutto in termini di costanti fondamentali
	mv_n^2=\frac{ze^2}{r_n}\implies m\frac{n^2\hslash^2}{m^2r_n^2} =\frac{ze^2}{r_n}
	\implies r_n=\frac{\hsalash^2}{mze^2}n^2
ritrovo tutto con grandezze fondamentali	
ho raggio minimo fondamentale con n=1 e tutti gli altri possibili raggi crescono coi quadrati

avendo quantizzato i raggi posso quantizzare anche le energie
	E_n= -\frac{mz^2e^4}{2\hslash^2}\frac{1}{n^2}
 abbiamo uno spettro discreto per l'energia: sono stati legati da segni -
 poi abbiamo abbastanza energia da liberare l'elettrone e quindi ionizzare l'atomo
 il livello successivo le righe si addensano e al di sopra di 0 lo spettro diventa continuo	
	
la nozione di traiettoria per un elettrone viene pi abbandonata in meccanica quantistica
per ragioni misteriose prende correzioni solo relativistiche, sopravvive a tutto il resto

adesso date queste ipotesi è numeri è chiaro perché la luce ha periodi di frequenza
perché ho livelli energetici occupabili dagli elettroni
tutti i salti possibili ci danno tutte le possibili energie e per Planck frequenze emissibili dall'atomo
l'atomo assorbe la stessa energia che emette

cosa succede se un elettrone salta da un livello all'altro?

	\nu_{n_2n_1}=\frac{c}{\lambda_{n_2n_1}=\frac{E_{n_1}-E_{n_2}}{h}
salto da livello più alto a più basso che corrisponde a quella differenza di energia
	\frac{1}{\lambda_{n_2n_1}=\frac{1}{hc}\frac{mz^2e^4}{2\hslash^2}\left(\frac{1}{n_2^2 -\frac{1}{n_1}^2\right)
per z=1 costante di hildberg (?)
	R=\frac{me^4}{4\pi\hslash^3 c}
%CIT trovata questa costante ha capito che avrebbe avuto una carriera in fisica teorica

quest però non è sistematico, indica solo che è la via giusta
si cerca di generalizzare

NOMENCLATURA
§ raggio di Bohr: r_1, detto a_0=\frac{\hslash^2}{me^2}\sim 0,5 A°
sono le dimensioni caratteristiche della roba tomica
§ costante di struttura fine: si chiama perché le righe spettrali si sdoppiano o triplicano: la struttura interna delle righe spettrlai è detta struttra fine, ora è nota come costante di accoppiamento dell'elettrodinamica su cui si sviluppa la seria di Taylor 			
	\alpha=\frac{e^2}{\hslash c} \sim \frac{1}{137}	numero puro
§ lunghezza d'onda compton dell'elettrone 	\lambda_c=\frac{h}{m_ec}\sim 2.4*10^{-10}cm
§massa dell'elettrone in unità di energia 	 m_ec^20.5111*10^{-6}erg= 0.511 MeV (megaelettronvolt)

raggio n-esima orbita	r_n=a_0 n^2
energia n-esima orbita	E_n=-\frac{e^2}{2a_0}{\frac{1}{n^2}
	%CIT se le guardate assieme hanno un sapore coulombiano
	R=\frac{\alpha}{4\pi a_0}
energia dello stato fondamentale, minimo di energia
	E_1=-\frac{e^2}{2a_0}=\frac{me^4}{2\hslash^2}=-\frac{\alpha^2}{2}mc^2\sim  -13,6 eV
l'energia potenziale o cinetica di un elettrone che orbita è enormemente più piccola di quando è a riposo
rappresenta l'ordine di grandezza delle energie necessarie per ionizzare atomi
partendo da quel numoer posso vedere quali sono le radiazioni che ionizzano
con la luce visibile non posso ionizzare: quanti con poca energia


%%%%%%%%%%%%%%%%%%%%%%%%%%%%%%%%%%%%%%%%%%%%%%%%%%%%%%%%%%%%%%%%%%%%%%%%%%%%%%%%%%%%%%%%%%%%%%%%%%%
%LEZ 37, 26/05/2022

%%ESAMI 
%scritto vale anche per l'appello successivo nella stessa sessione, da discutere se anche per la sessione successiva
%numero di esercizi: variabile, dipende dalla lunghezza degli esercizi
%siccome molti non fanno l'orale potrebbe avere anche delle domande di teoria, anche incorporate negli esercizi
%non varia la difficoltà complessiva
%voto unico dello scritto, così come lo scritto in sé 
%no formulario, però le formule molto derivate sono fornite nel testo dell'esercizio
%%CIT se copiate vi diamo fuoco
%da guardare vecchi scritti
2 ore di scritto
%anche chi non ha intenzione di dare l'orale ci si dovrebbe iscrivere e basta informare in qualunque modo che si vuole solo la conferma

abbiamo elencato una serie di esperimenti che suggerivano una cosa sorprendente: la luce, sebbene fosse dimostrato da 200 anni che ha un comportamento ondulatorio, ha anche una natura colpuscolare
modello atomico di Bohr, che riguarda comportamento degli elettroni: il modello atomico prima era corpuscolare
l'equazione del moto era di un corpuscolo in un cmapo elm
Bohr (con Sommerfeld) invece disse che la componente z del momento angolare è quantizzata
	L_z=n\hslash \implies (porta alla determinazione corretta della costante di hilgert(?) che è data solo da costanti della natura) R
questo è strano: è la componente di un vettore che assume solo certi valori: suggerisce che possa avere solo certe inclinazioni, il che non sembra avere senso perché tutto è invariante per rotazioni per sistemi di riferimento
quindi non stiamo capendo qualcosa di profondo
sperimentali: se c'è qualcosa di speciale nel momento angolare dovremmo vederlo anche con altri esperimenti con fasci di particelle ad esempio, con proprietà di momento magnetico: dipolo magnetico in un campo magnetico fa cose: tende ad allinearsi o moto di precessione attorno alla direzione del campo magnetico

Sperimentali: Stern e Gerlach, 1922
fascio di particelle tipo atomi o elettroni da far passare in campi elm , ma voglio misurare momento magnetico: oscurato su na particella carica perché effetti dei campi mgn sono più piccoli di quelli magnetici per un fattore $c$
quindi decidono di lavorare con atomi neutri, d'Argento Ag
l'idea è che dato un foglio con una fesura, lo si argenta come i gioielli, scaldandolo atomi di argento tendono ad essere liberati ed escono in direzioni assortite da un foro, serve un collimatore per avere un fascio
faccio passare fra dipoli di un magnete che va fatto in modo tale che il campo magnetico non sia costante
imponiamo le direzioni degli assi
	\pdv{B_z}{z}\neq 0
impongo che dipoli magnetici Nord e Sud non siano simmetrici, così le linee di forza del campo magnetico non sono omogeneee/simmetriche
hanno proprietà: esistenza vettore momento magnetico proporzionale a momento angolare dell'elettrone che gira intorno all'atomo
	\vba \mu\alpha\vba L 
%TODO: \alpha è di proporzionalità, non ricordo il simbolo
ma a quale elettrone ci stiamo riferendo?
l'argento è un metallo, e per principio di esclusione di Pauli gli elettroni stanno in orbite successive, non possono stare tutti nella stessa orbita: eccetto che per spin su e giù
in un atomo di argento o metallo: tutti gli orbitali che sono medi orbitano a zero: solo l'ultimo gira, il momento angolare si riferisce a quest'ultimo che gira
non è del tutto vero: l'elettrone ha uno spin che corrisponde ad un altro momento angolare, bisognerebbe capire come interagiscono
c'è una cosa che sta ruotando o si comporta come se stesse ruotando
l'eergia associata a questa interazione è
	\vba \mu \rightarrow \vba \mu\vdot\vba B
	è l'energia potenziale di interazione fra un momento magnetico ed un campo magnetico \vba B esterno
siccome è potenziale, la forza è il gradiente del potenziale
	F_z=\mu_z\pdv{B}{z}
immaginando che le altre componenti del gradiente siano nulle o non ci interessino

nella situazione descritta sopra nell'immagine
ci sarà una deflessione: se \mu_z positivo verso il basso (o alto?)
ci aspettiamo una deflessione delle particelle, mettiamo uno schermo al fondo: sono particelle neutre, in qualche modo è possibile
cogliamo guardare la distribuzione di come gli atomi colpiscono lo schermo
	F_z\alpha L_z\pdv{B}{z}
in una situazione classica siccome non abbiamo fatto nulla ai momenti
ci dovrebbe essere una sorta di distribuzione continua, sarebbero possibili tutte le orientazioni
ci sono effetti ai bordi che sopprimono
ci aspetteremmo una distribuzione (tipo gaussiana I guess, ma non lo è,  ha solo un picco) N(z)
situazione classica
%TODO: IMMAGINE distribuzione con assi z N(z)
invece quello che vediamo, in particolare per atomi di argento, una distribuzione con due picchi molto marcati
questo è possibile solo che L_z non ha un continuum di valori ma può prenderne soltanto due

la deduzione di questo esperimento è che per questo atomo L_z può assumere solo due valori
se cambiamo atomo può succedere che ci siano più picchi, comunque finiti e discreti
stiamo credendo di misurare le possibili orientazioni di un vettore nello spazio
quello che sembra emergere è che le possibili orientazioni non sono continue ma discrete

possiamo ruotare l'apparato e tenendo fisso succede sempre questo: non importa quale componente del momento angolare misuriamo esso è effettivamente quantizzato

Come Bohr aveva suggerito parlando di orbite degli elettroni, il momento angolare è quantizzato
il compito della meccanica quantistica è mostrare che non sono vettori ma qualcos'altro
una volta fatto questo esperimento in cui spezziamo in fasci in positivo e negativo(?)
una serie di esperimenti su uno dei due fasci

indico nella scatola quale direzione sto guardando
SG sta per esperimento di Stern Gerlach
posso iterare l'esperimento
siccome ho già misurato la componente z di questi atomi mi aspetto che non ci sia la componente negativa

cosa succede invece se misuro il momento angolare in un'altra direzione?
SG_z	L_z +	SG_x
ottengo due fasci L_x + ed L_x -
	posso fare questo esperimento anche con i fotoni, che possono avere solo due polarizzazioni: circolari oppure orizzontali
	gli SG sarebbero come le polaroid
aggiungiamo uno SG_z
ci aspettiamo che misurare in componente x sia irrilevante: ma in realtà li abbiamo perturbati, fa resuscitare le componenti con L_z -

ho una distruzione di informazione
non è possibile avere informazioni complete sulle componenti del momento angolare
è un'istanza del principio di indeterminazione: di solito viene presentato con posizione e velocità, serve lunghezza d'onda dell'ordine di grandezza della particella, quindi frequenze molto alte, serve molta energia
non si possono misurare contemporaneamente le componenti del (così pensato) momento angolare

l'atomo dei Bohr però inizia a farci sospettare
finora abbiamo scoperto che ciò che si pensava fossero solo onde in realtà hanno anche comportamento corpuscolare
ora sta per succedere il contrario!
l'elettrone ha anche proprietà ondulatorie

Louis De Broglie

%CIT le nuove generazioni sono più drastiche, tranne forse la vostra: lo dovete dimostrare
quando parliamo di fotoni
	Planck 	E=h\nu=\frac{h}{T}=\hslash \omega 	con \omega=\frac{2\pi}{T}=2\pi\nu frequenza angolare
	Einstein	E=pc= (Maxwell)\frac{hc}{\lambda}	
	\implies p=\frac{h}{\lambda}=\hslash k	con k=\frac{2\pi}{\lambda} numero d'onda
compaiono naturalmente nelle trasformate di Fourier

ora arriva De Broglie
	L_z=2\pi mvr\underbrace{=}_{Bohr} nh
	ma 2\pi r è la lunghezza dell'orbita \mathcal{L}
	\implies p\mathcal{L}=nh
se immagino che posso associare all'elettrone un'onda: se chiedo che dentro alla circonferenza ci stia un'onda stazionari
	cioè \mathcal{L}=n\lambda
	\implies p=\frac{h}{\lambda}
c'è un modo fisico di capire quest'idea di quantizzazione?
De Broglie suggerisce che così come le onde elm hanno una natura corpuscolare così li elettroni hanno una natura ondulatoria: di Bohr stabili perché lì ci associo un'onda stazionaria, che è intrinsecamente stabile

quindi 	p=\frac{h}{\lambda}

per fotoni p=\frac{E}{c}
per particella materiali E=\frac{p^2}{2m}


si apre un gigantesco dibattito: nelle sedi istituzionali è finto nel 1930, ma ancora ci si pensa
l'elettrone è spalmato: come l'energia ha densità di energia, la quale è spalmate, diffusa
cosa ne facciamo di questa particella massiva? la spalmiamo?
no, perché posso colpire l'elettrone in un punto specifico

la lunghezza d'onda \lambda è enormemente piccola
come lo vediamo?
se vogliamo vedere una figura di diffrazione classica dovrei avere distanza dell'ordine della lunghezza d'onda: si può fare ma ci vuole molto tempo

fu visto dall'esperimento di Davisson e Germer, 1927, dopo formulazione della meccanica quantistica da Heisenberg e Schroedinger
osservarono la diffrazione in un fascio di elettroni utilizzando cristalli come reticoli di diffrazione
a valle sullo schermo abbiamo figure di diffrazione caratteristiche
abbiamo una prova sperimentale che è diventata sempre più precisa che gli elettroni in realtà hanno una natura ondulatoria: rimane da capire come interpretare questo fatto

dal pov concettuale vediamo una cosa che non corrisponde nè alla diffrazione classica di un'onda elm nè allo sparare proiettili in modo probabilistico

la relatività ristretta è controintuitiva ma alla fine si capisce
questo invece non si capisce, è semplicemente come vanno le cose, poi ci si abitua

\section{L'esperimento delle 2 fenditure}
storiella scritta nel volume terzo delle lezioni di Feynman

lo facciamo 3 volte: biglie, onde elm ed elettroni

\subsection{Biglie}
abbiamo un cannoncino non precisissimo, abbiamo una parete con due fenditure e più in là uno schermo rilevatore di biglie
le dimensioni delle biglie sono circa uguali alle dimensioni delle fenditure
quando una biglia passa interagisce con la fenditura e la mira cambia, così una distribuzione di probabilità
l'idea è che queste imprecisioni possono essere curate
misuriamo il numero di proiettili che arrivano alla parete in funzione di z
posto un asse z e la probabilità di trovare una biglia nel punto z
se chiudo una delle due fenditure (la 2) molte saranno non deflesse ed abbiamo picco P_1(z)
facciamo il contrario ed abbiamo una curva analoga P_2(z)
se lascio tutto aperto faccio la somma delle curve
\begin{enumerate}
	\item	l'arrivo delle biglie è un fenomeno discreto: arrivano solo biglie intere
	\item	tutte le biglie che arrivano o sono passate da 1 o sono passate da 2 ognuna con la propria distribuzione
		P(z)=P_1(z)+P_2(z)
\end{enumerate}

\subsection{Onde elettromagnetiche}
potremmo fare lo stesso esperimento con qualsiasi tipo di onda, anche quelle del mare
abbiamo una sorgente di onde monocromatiche di qualche tipo
ci sono onde che si propagano con frequenza e lunghezza d'onda prefissata
dall'altra parte della fenditura abbiamo un fenomeno di diffrazione
vale il principio di Heygens: ogni fenditura si comporta a sua volta come una sorgente: sommo sui contributi del fronte d'onda precedente
abbiamo un fenomeno di interferenza
mettiamo un rivelatore sullo schermo: può misurare l'energia deposta dall'onda elm deposta lì
oppure l'oscillazione del galleggiante, misura dell'energia trasportata dall'onda

se chiudiamo una fessura misuriamo l'intensità I(z), abbiamo I_1(z) e I_2(z) simmetriche perché non abbiamo interferenza
se invece apriamo entrambe le fenditure avremo dei punti con interferenza costruttiva e altri con interferenza distruttiva
	I(z)\equiv \abs{A(z)}^2, A ampiezza ha un segno
	I_{tot}(z)=\abs{A_1(z)+A_2(z)}^2
l'energia è conservata perché viene solo spostata
abbiamo così una figura di distribuzione del tipo
è una quantità continua
\begin{enumerate}
	\item	L'energia è depositata in modo continuo, non è quantizzata
	\item	c'è interferenza
			I_{tot}(z)\neq I_1(z)+I_2(z)
			I_{1,2}(z)=\abs{A_{1,2}(z)}^2
			I_{tot}(z)=\abs{A_1(z)+A_2(z)}^2
		non ha senso dire quale onda è passata da quale fenditura: tutte le onde sono passate da tutte e due le fenditure: l'energia passa da entrambe le parti
\end{enumerate}

\subsection{Elettroni}
filo scaldato, collimatore

se chiudo una delle due fenditure abbiamo una distribuzione come prima
è letteralmente il numero di elettroni che arrivano, è analogo al processo delle biglie: arriva sempre un elettrone intero
idem con l'altra fenditura

\begin{enumerate}
	\item	arrivano sempre elettroni interi
	\item	la distribuzione di energia è quantizzata
	\item	N_{tot}\neq N_1(z)+N_2(z)
		abbiamo un'interferenza: suggerisce un fenomeno ondulatorio
		N(z)=\abs{A(z)}^2
		N_{tot}(z)=\abs{A_1(z)+A_2(z)}^2
\end{enumerate}

se apriamo entrambe le fenditure, si forma ua figura di interferenza, non è la somma di N_1 ed N_2

le affermazioni sembrano contraddirsi!
non ha senso porsi la domanda da dove sia passato l'elettrone, per quale fenditura

se so da quale fenditura è passato non ho figura di interferenza e ritrovo biglie
se il rilevatore è più fine da non disturbare l'elettrone esso scappa ed abbiamo di nuovo l'interferenza
possiamo fare questo elettrone anche uno per volta
l'osservare perturba, ma non potremmo mai essere abbastanza furbi da non interferire

interpretazione di Copenaghen della meccanica quantistica: l'onda associata all'elettrone non è di materia ma è un'onda di probabilità di trovarsi sullo schermo quando facciamo la misura
la probabilità è il modulo al quadrato dell'ampiezza, di un'onda

elettroni
associo onda
onda dice distribuzione di probabilità
onda esibisce fenomeno di interferenza
cerco equazione delle onde che vada bene e soddisfi vincoli fisici ed energetici etc
l'equazione delle onde che lo fa è quella di Schroedinger


l'ampiezza è interpretata come ampiezza di probabilità
%%%%%%%%%%%%%%%%%%%%%%%%%%%%%%%%%%%%%%%%%%%%%%%%%%%%%%%%%%%%%%%%%%%%%%%%%%%%%%%%%%%%%%%%%%%%%
%%%%%%%%%%%%%%%%%%%%%%%%%%%%%%%%%%%%%%%%%%%%%%%%%%%%%%%%%%%%%%%%%%%%%%%%%%%%%%%%%%%%%%%%%%%%%
%%%%%%%%%%%%%%%%%%%%%%%%%%%%%%%%%%%%%%%%%%%%%%%%%%%%%%%%%%%%%%%%%%%%%%%%%%%%%%%%%%%%%%%%%%%%%
%%%%%%%%%%%%%%%%%%%%%%%%%%%%%%%%%%%%%%%%%%%%%%%%%%%%%%%%%%%%%%%%%%%%%%%%%%%%%%%%%%%%%%%%%%%%%
%%%%%%%%%%%%%%%%%%%%%%%%%%%%%%%%%%%%%%%%%%%%%%%%%%%%%%%%%%%%%%%%%%%%%%%%%%%%%%%%%%%%%%%%%%%%%
%%%%%%%%%%%%%%%%%%%%%%%%%%%%%%%%%%%%%%%%%%%%%%%%%%%%%%%%%%%%%%%%%%%%%%%%%%%%%%%%%%%%%%%%%%%%%
%FINE































































































