% SVN info for this file
\svnidlong
{$HeadURL$}
{$LastChangedDate$}
{$LastChangedRevision$}
{$LastChangedBy$}

\chapter{Elettrodinamica relativistica}
\labelChapter{elettrodinamica relativistica}

\begin{introduction}
	‘‘Chiunque sa cos'è una curva, fino a quando non si ha studiato abbastanza Matematica per confondersi con tutte le innumerevoli eccezioni esistenti.''
	\begin{flushright}
		\textsc{Felix Klein,} dopo aver studiato troppa Matematica.
	\end{flushright}
\end{introduction}
\lettrine[findent=1pt, nindent=0pt]{N}{ella} 

Partiamo dalle equazioni di Maxwell
	\div\vba E=4\pi\rho
	\curl\vba B-\frac{1}{c}\pdv{\vba E}{t}=\frac{4\pi}{c}\vba j
	\curl\vba E+\frac{!}{c}\pdv{\vba B}{t}=0
	\div\vba B=0
	
	\abs{\vba F_{Coul}}=\frac{q_1q_2}{r_{12}^2}
	\vba F_{Lor}=q\left(\vba E+\frac{1}{c}\vba v\times \vba B\right)
	equazione di continuità \pdv{\rho}{t}=-\div\vba j
	
ad occhi non sembrano relativistiche
ispirati dall'ultima equazione, 
\begin{enumerate}
	\item abbiamo un quadrivettore covariante 
		\pdv{x^\mu}=\partial_\mu=\left(\frac{1}{c}\pdv{t}, \grad\right) \implies \partial_\mu\partial\equiv \square =\frac{1}{c^2}\pdv[2]{t}-\laplacian
	
	lorentz invariante: onde che si propagano alla velocità della luce
	\item 	l'elemento di volume dello spazio di Minkowski è Lorentz invariante
		d^4x=dx^0dx^1dx^2dx^3
		d^4x'=\abs{\pdv{(x_0',x_1',x_2',x_3')}{(x_0,x_1,x_2,x_3)}}d^4x=\abs{\det \Gamma}d^4x
		
		il tempo si dilata con fttore \gamma, lo spazio si resttringe con fattore 1/\gamma, assieme non succede nulla
	\item	la carica elettrica è Lorentz invariante: si vede da un fatto sperimentale
	\item 	equazione di continuità
		\pdv{\rho}{t}+\div\vba j=0 \implies (\rho,\vba j)\equiv j^\mu
		fa sospettare che densità di corrente si trasforma come il tempo da 3)
			dq=\rho d^3, ma dq è lorentz invariante, quindi diventa Lorentz invariante anche \rho d^3x se ammetto che \rho si trasforma come il tempo
		abbiamo così un altro quadrivettore j^\mu
		l'equazione di continuità divente: la quadridivergenza della quadricorrente è nulla
			\partial_\mu j^\mu=0
		non ho segno - perché di natura sono già contro e covariante
\end{enumerate}

accumulata l'eq di continuità \partial_\mu j^\mu=0, j^\mu=(c\rho,\vba j)
le equazioni di Maxwell nel gauge di Lorentz per i potenziali \vba A e \phi 
	\begin{cases}
		\square\vba A=\frac{4\pi}{c}\vba j\\
		\square\phi=4\pi\rho 
	\end{cases}
	con \frac{1}{c}\pdv{\phi}{t}+\div\vba A=0
	
\square Lorentz invariante, quello a cui li applico sono quadrivettori controvarianti
il gauge di Lorentz è scritto automaticament in maniera invariante per Lorentz

otteniamo così i quadrivettori 
	A^\mu\equiv(\phi,\vba A) quadrivettore controvariante
	\begin{itemize}
		\item 	\partial_\mu j^\mu=0
		\item 	\square A^\mu=\frac{4\pi}{c} j^\mu
		\item 	\partial_\mu A^\mu=0
	\end{itemize}

vediamo come si mescolano campo elettrico e magnetico, rivedendo come li abbiamo definiti

	(\vba E)^i=\left(-\frac{1}{c}\pdv{\vba A}{t}-\grad\phi \right)^i= -\partial_0 A^i-  \partial_i A^0=
	perché il gradiente nasce covariante, ora però voglio tutti gli indici in alto o in basso
	=-\partial^0 A^i+\partial^iA^0
magnetico
	(\vba B)^i=(\curl\vba A)^i \implies B^1=\partial_2A^3-\partial_3A^2=-\partial^2^3+\partial^3A^3
	B^2=\partial_3A^1 -\partial_1A^3= -\partial^3A^1+\partial^1A^3
	B^3=\partial_1A^2-\partial_2A^1= -\partial^1A^2+\partial^2A^1
	
cioè si organizzano in maniera naturale in un tensore antisimmetrico, cioè T^{\mu\nu}=-T^{\nu\mu}

definiamo F^{\mu\nu}\equiv \partial^\mu A^\nu-\partial^\nu A^\mu, controvariante e antisimmetrico

	F^{0i}=\partial^0A^i -\partial^iA^0 =-E^i
	F^{12}=-B^3
	F^{13}=B^2
	F^{23}=-B^1
	
in rappresentazione matriciale è 
	F=\begin{pmatrix}
		0 & -E^1 & -E^2 & -E^3\\
		E^1 & 0 & -B^3 & B^2 \\
		E^2 & B^3 & 0 & -B^1\\
		E^3 & -B^2 & B^1 & 0
	\end{pmatrix}
	F_{\mu\nu}=g_{\mu\rho}g_{\nu\sigma}F^{\rho\sigma} è lo stesso con $E$ scambiato con $-E$


abbiamo così 
\begin{itemize}
	\item 	\partial_\mu j^\mu=0
	\item 	\square A^\mu=\frac{4\pi}{c} j^\mu
	\item 	\partial_\mu A^\mu=0
	\item	F^{\mu\nu} = \partial^\mu A^\nu -\partial^\nu A^\mu
\end{itemize}
c'è un unico oggetto geometrico nello spazio, trasf lorentz cambia ogni indice
unico oggetto le cui proprietà di trasformazione sono semplici

le equazinoi di Maxwell saranno, iniziando da quelle omogenee (che consentono di scrivere F^{\mu\nu} = \partial^\mu A^\nu -\partial^\nu A^\mu, che sarà la soluzione)
	\begin{itemize}
		\item	\div\vba B=0\\
		\item 	\curl\vba E+\frac{1}{c}\pdv{\vba B}{t}=0 
			 \partial_1B^1+\partial_2B^"+\partial_3B^3=0
			 =-\partial_1F_{23}+\partial_2F_{13}-\partial_3F_{12}==
%CIT ha profumo antisimmetrico
			componente 1: \partial_0B^1+\partial_2E^3 -\partial_3E^2=0
						=\partial_0F_{32}+\partial_2F_{03}+\partial_3F_{20}
					suggerisce che ho permutazioin che sommano su permutazinoi cicliche, in galileiane avevamo \epsilon^{ijk}
					c'è un oggetto quadridimensionale che funziona così? sì, è \epsilon^{\mu\nu\rho\sigma}			 
			==-\epsilon^{0\nu\rho\sigma}\partial_\nu F_{\rho\sigma}
			componente 1
				\epsilon^{1\nu\rho\sigma}\partial_\nu F_{\rho\sigma}=0
			accorpiamo tutto e le equazioni di Maxwell omogenee saranno
				\epsilon^{\mu\nu\rho\sigma}\partial_\nu F_{\rho\sigma}=0
	\end{\itemize}

%TODO: sistemare l'itemiza insignificante
se F_{\mu\nu}=\partial_\mu A_\nu  -\partial_\nu A_\mu\implies \epsilon^{\mu\nu\rho\sigma}\partial_\nu (\partial_\rho A_\rho -\partial_\sigma A^(??))=0

primo termine simmetrico quando tensore antisimmetrico e viceversa, quindi fa 0

\begin{define}
	Tenore elettromagnetico duale contratto con tensore antisimmetrico	
		\tilde F^{\mu\nu}\equiv \frac{1}{2}\epsilon^{\mu\nu\rho\sigma} F_{\rho\sigma}= \begin{pmatrix}
			0 & -B^1 & -B^2 & -B^3 \\
			B^1 & 0 & E^3 & -E^2 \\
			B^2 & -E^3 & 0 & E^1 \\
			B^3 & E^2 & - E^1 & 0
		\end{pmatrix}
	ottnueta scambiando campi elettrici e magnetici
\end{define}


EQUAZIONI NPN OMOGENEE

	\div\vba B=4\pi\rho
	\partial_1 F^{10}+\partial_2F^{20} +\partial_3 F^{30}=\frac{4\pi}{c}j^0
	
	\curl\vba B-\frac{1}{c}\partial{\vba E}{t}=\frac{4\pi}{c}\avb j
	\partial_2F^{21} +\partial_3F^{31}+\partial_0F^{01}=\frac{4\pi}{c}j^1 	componente 1
	
	\partial_\mu F^{\mu\nu}=\frac{4\pi}{c}j^\nu
	
	possiamo riscriverla come \partial_\mu \tilde{F}^{\mu\nu}=0
	da \epsilon^{\mu\nu\rho\sigma}\partial_\nu F_{\rho\sigma}=0
	
se ci fosse una corrente magnetica avremmo simmetria completa fra scambio di campi elettrici e magnetici, ma fisicamente non c'è per assenza di monopoli magnetici


FORZA DI lORENTZ 
	\vba F\equiv\dv{\vba p}{t}=q\left(\vba E+\frac{\vba v}{2}\times\vba B\right)
lineare nei campi, quindi lineare di \vba F contratto a qualcosa: lineare nella velocità quindi quadrivelocità
quindi la quadriforza
	F^\mu\equiv \dv{\p^\mu}{\tau}=\frac{q}{c}F^{\mu\nu}u_\nu
	
	
quarta equazione
	F^0=\frac{q}{c}F^{0i}u_i= -\frac{q}{c}\gamma \vba E\vdot \vba v= \gamma \dv{p^0}{t}=\frac{\gamma}{c}\dv{E}{t}
informazione su lavoro ed energia
la variazione di energia della particella è determinata dal lavoro fatto dal campo elettrico, quindi il campo magnetico non fa lavoro
le componenti spaziali danno la forza di Lorentz, la componente temporale 	lavoro fatto solo dal campo elettrico

l'intera fisica contemporanea è scritta come una generalizzazione di questo




%%%%%%%%%%%%%%%%%%%%%%%%%%%%%%%%%%%%%%%%%%%%%%%%%%%%%%%%%%%%%%%%%%%%%%%%%%%%%%%%%%%%%%%%%%%%%%%5

%LEZ 34/5? 23/05/2022


la densità di energia della scatola per volume della scatola è 
	\rho(\nu)d\nu\equiv 8\pi k_BT\frac{\nu^2}{c^3}d\nu
	RayleighJeans
la costante di Boltzmann è k_B=1.38*10^{-23}JK^{-1}
	\frac{\nu^2}{c^3}d\nu viene dala pura anlisi dimensionale
	\rho(\nu)d\nu=\bar{E}(T)n(\nu)d\nu, per Boltzmann k_BT per il teorema/pricipio di equipartizione della meccanica statistica, e vedremo che è falsa
	
Se siamo in una dimensione il volume di una dimensione è una lunghezza, il numero di onde stacionarie lì dentro N, 
	L=N\frac{\lambda}{2}\implies N(\nu)=\frac{2L\nu}{c}, n(\nu)=\frac{2\nu}{c}
	
in dimensione 2, D=2, siccome stiamo parlando di lunghezze d'onda molto piccole basta considerare cubi n-dimensionali. Qui consideriamo un quadrato con lato di lunghezza $L$. prendiamo una qualunque direzione e lì c'è un'onda stazionaria, cioè che quale che sia la sua direzione ci devono stare un numero intero di semilunghezze d'onda. Dobbiamo farlo per due direzioni perchè abbiamo due angoli \theta_1 e \theta_2
ci sono delle onde piane, quindi ci saranno dei piani che sono perpendicolari e ce ne devono stare un numero intero di semilunghezze d'onda
chiamiamo il primo punto in cui si annulla B, lo proiettiamo sugli assi e le proiezioni saranno B' e B"
impongo la condizione da D=! per due volte
	\bar{AB]=\frac{\lambda}{2} per definizione = per triangoli rettangoli \bar{AB'} \cos\theta_1=\bar{AB"}\cos\theta_2
	\bar{AB'}=\frac{\lambda}{2}\frac{1}{\cos\theta_2}
	\bar{AB"}=\frac{\lambda}{2}\frac{1}{\cos\theta_2}
	
proiettiamo sui due assi
	L=n_1\frac{\lambda}{2}\rfac{1}{\cos\theta_1}=n_2\fracè\lambda}{2}\frac{1}{\cos\theta_2}
troviamo condizione che non dipende da n_1 ed n_2 quadrando
	n_1^2+n_2^2=4\frac{\nu^2}{c^2}\cos^2\theta_1 +4\frac{\nu^2}{c^2}\underbrace{\cos^\theta_2}_{\sin^2\theta_1}= \frac{4\nu^2}{c^2}L^2
adesso si può fare in dimensioni qualsiasi: diventa solo una fila di equazioni con cui si fa lo stesso giochetto
	D=3, n_1^2+n_2^2+n_3^2=\frac{4\nu^2 L^2}{c^2}
	uso i coseni direttori e tutti assieme sommano sempre ad 1, cioè \sum_i\cos^2\theta_i=1
è a frequenza fissata
se mettiamo un d\nu ambo i membri, quanti ce ne sono^ in quanti modi posso costruire quella somma in modo che venga quella roba?
son i numeri di punti a coordinate intere dentro il primo ottante di una corona sferica di raggio \frac{2\nu L}{c} a spessore d\nu
possiamo calcolarlo: 
	N(\nu)d\nu=
il numero di punti a coordinate intere è il volume, tranne per gli effetti di bordo
se però la sfera è gigantesca ed il reticolo a coordinate intere è molto fine ho un'eccellente approssimazione del volume
-> N(\nu)d\nu è il volume del primo ottante di una corona sferica
	N(\nu)d\nu=\frac{1}{8} 4\pi\left(\frac{2\nu L*{c}\right)^2 \frac{2L}{c}d\nu
v bene perché è proporzionale al volume, il fattore 2 mancante è dovuto agli stati di polarizzazione
	2(2polarizzazioni)4\pi\fracé\nu^2*{c^3}L^3d\nu 
	
torniamo così a quanto scritto prima
	\rho_{RJ}(\nu)d\nu=8\pi k_BT\frac{\nu^2}{c^3}d\nu per il teorema di equipartizione
questa è un catastrofe ultravioletta

ogni volta che (:::) è kT abbiamo infiniti gradi di libertà

interpretazione _empirica_ di Plien(?), dipende anche dalla frequenza
	\rho_W(\nu,T)\equiv a\nu^3 e^{\frac{-b\nu}{T}}
è una formula empirica che funziona per grandi frequenze
è stato fatto fittando i dati
è sbagliata!!!!!!!!!!!!!!!!!!!!!!!!!!!!!!!!!!!!!!!!!!!!!!!!


Planck costruisce una formula analitica che fittasse ad alte frequenze e basse frequenze
cercò una derivazione teorica di questa formula che funzionava bene
richiese però un'ipotesi rivoluzionaria, che all'inizio venne spiegata solo come un artificio matematico
%CIT non è detto che siano veri, però sono molto utili. poi scoprirono che erano veri
il problema è che ci sono un numero infinito di modi in cui vivono
possiamo semplificarlo se l'energia dipende dalla frequenza e la sequeunza di energie è discreta

IPOTESI DI PLANCK: $E$ non ha uno spettro continuo, ma solo valori discreti sono possibili
	E\longleftrightarrow E_n(\nu)= nE_1(\nu=nh\nu)
energie che dipensono dalla frequenza, tutti multipli di una singola energia fondamentale che è una costante di proporzionalità per la frequenza: h detta costante di Planck

viene introdotta una nuova costante cje è fondamentale della natura, comparira nella formulazione del corpo nero
	h=6.626*10^{-34}m^2*kg/s che sono unità di azione posizione per impulso o energia per tempo
	
le energie possibili per una radiazione elettromagnetica
%CIT allora quando arriva il tecnico gli diciamo che un fisico teorico se l'è cavata

abbiamo \rho_{RJ}, E=k_BT medio è sbagliato e va rifatto

finché non introduciamo una costante per ragioni dimensionali E non può dipendere da \nu
	E(\nu,T)=\frac{\sum_{n=0}^{+\infty} E_n e^{-\beta E_n} }{\sum_{n=0}^{+\infty} e^{-\beta E_n} }==
ho rinormalizato, al denominatore lo 0 conta perché è un'energia possibile
uso un trucco
	==-\dv{\beta} \log\left(\sum_{n=0}^\infty e^{-\beta E_n}\right) = (E_n è proporzionale a n, so fare la somma) =-\dv{\beta}\log\left(\sum_{n=0}^\infty \left(e^{-\beta h\nu}\right)^n  \right)= -\dv{\beta}\log \frac{1}{1-e^{-\beta h\nu}} =\dv{\beta}\log\left(1-e^[-\beta h\nu]\right)= \frac{h\nu e^{-\beta h\nu}{1-e^{-\beta h\nu}}=\frac{h\nu}{e^{\beta h\nu}-1}
abbiamo anche h\nu enrgia
posso usaro per ricavare dimensionalità
essendo numero puro posso costruire qualunque funzione
in particolare succede che 
	\rho_P(\nu)d\nu=\frac{8\pi\nu^2}{c^3}\frac{h\nu]{e^{\frac{h\nu}{k_BT}}-1}d\nu
scriviamola in funzione della frequenza invece che della lunghezza d'onda usndo \lambda=\fracéc}{\nu}
	\tilde{\rho_P}(\lambda)d\lambda= \frac{8\pi hc}{\lambda^5} \frac{1}{e^{\frac{hc}{k_BT\lambda}}-1}d\lambda 
mando \nu\rightarrow\infty ritrovo Wiern (?)
mando \nu\rightarrow 0 ritrovo RJ


Preso Boltzmann stefan E_{TOT}=\sigma T^4 irradiata basta integrare
	E_{TOT}=\int_=^\infty d\nu I(\nu, T)= \frac{c}{4}\frac{8\pi h}{c^3} \int_0^\infty d\nu \frac{\nu^3}{e^{\frac{h\nu}{k_B T} } -1}
detti A=\fracè2\pi h}c^2, B=\frac{h}{k_BT}
	=A\int_0^\infty d\nu \frac{\nu^3 e^{-B\nu}}{1-e^{-B\nu}} =A\int_0^\infty d\nu \nu^3 e^{-B\nu } \sum_{n=0 }^\infty \left(e^{-B\nu}\right)^n
%CIT: sono cose che ai vostri colleghi fisici non direi: la seria converge, l'integrale converge, è legale tirare fuori la serie 
	=A\sum_{n=1}^\infty \int_0^\infty d\nu \nu^3 e^{-Bn\nu} =
%CIT: lo chiamo con un nome qualunque, tipo x
	=(per la \Gamma di 4 )\frac{6A}{B^4} \sum_{n=1}^\infty \frac{1}{n^4}= \frac{6A}{B^4}\zeta(4)=\frac{6A}{B^4}\frac{\pi^4}{90} \equiv \sigma T^4
ho una nuova costante data solo da costanti della natura
	\sigma=\frac{2}{15}\pi^5 \frac{k_B^4}{h^3c^2}
%CIT credo che sia in quel momento che sentì un brividino, il profumo dell'aria fresca di Stoccolma nel suo ufficio

siamo così arrivati a dire che la radiazione elettromagnetica "si presenta" sotto forma di "quanti" indivisibili di energia, oggi chiamati fotoni che possono essere emessi o riassorbiti dalla materia  solo in multipli interi

perché non ce se ne era accorti prima? h è molto piccolo, lo spettro della luce in generale è ancora continuo perché è continuo lo spettro delle frequenze
me ne accorgo solo se la frequenza è esattamente fissata e l'energia è un multiplo intero
con una luce bianca invece ho molte più frequenze 
capacità di lavorare con fasci di luce monocromatica, ovvero di fissata lunghezza d'onda

come produciamo una luce monocromatica? fascio di luce con prisma di vetro fatto bene 
ogni direzione in uscita dal prisma corrisponde a particolare frequenze d'onda, si mette una fessura per fare uscire solo quella


\section{Effetto fotoelettrico}
Einstein, 1905

data una lastra di metallo, se la irradiamo di radiazioni elettromagnetiche
%TODO IMMAGINE
possiamo vederlo solo con luce monocromatica, se prendo la luce bianca no
usando una radiazione monocromatica a frequenza \nu e lunghezza d'onda \lambda=\frac{c}{\nu}

succede che
\begin{enuemrate}
	\item	il numero di fotone emetti è proporzionale all'intensità della radiazione, il che va d'accordo con la meccanica classica
	\item 	al di sotto di una frequenza critica \nu_0 che dipende dal materiale irradiato, non viene fuori niente! non vengono emessi elettroni
	\item	la massima energia cinetica degli elettroni usciti \textit{non} dipende dall'intensità della radiazione
	quando escono elettroni posso misurarne l'energia, che arriva fino ad un certo punto
\end{enuemrate}

per risolvere basta prendere seriamente l'idea di Planck: multipli di energia per fissata frequenza 

li elettroni in stati legati hanno energia negativa E<0 (metto a 0 l'energia di un elettrone che va ad infinito (?) e vengono catturati
	E=-V_0
tradizionalmente viene detto potenziale d ionizzazione che dipende dal materiale

Solo radiazione con frequenza \nu tale che h\nu>V_0 può estrarre un elettrone dal metallo
un elettrone sottoposto a frequenza può subire solo h\nu 
non è un effetto cumulativo, sono assorbiti uno per volta

stabilito questo, l'energia cinetica massima degli elettroni è l'energia data 
	h\nu -V_0>0
così \nu_0=\frac{v_0}{h}	

è conferma che l'interazione avviene per scambio di pacchetti interi di energia

ha un'interessante rilevanza pratica: c'è nascosto il fatto che per quanto ne sappiamo non esiste nessun meccanismo fisico noto per cui l'utilizzo di telefoni cellulari causi tumori, perché lavorano nella banda delle onde radio (..), i fotoni emessi non hanno energia sufficiente per ionizzare 
una radiazione che non è ionizzante non è ionizzante mai: come una macchina non prende fuoco cse ci butto addosso l'acqua, ma se butto la benzina sì




























































































